\chapter{Fundamentação Teórica}
\label{cap2}


\paragraph{} Este capítulo busca fornecer ao leitor alguns insumos para uma melhor contextualização do trabalho. Serão abordadas as dinâmicas básicas de funcionamento do mercado, seguida por uma introdução de ML, um tipo específico de análise e alguns índices relavantes a este estudo. Por fim, uma revisão bibliográfica é realizada.



\FloatBarrier
\section{Mercado de Capitais, Bolsa de Valores e Ações}

\paragraph{} O Mercado de Capitais, também conhecido como Mercado de Valores Mobiliários, é um dos segmentos do sistema financeiro responsável por fazer o intermédio entre agentes superávitarios, que detém capital de investimento, e agentes deficitários, que buscam capital para rentabilizá-lo, através da compra e venda valores mobiliários (\textit{i.e.}, ativos financeiros) \cite{mercado_de_capitais}. Consequentemente, gera-se uma maior liquidez destes ativos e também uma melhora no fluxo de capitais entre os agentes econômicos, seja os governos por meio dos bancos centrais, os bancos privados, as instituições financeiras ou até mesmo as pessoas físicas.

\paragraph{} No Brasil, o Mercado de Capitais é regulado e fiscalizado pela Comissão de Valores Mobiliários (CVM), uma autarquia federal vinculada ao Ministério da Fazenda e criada em 1976 através da Lei nº 6.385 \cite{lei_6385}.

\paragraph{} A Bolsa de Valores é uma plataforma onde se negociam os valores mobiliários do Mercado de Capitais, dentre eles ações (\textit{i.e.}, fatias, pedaços) de sociedades anônimas (ou companhias). No Brasil, a única Bolsa de Valores oficial existente é a Brasil, Bolsa, Balcão (B3) \cite{b3}, que administra os sistemas de negociação, compensação, liquidação, depósito e registro para todas as principais classes de ativos.

\paragraph{} O processo de abertura de capital de uma empresa é uma iniciativa que possui vantanges estratégicas como: o aumento da confiança na perspectiva do mercado, seja para o consumidor final ou para parceiros comerciais; a solução de problemas decorrentes de processos sucessórios; e também a captação de capital de investimento, a fim de contribuir para o crescimento ou para a consolidação da companhia \cite{vantagens_sa}. Esse processo acontece através de uma oferta pública inicial (IPO) \cite{oferta_publica}, onde as ações que compõe o capital social \cite{capital_social} de uma companhia são vendidas pela primeira vez ao público geral. Uma vez encerrado o IPO, estas mesmas ações passam para o mercado secundário \cite{mercado_secundario}, onde investidores as negociam entre si. Em retorno ao capital adquirido pela companhia, surgem algumas responsabilidades, dentre elas a publicação de demonstrações financeiras \cite{dem_finan}, auditadas pela própria CVM \cite{audi_dem_finan}.

\paragraph{} Para o acionista de uma sociedade anônima, existem duas formas de se obter lucro: através de proventos (dividendos e juros sobre capital próprio) \cite{proventos}; ou através de operações de compra e de venda de ações, mediante oscilações de seu valor de mercado.



\FloatBarrier
\subsection{Hipótese do Mercado Eficiente}

\paragraph{} A Hipótese do Mercado Eficiente (HME), definida por Eugene Fama \cite{fama1970efficient}, afirma que idealmente o preço de um ativo reflete toda a informação disponível sobre seu valor intrínseco. Em outras palavras, quanto menor o efeito de fatores que contribuam para uma inércia no fluxo de capital de investidores e na transmissão de informações, mais o mercado tende a ser eficiente. São estudados os três níveis de hipóteses:

\begin{itemize}
    \item HME fraca: Os preços atuais refletem todo o histórico de informações disponibilizados publicamente.
    \item HME semi-forte: Engloba a HME fraca, acrescentando-se a existência de uma mudança instantânea que os preços sofrem ao surgirem novas informações.
    \item HME forte: Engloba a HME semi-forte, porém entende-se que a mudança instantânea dos preços acompanha toda e qualquer informação existente sobre o ativo. Assim, absolutamente nenhum investidor conseguiria obter lucro superior à média do mercado, pois não há como acessar nenhuma informação privilegiada, uma vez que ela já estaria refletido no preço corrente do ativo.
\end{itemize}

\paragraph{} O autor menciona que a HME forte não é estritamente válida na realidade, o que é uma afirmação coerente quando se verifica a existência de casos em que o vazamento de informações confidenciais trouxe aos acusados uma lucratividade muito acima da média \cite{insider_trading}.

\paragraph{} A HME fraca foi verificada pela consistência da correlação dos preços dia após dia de determinadas ações, mesmo que esta fosse baixa.

\paragraph{} A hipótese semi-forte também foi sustentada por alguns fatores, dentre eles a verificação de que os futuros pagamentos de proventos das companhias se refletem em média no preços das ações \cite{fama1969adjustment}.

\paragraph{} Em resumo, o estudo da HME traz informações relevantes quanto se avalia a teoria por trás da possibilidade de aplicação de estratégias de \textit{trading} ao mercado financeiro. No entanto, é importante ressaltar que outros autores questionam ao menos parcialmente os estudos realizados por Eugene Fama, seja por resultados inconclusivos ou por anomalias detectadas no comportamento do mercado. Por exemplo, Frank Shostak critica abertamente a premissa de que todos os investidores teriam a mesma expectativa sobre os retornos da empresa \cite{shostak1997defense}. O ganhador do prêmio Nobel em ciências econômicas Paul Samuelson, que afirma que a HME funciona muito melhor para ações individuais do que para o mercado como um todo \cite{jung2005samuelson}. Já o investidor Jack Schwager afirma que a HME está correta pelos motivos errados, pois é muito difícil bater a média do mercado de forma consistente ao mesmo tempo que investidores possuem habilidades diferentes, portanto a informação não é interpretada e aplicada por todos da mesma forma \cite{schwager2012market}.



\FloatBarrier
\subsection{Índice de Bolsa de Valores}
\label{sub:ibov}

Índices de Bolsas de Valores \cite{stock_index} são métricas criadas para avaliar a saúde de um determinado grupo de ações negociadas na bolsa. Cada índice possui uma regra própria de criação que define quais ações são englobadas e com quais pesos, como por exemplo:

\begin{itemize}
    \item S\&P 500 (\textit{Standard and Poor's 500}): Um dos mais conhecidos no mercado global. É a média ponderada pelo capital social das 500 maiores companhias do mercado americano.
    \item DJIA (\textit{Dow Jones Industrial Average}): É a média ponderada pelo preço das ações das 30 maiores \textit{blue-chips}\footnote{Companhias bem conhecidas, bem estabelecidas e com grande capital social.} industriais e financeiras do mercado americano.
    \item Ibovespa (Índice Bovespa): Principal indicador de desempenho do mercado brasileiro. Possui alguns critérios específicos, mas basicamente é composto pelas ações com maior volume de negociação na B3 \cite{ibovespa}.
\end{itemize}

\paragraph{} Índices não são negociáveis pois não passam de métricas de mercado. Para isso existe um tipo de fundo de investimento chamado \textit{Exchange-Traded Fund} (ETF) \cite{etf}, que é especializado em seguir um determinado índice.

\paragraph{} No Brasil, um investidor que deseja que uma parte de seu capital acompanhe um rendimento equivalente ao Ibovespa deverá investir no ETF cujo código de negociação é BOVA11 \cite{etfs_listados}.



\FloatBarrier
\subsection{Mercado Fracionário}

\paragraph{} Ações são negociadas em múltiplos de um lote, que representa uma quantidade mínima de papéis. Nesse contexto, o Mercado Fracionário surge com o objetivo de facilitar negociações de volumes menores que o lote mínimo permitido. No Brasil, o lote é de 100 ações e o Mercado Fracionário permite a compra de no mínimo 1 ação \cite{mercado_fracionario}. No entanto, este mercado possui menor liquidez e maior volatilidade, mas sempre acompanha o preço do ativo negociado no mercado aberto.

\paragraph{} Ações fracionárias podem ser criadas devido: a um desdobramento de ações que não gera resultado par (\textit{e.g.}, 3 para 2); ou a fusões e aquisições de empresas que combinam suas ações em uma razão predeterminada \cite{mercado_fracionario}.

\paragraph{} Grandes investidores e fundos de investimentos não possuem problemas quanto ao capital mínimo necessário para a compra de um lote de ações, visto que negociam em quantidades muito maiores. O problema surge quando um investidor com pouco aporte financeiro deseja entrar no mercado e não consegue encontrar ativos cujo lote mínimo esteja dentro de seu orçamento.



\FloatBarrier
\section{Tipos de Análises}

\paragraph{} Nesta seção, será abordado as duas formas utilizadas para análise do comportamento das companhias ao longo do tempo.



\FloatBarrier
\subsection{Análise Fundamentalista}
\label{subsection:af}

\paragraph{} A Análise Fundamentalista (AF) é um bastante utilizada para identificar tendências de flutuação no preço de ações tendo em vista um horizonte de longo prazo \cite{bulkowski2012fundamental}. Ela se baseia em fatores econômicos relacionados à companhia, como: o quadro de diretores e dirigentes maiores; o fluxo de caixa; a saúde e a situação financeira; o contexto político do país; os concorrentes de mercado; as circunstâncias e os desastres climáticos, naturais ou não, dentre outros fatores.

\paragraph{} Devido à natureza desorganizada e desestruturada do acesso e da leitura dos dados que representam os fatores mencionados, torna-se difícil implementar uma automação eficaz.



\FloatBarrier
\subsection{Análise Técnica}
\label{subsection:at}

\paragraph{} A Análise Técnica (AT) busca identificar tendências de curto prazo na série temporal de preços de ações através da identificação de padrões e da criação de informações derivadas (indicadores técnicos) \cite{murphy1999technical, edwards2018technical}. Segundo a Teoria de Dow, o preço das ações é consequência de todos os acontecimentos relacionados direta ou indiretamente a uma companhia \cite{kirkpatrick2010technical}.

\paragraph{} Diferentemente da AF, a automação desta análise é mais fácil pois os dados normalmente são organizados e estruturados. No entanto, uma das dificuldades desta análise está na separação entre o que é ruído e o que é de fato tendência de mercado.

\paragraph{} Para facilitar a análise, são utilizados indicadores, dentre os quais pode-se citar: o volume financeiro; a identificação de tendências de alta, de baixa e de consolidação de acordo com a Teoria de Dow; as linhas de suporte e de resistência do mercado; as médias móveis; as bandas de Bollinger \cite{bollinger2002bollinger}; e o MACD \cite{appel2007understanding}.


\FloatBarrier
\subsubsection*{Leitura de Gráficos de Candlesticks}

\paragraph{} Gráficos de \textit{Candlesticks}\footnote{Em português: Gráfico de Velas.} são bastante utilizados na AT. A leitura é padronizada de acordo com a Figura \ref{fig:1}. Neste tipo de gráfico as cores importam, pois indicam se o balanço do período foi positivo ou negativo.

\begin{figure}[h]
    \includegraphics[scale=0.50]{candlestick.png}
    \centering
    \caption{Leitura de um gráfico de \textit{candlestick} \cite{candlestick}}
    \label{fig:1}
\end{figure}


\FloatBarrier
\subsubsection*{Teoria de Dow}

\paragraph{} A Teoria de Dow, criada pelo americano Charles Henry Dow em 1884 é considerada a base da AT moderna \cite{kirkpatrick2010technical}. Embora não tivesse sido formalizada explicitamente pelo autor enquanto estava vivo, amigos e profissionais da época tiveram o trabalho de divulgar e fazer alguns ajustes. Baseada na HME, a ideia central por trás da Teoria de Dow é que a lógica econômica deve ser usada para explicar os movimentos do mercado, que em condições ideais segue o padrão de: tendência de alta; topo; tendência de baixa; e fundo, intercalados com períodos de consolidação. A Figura \ref{fig:2} ilustra esse comportamento.

\begin{figure}[h]
    \includegraphics[scale=0.50]{dow_theory.png}
    \centering
    \caption{Comportamento do mercado ideal segundo a Teoria de Dow \cite{kirkpatrick2010technical}}
    \label{fig:2}
\end{figure}



\FloatBarrier
\subsubsection*{Suporte, Resistência e Linhas de Tendência}

\paragraph{} Suporte e Resistência são regiões em um gráfico de \textit{candlestick} onde existe um efeito memória associado a grandes ganhos ou perdas históricas \cite{moraes2007se}. Normalmente estão associadas a eventos econômicos relevantes. A Figura \ref{fig:3} ilustra essas regiões, comumente chamadas de Linhas de Suporte e de Resistência.

\begin{figure}[h]
    \includegraphics[scale=0.50]{suporte_resistencia.png}
    \centering
    \caption{Formação de linhas de Suporte e de Resistência \cite{moraes2007se}}
    \label{fig:3}
\end{figure}

\paragraph{} De maneira semelhante, as Linhas de Tendência oferecem uma inspeção gráfica do quanto o preço de um ativo está crescendo o diminuindo. Portanto, estão necessariamente atreladas a movimentos de tendência de alta ou de tendência de baixa. Em essência, não deixam de ser linhas de Supote e de Resistência. As Figuras \ref{fig:4} e \ref{fig:5} exemplificam esses indicadores.

\begin{figure}[h]
    \includegraphics[scale=0.50]{lta.png}
    \centering
    \caption{Formação de uma Linha de Tendência de Alta \cite{moraes2007se}}
    \label{fig:4}
\end{figure}

\begin{figure}[h]
    \includegraphics[scale=0.50]{ltb.png}
    \centering
    \caption{Formação de uma Linha de Tendência de Baixa \cite{moraes2007se}}
    \label{fig:5}
\end{figure}



\FloatBarrier
\section{Aprendizado de Máquina}

\paragraph{} Aprendizado de Máquina é um campo de estudo dentro de Inteligência Artificial \cite{ibm_ai}. O objetivo é extrair conhecimento a partir de uma conjunto de dados \cite{muller2016introduction}. A terminologia foi criada por um pesquisador da IBM chamado Arthur Samuel em 1959 \cite{ibm_ml} para um estudo de caso do jogo de damas \cite{arthur1959some}.

\paragraph{} Em geral, algoritmos de ML buscam realizar tarefas extremamente complexas computacionalmente sem serem explitamente programadas caso a caso. Alguns exemplos de aplicações que deixam evidente os benefícios deste método são: visão computacional \cite{khan2020machine}, reconhecimento de faces \cite{tripathi2017complex}, recomendação de produtos em plataformas de \textit{e-commerce} \cite{zhou2020product}, identificação de transações financeiras fraudulentas \cite{kumar2019credit} e suporte a diagnósticos médicos \cite{richens2020improving}.

\paragraph{} Dentre as diferentes abordagens de ML que podem ser utilizadas, este trabalho utiliza apenas Aprendizado Supervisionado (AS) \cite{muller2016introduction}.



\FloatBarrier
\subsection{Aprendizado Supervisionado}

\paragraph{} Uma das metodologias mais comuns de ML, seu objetivo é a predição de um resultado a partir de um conjunto de dados de entrada, com a condição de que o modelo tem acesso a vários exemplos de entradas e de saídas de dados para assim obter uma melhor performance \cite{muller2016introduction}.

\paragraph{} O conjunto de dados (\textit{dataset}) com exemplos de entrada e saída utilizado para criação do modelo é chamado de conjunto de dados de treinamento (\textit{training set}). Existe um outro conjunto de dados utilzado para testar a performance do modelo. Este por sua vez é chamado de conjunto de dados de teste (\textit{test set}) e precisa ser diferente dos dados de treinamento para evitar que o efeito memória se sobreponha à qualidade de generalização do modelo, explicado a seguir.

\paragraph{} Todo modelo pode ser avaliado sob o ponto de vista da generalização. Essa característica indica a capacidade de realizar predições acuradas no conjuntos de dados de teste. Um dos métodos de avaliação de performance de generalização é chamado de validação cruzada (\textit{cross-validation} \cite{muller2016introduction}. Neste método, quanto maior a taxa de acerto no conjunto de teste, melhor tende a ser a capacidade de generalização.

\paragraph{} Outras características importantes são conhecidas como \textit{overfitting} e \textit{underfitting}. Quando um modelos está muito complexo a ponto de ser sensível demais aos ruídos do conjunto de treinamento, trazendo dificuldades de generalização, diz-se que ocorreu um \textit{overfitting}. De forma análoga, quando a complexidade do modelo é baixa de forma a não aproveitar devidamente as características importantes do conjunto de treinamento, implicando também em perda de generalização, diz-se que ocorreu um \textit{underfitting}. O objetivo do projetista de um modelo por AS é encontrar um ponto de equilíbrio entre essas características, chamada de ``\textit{sweet spot}'' na Figura \ref{fig:6}, que mostra a relação entre generalização, \textit{overfitting} e \textit{underfitting}.

\begin{figure}[h]
    \includegraphics[scale=0.35]{generalisation.png}
    \centering
    \caption{Relação entre complexidade e acur\'acia de um modelo \cite{muller2016introduction}}
    \label{fig:6}
\end{figure}

\paragraph{} Existem dois tipos principais de problemas de AS, os problemas de Regressão e os problemas de Classificação \cite{muller2016introduction}.



\FloatBarrier
\subsection{Problema de Regressão}

\paragraph{} Este problema envolve a predição de um número contínuo a partir dos dados de entrada \cite{muller2016introduction}. Para exemplificar, pode-se citar a probabilidade de uma pessoa desenvolver uma doença auto-imune a partir de indicadores médicos específicos. Ou também um índice que traz uma espectativa de quantos kilogramas de milho serão colhidos em uma safra a partir de dados geológicos e meteorológicos.



\FloatBarrier
\subsection{Problema de Classificação}
\label{sub:class_prob}

\paragraph{} Os problemas de classificação buscam escolher um rótulo (ou classe) mais provável dentre uma lista de possibilidades finitas e pré-estabelecidas \cite{muller2016introduction}. Como aplicações, pode-se citar: a previsão de escolha eleitoral de pessoas a partir de indicadores socioeconômicos; o diagnóstico de câncer em pacientes a partir de informações médicas; ou mesmo a presença e ausência de animais catalogados em um conjunto de imagens.

\paragraph{} É importante mencionar que problemas de classificação precisam de atenção ao balanceamento das classes (\textit{i.e.} mesma relevância para cada classe durante o treinamento). Em outras palavras, um conjunto de dados não balanceado pode apresentar altos valores de acurácia a partir de um modelo extremamente simples para uma determinada aplicação. Isso acontece porque o modelo aprende que é mais fácil escolher a classe com maior frequencia em seu treinamento do que tentar se aperfeiçoar \cite{provost2000machine}. Para corrigir este efeito, deve-se deixar todas as classes com a mesma relevância durante o treinamento do modelo, o que pode ser feito através dos seguintes métodos:

\begin{itemize}
    \item \textit{Undersampling}: Diminuição de amostras pertencentes à classe mais presente. É aconselhável quando o \textit{dataset} é grande o suficiente para suportar a perda de dados sem perda significativa de generalização. Como vantagem, diminui o tempo de treinamento de um modelo. Ver Figura \ref{fig:7}.
    \item \textit{Oversampling}: Replica ou gera sinteticamente amostras pertencentes à classe menos presente. Como consequência, não há perda de informação potencialmente relevante, porém pode gerar \textit{overfitting}. Pode ser uma boa opção em \textit{datasets} pequenos \cite{weiss2007cost}. Ver Figura \ref{fig:7}.
    \item \textit{Cost Sensitive Learning} (CSL): Ao invés de alterar o tamanho do \textit{dataset}, criam-se pesos diferentes para um erro de classificação durante o treinamento. Portanto, um erro em uma classe menos frequente deve ser mais penalizado do que o contrário. É aconselhável em \textit{datasets} grandes ($>$ 10000) \cite{weiss2007cost}.
\end{itemize}

\begin{figure}[h]
    \includegraphics[scale=0.55]{over_under_sampling.png}
    \centering
    \caption{\textit{Oversampling} e \textit{Undersampling} de classes desbalanceadas \cite{over_under_sampling}}
    \label{fig:7}
\end{figure}



\FloatBarrier
\subsection{Algoritmos de Aprendizado Supervisionado}
\label{sub:alg_apren_sup}

\paragraph{} Esta seção trará uma visão simplificado sobre os algoritmos de AS mais pertinentes ao presente trabalho, em ordem crescente de complexidade. O exemplos citados serão focados em problemas de classificação apenas para entendimento do raciocínio por detrás dos modelos, porém todos possuem variantes para problemas de regressão.


\FloatBarrier
\subsubsection*{\textit{k-Nearest Neighbors}}

\paragraph{} k-NN é talvez o algoritmo mais simples de todos. Consiste na memorização dos dados de treinamento para predizer a classe ou o valor a partir da média dos K registros mais próximos encontrados \cite{muller2016introduction, peterson2009k}. Pode-se citar o uso deste algoritmo por Wolberg e Mangasarian para identificação de malignidade de amostras citológicas de mamas \cite{wolberg1990multisurface}.

\paragraph{} A simplicidade deste algoritmo é uma grande vantagem, mas também é possível citar a facilidade de treinamento e a robustez a dados ruidosos \cite{bhatia2010survey}. Por outro lado, a limitação de memória, a execução demorada devido ao \textit{lazy learning} \cite{aha2013lazy} e a alta sensibilidade a características irrelevantes são seus pontos negativos. A Figura \ref{fig:8} mostra como funciona o critério de seleção da classe de uma amostra de teste a partir dos dados de treinamento e do parâmetro K de vizinhos selecionados.

\begin{figure}[h]
    \includegraphics[scale=0.60]{knn_example.png}
    \centering
    \caption{Funcionamento de um algoritmo k-NN para o problema de classificação. Para K=3 a classe é B e para K=7 a classe é A \cite{knn_classification}.}
    \label{fig:8}
\end{figure}


\FloatBarrier
\subsubsection*{\textit{Decision Tree}}

\paragraph{} Em essência, uma Árvore de Decisão é uma sequência hierárquica de estruturas de decisão \textit{if/else} acerca das características do conjunto de dados. Pode-se mencionar o uso deste algoritmo por Lobato \textit{et al} nos sistemas de energia espanhóis \cite{lobato2006decision}.

\paragraph{} Tecnicamente, é possível construir uma Árvore de Decisão até que todas as suas folhas estejam totalmente puras, ou seja, as sequências de decisão que levam a um resultado só englobam amostras de um tipo de classe. Por outro lado, folhas impuras contém a presença de mais de uma classe, onde se escolhe a classe de maior número de amostras como resultado. O problema é que a presença excessiva de folhas totalmente puras normalmente é acompanhado de um \textit{overfitting} do modelo, portanto precisa ser controlado. Para isso, é possível ajustar alguns parâmetros, como por exemplo: a profundidade, que define a quantidade máxima de camadas que a árvore atingirá qualquer que seja o ramo; o número mínimo de amostras necessário para criação de uma nova ramificação; dentre outros.

\paragraph{} Algumas das vantagens deste modelo estão no fácil entendimento e visualização dos critérios de decisão em árvores pequenas aos olhos do projetista. O tempo de processamento computacional envolvido na criação deste modelo é razoavelmente curto. Não é necessário um pré-processamento dos dados, uma vez que cada característica é processada separadamente. A Figura \ref{fig:9} exemplifica a estrutura por trás de uma Árvode de Decisão.

\begin{figure}[h]
    \includegraphics[scale=0.35]{decision_tree.png}
    \centering
    \caption{Visualização de uma Árvose de Decisão para um \textit{dataset} de câncer de mama \cite{muller2016introduction}.}
    \label{fig:9}
\end{figure}

\paragraph{} Por outro lado, uma desvantagem eminente é a tendência \textit{overfitting} e a baixa capacidade de generalização, que podem ser mitigados através de um algoritmo derivado chamado \textit{Random Forest}.


\FloatBarrier
\subsubsection*{\textit{Random Forest}}

\paragraph{} Um dos modelos mais utilizados atualmente, o algoritmo \textit{Random Forest} é a combinação de diversas Árvores de Decisão ligeiramente diferentes entre si \cite{muller2016introduction}. A ideia é que apesar da tendência de \textit{overfitting} existente, a média dos resultados de cada árvore tende a diminuir esse fator. Além dos parâmetros responsáveis por configurar as árvores individualmente, este modelo também precisa do número de árvores que serão utilizadas.

\paragraph{} Normalmente é preferível utilizar \textit{Random Forests} ao invés de Árvores de Decisão, salvo casos em que o entendimento e a visualização clara do modelo se torna um fator importante. É possível compensar o aumento do tempo de processamente envolvido na criação de uma \textit{Random Forest} com a paralelização dos núcleos de processamento da CPU\footnote{Do inglês: \textit{Central Process Unit}.}.



\FloatBarrier
\section{\textit{Walk-Forward Analysis}}

\paragraph{} \textit{Walk-Forward Analysis} (WFA) \cite{pardo2011evaluation} é um processo de otimização mais voltado para séries temporais no contexto de finanças. Para isso, o \textit{dataset} é dividido em múltiplos segmentos consecutivos, que são iterados progressivamente a fim de se obter os parâmetros ou modelos desejados. A Figura \ref{fig:10} ilustra o processo. Observa-se que são utilizados os termos \textit{in-sample data} (IS) como sinônimo de \textit{training set} e \textit{out-of-sample data} (OOS) como sinônimo de \textit{test set}. A imagem à esquerda representa um processo não-ancorado, onde o início do IS caminha com o decorrer do processo, já a imagem à direita mantém o início fixo por ser um processo ancorado.

\begin{figure}[h]
    \includegraphics[scale=0.45]{wfa.png}
    \centering
    \caption{\textit{Walk-Forward Analysis} Não-Ancorado e Ancorado \cite{wfo}.}
    \label{fig:10}
\end{figure}

\paragraph{} A natureza do WFA ancorado permite uma maior adaptação dos modelos de ML de acordo com as tendências de mercado, que mudam significativamente com o tempo. Durante o treinamento de um modelo, considerar informações temporalmente muito distantes do período de aplicação do mesmo pode comprometer sua acurácia, pois os padrões que guiavam os preços anteriormente não necessariamente são iguais aos padrões atuais.



\FloatBarrier
\section{Considerações para Análise de Resultados}



\FloatBarrier
\subsection{Índice de Sharpe}

\paragraph{} Criado pelo americano William F. Sharpe em 1966 e revisado em 1994, o Índice de Sharpe tem como objetivo medir a performance de um investimento em relação a sua volatilidade, levando também em consideração o rendimento e a volatilidade de um investimento relativamente livre de risco (\textit{e.g.}, título público) \cite{sharpe1998sharpe}. Seja \begin{math}R_a\end{math} o retorno do investimento alvo, \begin{math}R_b\end{math} o retorno do investimento livre de risco e \begin{math}\sigma_a\end{math} seu respectivo desvio padrão, pode-se calcular o Índice de Sharpe através da Equação \ref{eq:1}.

\begin{equation} \label{eq:1}
    S_a = \frac{E[R_a - R_b]}{\sigma_a}
\end{equation}



\FloatBarrier
\subsection{Índice de Sortino}

\paragraph{} O Índice de Sortino, criado pelo americano Frank Sortino \cite{rollinger2013sortino} é uma variante do Índice de Sharpe que considera o desvio padrão apenas dos rendimentos abaixo da média. As Equações \ref{eq:5} e \ref{eq:6} mostram seu cálculo, onde assim como no Índice de Sharpe, \begin{math}R_a\end{math} é o retorno do investimento alvo, \begin{math}R_b\end{math} é o retorno do investimento livre de risco e \begin{math}\sigma_a\end{math} seu respectivo desvio padrão. Considere também \begin{math}X_i\end{math} o i-ésimo retorno e \begin{math}T\end{math} o retorno médio do investimento.

\begin{equation} \label{eq:5}
    S_a = \frac{E[R_a - R_b]}{\sigma_a}
\end{equation}

\begin{equation} \label{eq:6}
    \sigma_a = \sqrt{ \frac{1}{N} \sum_{i=1}^{N} (Min(0, X_i-T))^2}
\end{equation}



\FloatBarrier
\subsection{Correlação de Spearman}

\paragraph{} A Correlação de Postos de Spearman ou simplesmente Correlação de Spearman for criada pelo psicólogo inglês Charles Edward Spearman e revelada em 1904 \cite{spearman1961proof}. Em resumo, a Correlação avalia o grau de proximidade que duas variáveis aleatórias possuem em relação a uma função monotônica. Matematicamente, é o mesmo que a correlação de Pearson aplicada aos postos\footnote{Classificação ordenada das amostras em escala ordinal. Do inglês: \textit{ranks}.} das duas variáveis envolvidas.

\paragraph{} Para duas variáveis aleatórias \begin{math}X_i\end{math} e \begin{math}Y_i\end{math}, são criados os postos \begin{math}rgX_i\end{math} e \begin{math}rgY_i\end{math} para as N amostras presentes. Existem duas formas de se calcular o coeficiente: a primeira, mostrada pela Equação \ref{eq:7}, é para o caso em que há apenas postos inteiros distintos, sem presença de nós (\textit{i.e.}, valores iguais em cada uma das variáveis); já a segunda, ilustrada pela Equação \ref{eq:8}, é para o caso em que há presença de nós.

\begin{equation} \label{eq:7}
    \rho_s = 1 - \frac{6\sum_{i=1}^{N}d_i^2}{N(N^2-1)}
\end{equation}

\begin{equation} \label{eq:8}
    \rho_s = \rho_{rgX, rgY} = \frac{cov(rgX, rgY)}{\sigma_{rgX} \sigma_{rgY}}
\end{equation}

\paragraph{} Na Equação \ref{eq:7}, \begin{math}d_i\end{math} é diferença entre os dois postos de cada variáveis aleatória, mostrado através da Equação \ref{eq:9}.

\begin{equation} \label{eq:9}
    d_i = rgX_i - rgY_i
\end{equation}

\paragraph{} Um exemplo prático do cálculo da correlação pode ser analisado a partir da Tabela \ref{tab:1}, onde são ilustradas amostras para duas variáveis aleatórias \begin{math}X\end{math} e \begin{math}Y\end{math}. A Tabela \ref{tab:2} acrescenta a informação dos postos \begin{math}rgX\end{math} e \begin{math}rgY\end{math}, que ordenam as amostram em ordem decrescente. Observa-se em \begin{math}rgX\end{math} a presença de dois postos com valores de \begin{math}6.5\end{math}, causados pelo nó em \begin{math}X\end{math} de dois valores repetidos (\begin{math}61\end{math}). Neste caso, a regra é a escolha valor médio dos postos que seriam ocupados, no caso 6 e 7.

\begin{table}[h!]
    \begin{center}
        \begin{tabular}{ c|cccccccccc }
            & 1 & 2 & 3 & 4 & 5 & 6 & 7 & 8 & 9 & 10 \\
            \hline
            X & 56 & 75 & 45 & 71 & 61 & 64 & 58 & 80 & 76 & 61 \\
            Y & 66 & 70 & 40 & 60 & 65 & 56 & 59 & 77 & 67 & 63 \\
        \end{tabular}
        \caption{Amostras das variáveis aleatórias X e Y}
        \label{tab:1}
    \end{center}
\end{table}

\begin{table}[h!]
    \begin{center}
        \begin{tabular}{ c|cccccccccc }
            & 1 & 2 & 3 & 4 & 5 & 6 & 7 & 8 & 9 & 10 \\
            \hline
            X & 56 & 75 & 45 & 71 & 61 & 64 & 58 & 80 & 76 & 61 \\
            Y & 66 & 70 & 40 & 60 & 65 & 56 & 59 & 77 & 67 & 63 \\
            \hline
            rgX & 9 & 3 & 10 & 4 & \textbf{6.5} & 5 & 8 & 1 & 2 & \textbf{6.5} \\
            rgY & 4 & 2 & 10 & 7 & 5 & 9 & 8 & 1 & 3 & 6 \\
        \end{tabular}
        \caption{Postos rgX e rgY}
        \label{tab:2}
    \end{center}
\end{table}

\paragraph{} Após o cálculo dos postos, deve-se utilizar a Equação \ref{eq:8}, encontrando-se o valor aproximado de 0.6687.



\FloatBarrier
\section{Trabalhos Relacionados}

\paragraph{} Tendo em vista o conflito de interesses existente por trás de trabalhos de cujo tema está relacionado à previsibilidade do mercado financeiro, pode-se questionar se as estratégias mais promissoras de fato são encontradas em domínio público. Isso ocorre pois a democratização de uma estratégia lucrativa poderia implicar na redução das lucratividades individuais, especialmente se for utilizada em escala.

\paragraph{} Segundo Kendall Kim \cite{kim2010electronic}, somente a partir dos anos 80 que as corretoras começaram a utilizar protocolos de comunicação eletrônica para substituir a corretagem por voz. Essa inovação permitiu o desenvolvimento do \textit{Algorithmic Trading}, que é a automatização da tomada de decisões de estratégias por um computador capaz de enviar ordens de compra e venda diretamente ao mercado.

\paragraph{} A partir do trabalho de Danilo Pereira \cite{pereira2018financial}, pode-se simplificar os modelos de AT aplicados ao mercado financeiro em três metodologias centrais: modelos baseados em indicadores técnicos; modelos baseados em processos estocásticos; e modelos baseados em aprendizado de máquina.



\FloatBarrier
\subsection{Modelos Baseados em Indicadores Técnicos}

\paragraph{} Este tipo de abordagem utiliza informações derivadas da série temporal de preços para criar uma combinação de indicadores que possuam algum poder de previsibilidade de tendência de mercado. Quando comparada aos outros tipos, é a metodologia mais simples e democrática, uma vez que investidores com um conhecimento superficial sobre estatística e inteligência artificial já podem operar em estratégias próprias.

\paragraph{} Diversos \textit{traders}\footnote{Em português: negociantes. Pessoas que compram e vendem bens, moedas ou ações com o objetivo de lucrar, mas não necessariamente com foco em investimento, podendo até assumir um papél especulativo.} e investidores utilizam este tipo de abordagem \cite{ijegwa2014predictive}. Dentre eles podemos citar André Morais \cite{moraes2007se}, de cujas contribuições servirão como base neste trabalho para um aperfeiçoamento via aprendizado de máquina.



\FloatBarrier
\subsection{Modelos Baseados em Processos Estocásticos}

\paragraph{} De acordo com Michael Godfrey \textit{et al.} \cite{godfrey1964random}, a hipótese de que a flutuação de preços no mercado de ações poderia ser explicada por uma \textit{Random Walk}\footnote{Processo aleatório definido pela equação \begin{math}y_t = y_{t-1} + X\end{math}, onde X é uma variável aleatória e \(y\) é a variável resultante.} foi feita por Louis Bachelier \cite{bachelier1900theorie}. A partir da década de 60, muitos trabalhos acadêmicos foram realizados nessa linha na tentativa de entender o comportamento e a previsibilidade do mercado \cite{fama1970efficient, solnik1973note, cooper1982world}, assim como estratégias \cite{malkiel2019random}. Nota-se que até hoje utiliza-se \textit{Random Walks} para testar a hipótese de eficiência de mercados \cite{said2015efficiency}.

\paragraph{} Outra abordagem utilizada são os Modelos Ocultos de Markov (do inglês \textit{Hidden Markov Model}) \cite{rabiner1989tutorial}. Uma Cadeia de Markov é um processo estocástico que modela um sistema por meio de uma sequência finita de estados. A mudança ou a permanência em cada estado é determinada por probabilidades que dependem somente do estado atual. Em uma Cadeia de Markov, pressupõe-se que seus estados sejam observáveis, o que para algumas aplicações, pode não ser verdade. Nesse sentido surge o modelo HMM, que busca aprender sobre um processo não observável (oculto) a partir de um processo observável.

\paragraph{} Em sua pesquisa, Aishwary Jadhav \textit{et al.} \cite{jadhav2021forecasting} utiliza um modelo HMM para previsão do preço de fechamento do dia seguinte para ações FAANG\footnote{Facebook, Amazon, Apple, Netflix, Google.}. A partir da série histórica de preços OHLC\footnote{Open, High, Low, Close. Em português: Abertura, Máximo, Mínimo, Fechamento.}, seu modelo atinge uma eficiência de 97\%-99\%, calculado a partir do erro percentual absoluto médio\footnote{Mean Absolute Percentage Error (MAPE): \begin{math} \frac{1}{N}\sum_{i=1}^{N} \frac{|Predicted(i)-Actual(i)|}{Actual(i)} \end{math}}.

\paragraph{} Uma outra aplicação de modelos HMM é dada por Luca De Angelis \textit{et al.} \cite{de2013dynamic}, que criou uma metodologia a partir de índices da bolsa americana capaz de identificar períodos estáveis e instáveis (\textit{i.e.} crises econômicas), assim como as probabilidades de transição entre um estado e o outro.

\paragraph{} Por fim, pode-ser mencionar o uso de modelos ARCH\footnote{Em português: Heteroscedasticidade Condicional Auto-regressiva.}. A ideia central está na modelagem de uma variância condicional, ou seja, que muda de acordo o instante da série \cite{enders2008applied}. Essa característica se faz muito útil em séries que possuem períodos de alta volatilidade se alternando com períodos de baixa volatilidade. Para um modelo genérico ARCH(q), seja \begin{math}\epsilon_t\end{math} o erro (resíduo) no instante t e \begin{math}\alpha_0\end{math} um ruído branco, pode-se descrever a variância condicional de acordo com a Equação \ref{eq:4}.

\begin{equation} \label{eq:4}
\sigma_t^2 = \alpha_0 + \sum_{i=1}^{q} \alpha_i\epsilon_{t-1}^{2}
\end{equation}

\paragraph{} O modelo ARCH foi proposto por Robert Engle em 1982 para estimar a variância da inflação do Reino Unido \cite{engle1982autoregressive}. A partir daí, várias derivações surgiram, como por exemplo: GARCH \footnote{Generalised ARCH.} por Tim Bollerslev \cite{bollerslev1986generalized} em 1986, EGARCH \footnote{Exponential Generalised ARCH.} por Daniel Nelson \cite{nelson1991conditional} em 1991, NGARCH \footnote{Non-linear Generalised ARCH.} por Matthew Higgins \textit{et al.} \cite{higgins1992class} em 1992, TGARCH \footnote{Threshold Generalised ARCH.} por Roger Rabemananjara \textit{et al.} \cite{rabemananjara1993threshold} em 1993, dentre outros. Alguns dos modelos da família ARCH podem ser encontrados nos trabalhos de Philip Franses \textit{et al.} \cite{franses1996forecasting}, de Juri Marcucci \cite{marcucci2005forecasting} e de Dima Alberg \textit{et al.} \cite{alberg2008estimating}.



\FloatBarrier
\subsection{Modelos Baseados em Aprendizado de Máquina}
\label{sub:ml_models_study}

\paragraph{} Existem registros de estudos sobre inteligência artificial aplicados ao mercado financeiro por volta da década de 70 \cite{felsen1975artificial}, porém ainda em um estágio embrionário devido às dificuldades de processamento computacional e de acesso a dados na época. Por ser uma área de estudo extremamente dependente de ambas as questões, conforme elas foram evoluindo, mais trabalhos puderam ser realizados sobre o tema.

\paragraph{} Isaac Nti \textit{et al.} \cite{nti2020systematic} relata que dos 122 trabalhos mais relevantes publicados entre 2007 e 2018 com o tema de predição do mercado financeiro usando ML, 66\% são baseados em AT, 23\% são baseados em AF e 11\% usam análises mistas. Além disso, os algoritmos mais utilizados são ANN (\textit{Artificial Neural Networks}) e SVM (\textit{Support Vector Machine}).

\paragraph{} De forma semelhante, Dattatray Gandhmal \textit{et al.} \cite{gandhmal2019systematic} verificou que a partir de uma análise detalhada de 50 trabalhos com o tema de predição do mercado financeiro, os algoritmos que mais costumam trazer resultados efetivos são ANN e técnicas baseadas em lógica \textit{Fuzzy}

\paragraph{} É possível encontrar também modelos híbridos, com uma combinação de GARCH com ANN feita por Melike Bildirici \textit{et al.} \cite{bildirici2009improving}.