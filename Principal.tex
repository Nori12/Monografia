\documentclass[a4paper,12pt,oneside,openany]{book}
\input{TesePack}
\usepackage[hidelinks]{hyperref}
\usepackage[bottom]{footmisc}
\usepackage{array}
\usepackage{graphicx}
\usepackage{longtable}
\usepackage{listings}
\usepackage[normalem]{ulem} % Strikethrough
\usepackage{pgfplots} % Plot
\pgfplotsset{holdot/.style={color=blue,fill=white,only marks,mark=*}} % Plot
\usepackage{placeins} % To use \FloatBarrier and keep image inside section
\usepackage[page,toc,titletoc,title]{appendix} % Appendix
% \usepackage{pythonhighlight} % Python highlighting
% \usepackage{minted} % Python highlighting
\usepackage{lscape} % Landscape page for database ERD Image
\usepackage{arydshln} % For horizontal dashed line in tables (\hdashline)
\usepackage{notoccite} % Ensure citation sequence

\usepackage{fifo-stack}
\FSCreate{colors}{black}
\makeatletter
\let\old@color\color
\renewcommand\color[1]{\FSPush{colors}{#1}\old@color{#1}}
\newcommand\colorend{\FSPop{colors}\old@color{\FSTop{colors}}}

\newcommand{\titulo}{TÉCNICAS DE APRENDIZADO DE MÁQUINA APLICADAS A ESTRATÉGIA DE SWING TRADE NO MERCADO FINANCEIRO}

\begin{document}

\frontmatter
\thispagestyle{empty}

\includegraphics[scale=0.7]{Poli.eps}

\begin{center}
\large{\titulo{}}\\
   \vspace{1cm}
\large{Pedro Henrique Barbosa Nori}\\
\end{center}
   \vspace{3cm}
\hspace{7cm}
\hfill \parbox{8.0cm}{Projeto de Graduação apresentado ao Curso de Engenharia Eletrônica e de Computação da Escola Politécnica, Universidade Federal do Rio de Janeiro, como parte dos requisitos necessários à obtenção do título de Engenheiro.\\}
   \vspace{2cm}
\hfill \parbox{8.0cm}{Orientador: Heraldo Luis Silveira de Almeida} \\
   \vspace{2cm}
\begin{center}
Rio de Janeiro

Setembro de 2022
\end{center}


\pagebreak


\begin{center}
\large{\titulo{}}\\
   \vspace{1cm}
\large{Pedro Henrique Barbosa Nori}\\
\end{center}
   \vspace{2cm}
PROJETO DE GRADUAÇÃO SUBMETIDO AO CORPO DOCENTE DO CURSO DE ENGENHARIA ELETRÔNICA E DE COMPUTAÇÃO DA ESCOLA POLITÉCNICA DA UNIVERSIDADE FEDERAL DO RIO DE JANEIRO COMO PARTE DOS REQUISITOS NECESSÁRIOS PARA A OBTENÇÃO DO GRAU DE ENGENHEIRO ELETRÔNICO E DE COMPUTAÇÃO

   \vspace{1cm}
Autor:
      \vspace{0.5cm}
      \begin{flushright}
         \parbox{10cm}{
            \hrulefill

            \vspace{-.375cm}
            \centering{Pedro Henrique Barbosa Nori}

            \vspace{0.1cm}
         }
      \end{flushright}


Orientador:
      \vspace{0.5cm}
      \begin{flushright}
         \parbox{10cm}{
            \hrulefill

            \vspace{-.375cm}
            \centering{Prof. Heraldo Luis Silveira de Almeida, D. Sc.}

            \vspace{0.1cm}
         }
      \end{flushright}

Examinador:
      \vspace{0.5cm}
      \begin{flushright}
         \parbox{10cm}{
            \hrulefill

            \vspace{-.375cm}
            \centering{Prof. Flávio Luis de Mello, D. Sc.}

            \vspace{0.1cm}
         }
      \end{flushright}

Examinador:
      \vspace{0.5cm}
      \begin{flushright}
         \parbox{10cm}{
            \hrulefill

            \vspace{-.375cm}
            \centering{Prof. Natanael Nunes de Moura Junior, D. Sc.}

            \vspace{0.1cm}
         }
      \end{flushright}


      \vfill


\begin{center}
Rio de Janeiro

Setembro de 2022
\end{center}


% Declaracao
\begin{center}
Declara\c{c}\~ao de Autoria e de Direitos
\end{center}

\vspace{0.5cm}

Eu, \emph{Pedro Henrque Barbosa Nori}, CPF \emph{134.129.077-82}, autor da monografia \emph{An\'alise de Estrat\'egia de Swing Trade do Mercado Financeiro e Otimiza\c{c}\~ao Atrav\'es de Aprendizado de M\'aquina}, subscrevo para os devidos fins, as seguintes informa\c\~oes:\\
1. O autor declara que o trabalho apresentado na disciplina de Projeto de Gradua\c{c}\~ao da Escola Polit\'ecnica da UFRJ \'e de sua autoria, sendo original em forma e conte\'udo.\\
2. Excetuam-se do item 1. eventuais transcri\c{c}\~oes de texto, figuras, tabelas, conceitos e id\'eias, que identifiquem claramente a fonte original, explicitando as autoriza\c{c}\~oes obtidas dos respectivos propriet\'arios, quando necess\'arias.\\
3. O autor permite que a UFRJ, por um prazo indeterminado, efetue em qualquer m\'idia de divulga\c{c}\~ao, a publica\c{c}\~ao do trabalho acad\^emico em sua totalidade, ou em parte. Essa autoriza\c{c}\~ao n\~ao envolve \^onus de qualquer natureza \`a UFRJ, ou aos seus representantes.\\
4. O autor pode, excepcionalmente, encaminhar \`a Comiss\~ao de Projeto de Gradua\c{c}\~ao, a n\~ao divulga\c{c}\~ao do material, por um prazo m\'aximo de 01 (um) ano, improrrog\'avel, a contar da data de defesa, desde que o pedido seja justificado, e solicitado antecipadamente, por escrito, \`a Congrega\c{c}\~ao da Escola Polit\'ecnica.\\
5. O autor declara, ainda, ter a capacidade jur\'idica para a pr\'atica do presente ato, assim como ter conhecimento do teor da presente Declara\c{c}\~ao, estando ciente das san\c{c}\~oes e puni\c{c}\~oes legais, no que tange a c\'opia parcial, ou total, de obra intelectual, o que se configura como viola\c{c}\~ao do direito autoral previsto no C\'odigo Penal Brasileiro no art.184 e art.299, bem como na Lei 9.610.\\
6. O autor \'e o \'unico respons\'avel pelo conte\'udo apresentado nos trabalhos acad\^emicos publicados, n\~ao cabendo \`a UFRJ, aos seus representantes, ou ao(s) orientador(es), qualquer responsabiliza\c{c}\~ao/ indeniza\c{c}\~ao nesse sentido.\\
7. Por ser verdade, firmo a presente declara\c{c}\~ao.\\

      \vspace{0.5cm}
      \begin{flushright}
         \parbox{10cm}{
            \hrulefill

            \vspace{-.375cm}
            \centering{Nome do aluno}

            \vspace{0.1cm}
         }
      \end{flushright}

\pagebreak

% Copyright
      \vspace{0.5cm}

UNIVERSIDADE FEDERAL DO RIO DE JANEIRO \\
Escola Polit\'ecnica - Departamento de Eletr\^onica e de Computa\c{c}\~ao \\
Centro de Tecnologia, bloco H, sala H-217, Cidade Universit\'aria \\
Rio de Janeiro - RJ      CEP 21949-900\\
\vspace{0.5cm}
\paragraph{}Este exemplar \'e de propriedade da Universidade Federal do Rio de Janeiro, que poder\'a inclu\'i-lo em base de dados, armazenar em computador, microfilmar ou adotar qualquer forma de arquivamento.
\paragraph{}\'E permitida a men\c{c}\~ao, reprodu\c{c}\~ao parcial ou integral e a transmiss\~ao entre bibliotecas deste trabalho, sem modifica\c{c}\~ao de seu texto, em qualquer meio que esteja ou venha a ser fixado, para pesquisa acad\^emica, coment\'arios e cita\c{c}\~aes, desde que sem finalidade comercial e que seja feita a refer\^encia bibliogr\'afica completa.
\paragraph{}Os conceitos expressos neste trabalho s\~ao de responsabilidade do(s) autor(es).


\pagebreak

% Dedicat�ria
\begin{center}
\textbf{DEDICAT\'ORIA}
\end{center}
      \vspace{0.5cm}

\paragraph{}Opcional.

\pagebreak


% Agradecimento
\begin{center}
\textbf{AGRADECIMENTO}
\end{center}
      \vspace{0.5cm}

\paragraph{}Sempre haver�. Se n�o estiver inspirado, aqui est� uma sugest�o: dedico este trabalho ao povo brasileiro que contribuiu de forma significativa � minha forma��o e estada nesta Universidade. Este projeto � uma pequena forma de retribuir o investimento e confian�a em mim depositados.

\pagebreak


% Resumo
\begin{center}
\textbf{RESUMO}
\end{center}
      \vspace{0.5cm}

\paragraph{}Inserir o resumo do seu trabalho aqui. O objetivo � apresentar ao pretenso leitor do seu Projeto Final uma descri��o gen�rica do seu trabalho. Voc� tamb�m deve tentar despertar no leitor o interesse pelo conte�do deste documento.
\paragraph{}
\noindent Palavras-Chave: trabalho, resumo, interesse, projeto final.

\pagebreak


% Abstract
\begin{center}
\textbf{ABSTRACT}
\end{center}
      \vspace{0.5cm}

\paragraph{}Insert your abstract here. Insert your abstract here. Insert your abstract here. Insert your abstract here. Insert your abstract here.
\paragraph{}
\noindent Key-words: word, word, word.

\pagebreak


% Siglas
\begin{center}
\textbf{SIGLAS}
\end{center}
      \vspace{0.5cm}

\paragraph{}UFRJ - Universidade Federal do Rio de Janeiro
\paragraph{}WYSIWYG - \textit{What you see is what you get}


\pagebreak









% Table of Contents
% ---------------------------------------------------------------
\tableofcontents
% ---------------------------------------------------------------
% Lista de figuras
% ---------------------------------------------------------------
%\cleardoublepage
%\addcontentsline{toc}{chapter}{Lista de Figuras}
\listoffigures
% ---------------------------------------------------------------
% Lista de Tabelas
% ---------------------------------------------------------------
%\cleardoublepage
%\addcontentsline{toc}{chapter}{Lista de Tabelas}
\listoftables

\mainmatter
\cleardoublepage


\chapter{Introdução}
\label{cap1}



\FloatBarrier
\section{Tema}

\paragraph{} O tema deste trabalho se resume na elaboração de uma estratégia de \textit{swing trade} na bolsa de valores brasileira através de métodos de aprendizado de máquina.

\paragraph{} Nesse contexto, o problema a ser abordado é a identificação do momento apropriado para compra de um determinado ativo, como também os preços alvos determinantes para venda, tendo em vista uma variação positiva de seu preço.



\FloatBarrier
\section{Delimitação}

\paragraph{} Este trabalho se limita aos ativos negociados na Bolsa de Valores de São Paulo, a B3, de cujos dados diários de domínio público foram adquiridos através da API \textit{open-source yfinance}, disponível em Python. Não são levadas em consideração informações sobre proventos (dividendos e juros sobre capital próprio) devido à inconsistência dos mesmos na API supracitada, aliada à dificuldade técnica para automatização da busca de tais dados. O Apêndice \ref{ApendiceA} evidencia os problemas encontrados em mais detalhes.

\paragraph{} A duração das operações tem em vista um horizonte mínimo de um dia, sendo portanto operações de \textit{swing trade}. Não são realizas vendas a descoberto\footnote{Venda a descoberto (\textit{short selling}): Venda de ações anterior a compra das mesmas através de um contrato de aluguél. O lucro ocorre na queda do preço de mercado.}, portanto só há lucro em variações positivas de preço dos ativos. Apenas uma operação por ativo pode existir em um determinado instante de tempo para uma estratégia. Em outras palavras, só é possível comprar mais ações de uma empresa após a venda completa das ações da mesma, caso existam.

\paragraph{} A incidência de impostos devidos (\textit{e.g.}, imposto de renda) está fora do escopo. Assim como a utilização de critérios baseados em análise fundamentalista, por causa da dificuldade de obtenção dessas informações de maneira automatizada e estruturada.



\FloatBarrier
\section{Justificativa}

\paragraph{} O crescimento do número de investidores na bolsa de valores brasileira \cite{aumento_investidores} demonstra um maior interesse da população na busca por um complemento da renda familiar ou até na substituição da fonte de renda principal.

\paragraph{} No cenário global, o aumento do uso de robôs de \textit{trading} (ou algoritmos) tem se mostrando expressivo \cite{robos_investidores}, seja por pessoas físicas ou fundos de investimento, de forma total ou parcial em suas estratégias. Por outro lado, tal crescimento não vem sendo igualmente representado no Brasil devido às pecualiaridades do mercado de capitais nacional, como a alta volatilidade e a alta sensibilidade a notícias \cite{robos_e_fundos}.

\paragraph{} Paralelamente, estudos relacionados a aprendizado de máquina vem trazendo resultados práticos no dia-a-dia das pessoas, desde o clássico exemplo de reconhecimento de mensagens de \textit{spam} em um caixa de email à identificação do perfil de consumo de clientes em uma loja. Da mesma forma, instituições financeiras e bancos centrais também estão, com cautela, incorporando aplicações de aprendizado de máquina em tarefas internas \cite{fernandez2019artificial}.

\paragraph{} Apesar das dificuldades inerentes ao cenário atual do mercado de capitais brasileiro, não se pode ignorar o potencial que os algortimos podem trazer. Desta forma, o presente trabalho visa a união de técnicas de aprendizado de máquina a práticas de \textit{trading}, consolidando uma estratégia que sirva de suporte a uma maior variedade de opções de investimento à população brasileira.



\FloatBarrier
\section{Objetivos}

\paragraph{} O objetivo geral deste trabalho é implementar um \textit{software} capaz de simular uma estratégia de \textit{swing trade} utilizando aprendizado de máquina. Especificamente, o software deve: (1) criar um ambiente automatizado que permita buscar, atualizar e armazenar dados diários da bolsa brasileira de forma simples e conforme necessidade do usuário da aplicação; (2) criar a estrutura de uma estratégia por meio de um conjunto de regras e premissas baseadas em práticas de \textit{trading}; (3) gerar os modelos de aprendizado de máquina e acoplá-los à estrutura criada; (4) simular a estratégia obtida; (5) criar um mecanismo de fácil visualização dos resultados das simulações; e (6) analisar os resultados gerados.



\FloatBarrier
\section{Metodologia}

\paragraph{} O trabalho tem início na criação de um ambiente propício à simulação de estratégias, bem como sua configuração e manutenção. Para isso, a fim de: otimizar o tráfego de dados pela internet; minimizar o processamento necessário para a geração de dados derivados (pré-processamento); e armazenar os resultados das estratégias de forma organizada, foi utilizado um banco de dados PostgreSQL. Dentre as atividades realizadas durante o pré-processamento dos dados, anteriores à simulação, é possível citar a identificação de picos na série histórica e os valores de risco mínimo e máximo a serem utilizados nas operações.

\paragraph{} Em seguida, a etapa de simulação começa na leitura de um arquivo JSON contendo todos parâmetros necessários para a execução das estratégias. Nesta etapa, o programa itera dia após dia para cada estratégia configurada verificando os momentos e os valores de compra e de venda para cada ativo que compõe as carteiras. Ao final, registram-se no banco todas as operações executadas, independente da obtenção de lucro, junto com as informações estatísticas necessárias para a avaliação da performance.

\paragraph{} Por fim, com o objetivo de facilitar a análise dos resultados gerados, criou-se um \textit{dashboard} resposável por centralizar todas as informações pertinentes a uma execução de estratégia em uma única página web.

\paragraph{} Observa-se que além do uso de estruturas do banco de dados PostgreSQL, como \textit{triggers} e \textit{functions}, o código foi construído em Python devido à ampla variedade de bibliotecas e ao suporte da comunidade, apesar da desvantagem de desempenho por ser uma linguagem interpretada. O código utiliza as bibliotecas \textit{yfinance}, \textit{pandas}, \textit{numpy}, \textit{scikit-learn}, \textit{multiprocessing}, \textit{matplotlib} e \textit{dash}. Também utilizou-se \textit{containers} Docker \cite{docker} para simplificar a execução.



\FloatBarrier
\section{Descrição}

\paragraph{} No capítulo 2 é desenvolvida a fundamentação teórica acerca de temas relevantes ao entendimento geral de mercado financeiro e de aprendizado de máquina. Também é realizada uma revisão dos trabalhos relacionados ao tema, em outras palavras, a revisão bibliográfica.

\paragraph{} No capítulo 3, toda a metodologia é descrita, o que envolve as etapas de: pré-processamento de dados; simulações e otimizações de parâmetros; e a criação dos modelos de ML.

\paragraph{} No capítulo 4, é apresentado o resultado da simulação a partir dos modelos criados e dos parâmetros otimizados no capítulo 3.

\paragraph{} Por fim, o capítulo 5 encerra com a conclusão e as recomendações de trabalhos futuros.
\chapter{Fundamentação Teórica}
\label{cap2}

\paragraph{} Neste capítulo, são introduzidos alguns conceitos chave para o entendimento do projeto. Nas próximas seções, são feitas contextualizações sobre o Mercado de Capitais, Bolsa de Valores, Ações e Aprendizado de Máquina.


\section{Mercado de Capitais, Bolsa de Valores e Ações}

\paragraph{} O Mercado de Capitais, também conhecido como Mercado de Valores Mobiliários, é um dos segmentos do sistema financeiro responsável por fazer o intermédio entre agentes superávitarios, que tem capital de investimento, e agentes deficitários, que buscam capital para rentabilizá-lo, através da compra e venda valores mobiliários (i.e., ativos financeiros) \cite{mercado_de_capitais}. Consequentemente, gera-se uma maior liquidez destes ativos e também uma melhora no fluxo de capitais entre os agentes econômicos, sejam eles os governos por meio dos bancos centrais, os bancos privados, as insituições financeiras ou até mesmo as pessoas físicas.

\paragraph{} No Brasil, o Mercado de Capitais é regulado e fiscalizado pela CVM (Comissão de Valores Mobiliários), uma autarquia federal vinculada ao Ministério da Fazenda e criada em 1976 através da Lei nº 6.385 \cite{lei_6385}.

\paragraph{} A Bolsa de Valores é uma plataforma onde se negociam os valores mobiliários do Mercado de Capitais, dentre eles ações (i.e., fatias, pedaços) de sociedades anônimas (ou companhias). No Brasil, a única Bolsa de Valores oficial existente é a B3 (Brasil, Bolsa, Balcão) \cite{b3}, que administra os sistemas de negociação, compensação, liquidação, depósito e registro para todas as principais classes de ativos.

\paragraph{} O processo de abertura de capital de uma empresa é uma iniciativa que possui vantanges estratégicas \cite{vantagens_sa} como: o aumento da confiança na perspectiva do mercado, seja para o consumidor final ou para parceiros comerciais; a solução de problemas decorrentes de processos sucessórios; e também a captação de capital de investimento, a fim de contribuir para o crescimento ou para a consolidação da companhia. Esse processo acontece através de uma oferta pública \cite{oferta_publica}, ou IPO (Initial Public Offering), onde as ações que compõe o capital social \cite{capital_social} de uma companhia são vendidas pela primeira vez ao público geral. Uma vez encerrado o IPO, estas mesmas ações passam para o mercado secundário \cite{mercado_secundario}, onde investidores as negociam entre si. Em retorno ao capital adquirido pela companhia, surgem algumas responsabilidades, dentre elas a publicação de demonstrações financeiras \cite{dem_finan}, auditadas pela própria CVM \cite{audi_dem_finan}.

\paragraph{} Para o acionista de uma sociedade anônima, existem duas formas de se obter lucro: através de proventos (dividendos e juros sobre capital próprio) \cite{proventos}; ou através de operações de compra e de venda de ações, mediante oscilações de seu valor de mercado. Conforme a expectavida corretamente induz, o lucro é comumente aferido durante a venda de um determinado papél (i.e., ação) posteriormente à sua aquisição a um preço de compra inferior. No entando, também é possível trabalhar com posições vendidas (short selling) \cite{short_selling}, onde um investidor aluga ações de outro investidor por meio de um contrato. Em seguida as vende para posteriormente recomprá-las a um preço inferior, devolvendo-as assim ao respectivo dono. Neste caso, o lucro é obtido quando expectativa de queda de um ativo se mostra verdadeira.

\subsection{Hipótese do Mercado Eficiente}

\paragraph{} A Hipótese do Mercado Eficiente, definida por FAMA \cite{fama1970efficient}, afirma que idealmente o preço de um ativo reflete toda a informação disponível sobre seu valor intrínseco. Em outras palavras, quanto menor o efeito de fatores que contribuam para uma inércia no fluxo de capital de investidores e na transmissão de informações, mais o mercado tende a ser eficiente. São estudados os três níveis de hipóteses:

\begin{itemize}
    \item HME fraca: Os preços atuais refletem o todo o histórico de informações disponibilizados publicamente.
    \item HME semi-forte: Engloba a HME fraca, acrescentando a existência de uma mudança instantânea que os preços sofrem ao surgirem novas informações.
    \item HME forte: Engloba a HME semi-forte, porém entende que a mudança instantânea dos preços acompanha toda e qualquer informação existente sobre o ativo. Assim, absolutamente nenhum investidor conseguiria obter lucro superior à média do mercado, pois não há como acessar nenhuma informação privilegiada, uma vez que ela já estaria refletido no preço corrente do ativo.
\end{itemize}

\paragraph{} O autor menciona que o HME forte não é estritamente válida na realidade, o que é uma afirmação coerente quando se verifica a existência de casos em que o vazamento de informações confidenciais trouxe aos acusados uma lucratividade significativa \cite{insider_trading}.

\paragraph{} A HME fraca foi verificada devido à consistência da correlação dos preços dia após dia de determinadas ações, mesmo que esta fosse baixa.

\paragraph{} A hipótese semi-forte também foi sustentada por alguns fatores, dentre eles a verificação de que os futuros pagamentos de dividendos das companhias se refletem, em média, no preços das ações \cite{fama1969adjustment}.

\paragraph{} Em resumo, o estudo das Hipóteses de Mercado Eficiente traz informações relevantes quanto se avalia a teoria por trás da possibilidade de aplicação de estratégias de trading no mercado financeiro. No entanto, é importante ressaltar que outros autores questionam ao menos parcialmente os estudos realizados por FAMA, sejam por resultados inconclusivos ou por anomalias detectadas no comportamento do mercado. Por exemplo, SHOSTAK \cite{shostak1997defense} critica abertamente a premissa de que todos os investidores teriam a mesma expectativa sobre os retornos da empresa. O ganhador do prêmio Nobel em ciências econômicas Paul Samuelson, que afirma que o a HME funciona muito melhor para ações individuais do que para o mercado como um todo \cite{jung2005samuelson}. Já o investidor Jack Schwager afirma que a HME está correta pelos motivos errados \cite{schwager2012market}, pois é muito difícil bater a média do mercado de forma consistente ao mesmo tempo que investidores possuem habilidades diferentes, portanto a informação não é interpretada e aplicada por todos da mesma forma.


\subsection{Índice de Bolsa de Valores}

Índices de Bolsas de Valores \cite{stock_index} são métricas criadas para avaliar a saúde de um determinado grupo de ações negociadas na bolsa. Cada índice possui uma regra própria de criação que define quais ações são englobadas e com quais pesos, como por exemplo:

\begin{itemize}
    \item S\&P 500: Um dos mais conhecidos no mercado. É a média ponderada pelo capital social das 500 maiores companhias do mercado americano.
    \item Dow Jones Industrial Average: É a média ponderada pelo preço da ação das 30 maiores blue-chips industriais e financeiras do mercado americano (i.e., companhias bem conhecidas, bem estabelecidas e com grande capital social).
    \item Ibovespa: Principal indicador de desempenho do mercado brasileiro. Possui alguns critérios específicos, mas basicamente é composto pelas ações com maior volume de negociação na B3 \cite{ibovespa}.
\end{itemize}

\paragraph{} Índices não são negociáveis pois não passam de métricas de mercado. Para isso existem fundos de investimentos chamados ETFs (Exchange-Traded Funds) \cite{etf}, especializados em seguir um determinado índice.

\paragraph{} No Brasil, um investidor que deseja que uma parte de seu capital acompanhe um rendimento equivalente ao iBovespa deverá investir no ETF, cujo código de negociação é BOVA11.


\subsection{Mercado Fracionário}

\paragraph{} Ações são negociadas em múltiplos de um lote, que representa uma quantidade mínima de papéis a transacionar. Nesse contexto, o Mercado Fracionário \cite{mercado_fracionario} surge com o objetivo de facilitar negiociações de volumes menores que o lote mínimo permitido. Na prática, ações fracionárias são agrupadas até formarem um lote para então serem negociadas. Normalmente o Mercado Fracionário possui menor liquidez e maior volatilidade, mas sempre acompanha o preço do ativo negociado no mercado aberto.

\paragraph{} Ações fracionárias podem ser criadas devido: a um desdobramento de ações que não gera resultado par (e.g., 3 para 2); ou a fusões e aquisições de empresas que combinam suas ações em uma razão predeterminada.

\paragraph{} Grandes investidores e fundos de investimentos não possuem problemas quanto ao capital mínimo necessário para a compra de um lote de ações, visto que negociam em quantidades muito maiores. O problema surge quando um investidor com pouco aporte financeiro deseja entrar no mercado e não consegue encontrar ativos cujo lote mínimo esteja dentro de seu orçamento.

\paragraph{} No Brasil, o lote mínimo é de 100 ações e o Mercado Fracionário permite a compra de no mínimo 1 ação.

\subsection{Índice de Sharpe}

\paragraph{} Criado pelo americano William F. Sharpe em 1966 e revisado em 1994, o Índice de Sharpe\footnote{Também conhecido como \textit{Sharpe Index}, \textit{Sharpe Ratio} ou até \textit{Sharpe Measure}.} tem como objetivo medir a performance de um investimento em relação a sua volatilidade, levando também em consideração o rendimento e a volatilidade de um investimento relativamente livre de risco (e.g. título público) \cite{sharpe1998sharpe}. Seja \begin{math}R_a\end{math} o retorno do investimento alvo, \begin{math}R_b\end{math} o retorno do investimento livre de risco, \begin{math}\sigma_a\end{math} e \begin{math}\sigma_b\end{math} seus respectivos desvios padrão, pode-se calcular o Índice de Sharpe através da Equação \ref{eq:1}.

\begin{equation} \label{eq:1}
    S_a = \frac{E[R_a - R_b]}{\sigma_a - \sigma_b}
\end{equation}


\section{Tipos de Análises}

\subsection{Análise Fundamentalista}
\label{subsection:af}

\paragraph{} A Análise Fundamentalista (AF) é um muito utilizada para identificar tendências de flutuação no preço de ações tendo em vista um horizonte de longo prazo \cite{bulkowski2012fundamental}. Ela se baseia em fatores econômicos relacionados à companhia, como: o quadro de diretores e dirigentes maiores; o fluxo de caixa; a saúde e a situação financeira; o contexto político do país; os concorrentes de mercado; as circunstâncias climáticas; os desastres climáticos, naturais ou não, dentre outros fatores.

\paragraph{} Devido à natureza desorganizada e desestruturada dos dados que representam os fatores mencionados, torna-se muito difícil implementar uma automação.

\subsection{Análise Técnica}
\label{subsection:at}

\paragraph{} A Análise Técnica (AT) busca identificar tendências de curto prazo na série temporal de preços de ações através da identificação de padrões e da criação de informações derivadas (indicadores técnicos) \cite{murphy1999technical, edwards2018technical}. Segundo a Teoria de Dow, o preço das ações é consequência de todos os acontecimentos relacionados direta ou indiretamente a uma companhia \cite{kirkpatrick2010technical}.

\paragraph{} Diferentemente da AF, a automação desta análise é muito mais fácil pois os dados normalmente são organizados e estruturados. No entanto, como são obtidos a posteriori, a dificuldade desta análise se dá na separação entre o que é ruído e o que é de fato tendência de mercado, além da criação de informações derivadas que se mostram relativamente úteis.

\paragraph{} Dentre os indicadores mais famosos e portanto utilizados, podemos citar: o volume financeiro; a identificação de tendências de alta, de baixa e de consolidação de acordo com a Teoria de Dow; as linhas de suporte e de resistência do mercado; as médias móveis; as bandas de Bollinger \cite{bollinger2002bollinger}; e o MACD (Moving Average Convergence-Divergence) \cite{appel2007understanding}.

\subsubsection*{Leitura de Gráficos de Candlesticks}

\paragraph{} Gráficos de Candlesticks\footnote{Em português: Gráfico de Velas.} são bastante utilizados na AT. A leitura é padronizada de acordo com a Figura \ref{fig:1}. Neste tipo de gráfico as cores importam, pois indicam se o balanço do período foi positivo ou negativo.

\begin{figure}[h]
    \includegraphics[scale=0.50]{candlestick.png}
    \centering
    \caption{Leitura de um gráfico de \textit{candlestick} \cite{candlestick}}
    \label{fig:1}
\end{figure}

\subsubsection*{Teoria de Dow}

\paragraph{} A Teoria de Dow, criada pelo americano Charles Henry Dow em 1884 é considerada a base da AT moderna \cite{kirkpatrick2010technical}. Embora não tivesse sido formalizada explicitamente pelo autor enquanto estava vivo, amigos e profissionais da época tiveram o trabalho de divulgar e fazer alguns ajustes. Baseada na HME, a ideia central por trás da Teoria de Dow é que a lógica econômica deve ser usada para explicar os movimentos do mercado, que em condições ideais segue o padrão de: tendência de alta\footnote{Topos e fundos ascendentes.}; topo; tendência de baixa\footnote{Topos e fundos descendentes.}; e fundo, intercalados com períodos de consolidação\footnote{Topos e fundos lateralizados.}. A Figura \ref{fig:2} ilustra esse comportamento.

\begin{figure}[h]
    \includegraphics[scale=0.50]{dow_theory.png}
    \centering
    \caption{Comportamento do mercado ideal segundo a Teoria de Dow \cite{kirkpatrick2010technical}}
    \label{fig:2}
\end{figure}

\subsubsection*{Média Móvel Exponencial}

\paragraph{} A Média Móvel Exponencial (MME) possui uma característica que a torna relevante para estratégias de AT. Ela dá um maior peso relativo às amostras mais recentes dentro de uma série temporal. As Equações \ref{eq:2} e \ref{eq:3} mostram o seu cálculo, onde $P_t$ representa o preço atual, $\mathit{MME_{t-1}}$ é a média acumulada até o instante anterior e $K$ é uma constante definida pela quantidade de amostras desejadas $n>0$.

\begin{equation} \label{eq:2}
    \mathit{MME_t} = (P_t - \mathit{MME_{t-1}}) * K + \mathit{MME_{t-1}}
\end{equation}
\begin{equation} \label{eq:3}
    K = \frac{2}{n+1}
\end{equation}

\subsubsection*{Suporte, Resistência e Linhas de Tendência}

\paragraph{} Suporte e Resistência são regiões em um gráfico de \textit{candlestick} onde existe um grande efeito memória associado a grandes ganhos ou perdas históricas \cite{moraes2007se}. Normalmente estão associadas a eventos econômicos relevantes. A Figura \ref{fig:3} ilustra essas regiões, comumente chamadas de Linhas de Suporte e de Resistência.

\begin{figure}[h]
    \includegraphics[scale=0.50]{suporte_resistencia.png}
    \centering
    \caption{Formação de linhas de Suporte e de Resistência \cite{moraes2007se}}
    \label{fig:3}
\end{figure}

\paragraph{} De maneira semelhante, as Linhas de Tendência oferecem uma inspeção gráfica do quanto o preço de um ativo está crescendo o diminuindo. Portanto, estão necessariamente atreladas a movimentos de tendência de alta ou de tendência de baixa. Em essência, não deixam de ser linhas se Supote e de Resistência. As Figuras \ref{fig:4} e \ref{fig:5} exemplificam esses indicadores.

\begin{figure}[h]
    \includegraphics[scale=0.50]{lta.png}
    \centering
    \caption{Formação de uma Linha de Tendência de Alta \cite{moraes2007se}}
    \label{fig:4}
\end{figure}

\begin{figure}[h]
    \includegraphics[scale=0.50]{ltb.png}
    \centering
    \caption{Formação de uma Linha de Tendência de Baixa \cite{moraes2007se}}
    \label{fig:5}
\end{figure}


\section{Aprendizado de Máquina}

\paragraph{} Aprendizado de Máquina (Machine Learning) é um campo de estudo dentro de Ingeligência Artificial \cite{ibm_ai} que engloba estatística e ciência da computação. O objetivo é extrair conhecimento a partir de uma conjunto de dados \cite{muller2016introduction}. A terminologia foi criada por um pesquisador da IBM chamado SAMUEL em 1959 \cite{ibm_ml} para um estudo de caso do jogo de damas \cite{arthur1959some}.

\paragraph{} Em geral, algoritmos de ML buscam realizar tarefas extremamente complexas computacionalmente sem serem explitamente programadas caso a caso. Alguns exemplos de aplicações que deixam evidente os benefícios deste método são: visão computacional, identificação de rosto, recomendação de produtos em plataformas de \textit{e-commerce}, identificação de transações financeiras fraudulentas, suporte a diagnósticos médicos, dentre diversos outros.

\paragraph{} Algoritmos de ML podem ser baseados em Aprendizado Supervisionado, Aprendizado Não Supervisionado ou até mesmo um modelo híbrido. Este trabalho utiliza apenas AS para a criação de modelos.


\subsection{Aprendizado Supervisionado}

\paragraph{} Uma das metodologias mais comuns de ML, seu objetivo é a predição de um resultado a partir de um conjunto de dados de entrada, com a condição de que o modelo tem acesso a vários exemplos de entrada e saída de dados para uma melhor performance \cite{muller2016introduction}.

\paragraph{} O conjunto de dados (\textit{dataset}) com exemplos de entrada e saída utilizado para criação do modelo é chamado de dados de treinamento (\textit{training set}). Existe um outro conjunto de dados utilzado para testar a performance do modelo. Este segundo conjunto, chamado de dados de teste (\textit{test set}), precisa ser necessariamente diferente dos dados de treinamento para evitar que o efeito memória se sobreponho à qualidade de generalização do modelo (explicado a seguir). Como regra geral de uso, é aconselhável separar 75\% dos dados para os dados de treinamento e 25\% para os dados de teste, ou algo próximo desta proporção \cite{muller2016introduction}.

\paragraph{} Todo modelo pode ser avaliado sob o ponto de vista da generalização. Essa característica indica a capacidade de realizar predições acuradas em conjuntos de dados semelhantes ao de treinamento, porém jamais vistos (dados de teste). Quanto maior a taxa de acerto nos dados de teste, melhor tende a ser a capacidade de generalização.

\paragraph{} Outras características importantes são conhecidas como \textit{overfitting} e \textit{underfitting}. Quando um modelos está muito complexo a ponto de ser sensível demais aos ruídos dos dados de treinamento, trazendo dificuldades de generalização, diz-se que ocorreu um \textit{overfitting}. De forma análoga, quando a complexidade do modelo é baixa de forma a não aproveitar devidamente as característica importantes dos dados de treinamento, implicado também em perda de generalização, diz-se que ocorreu um \textit{underfitting}. O objetivo do projetista de um modelo por AS é encontrar um ponto de equilíbrio entre essas características, chamada de ``\textit{Sweet spot}'' na Figura \ref{fig:6}, que mostra a relação entre generalização, \textit{overfitting} e \textit{underfitting}.

\begin{figure}[h]
    \includegraphics[scale=0.35]{generalisation.png}
    \centering
    \caption{Relação entre complexidade e acur\'acia de um modelo \cite{muller2016introduction}}
    \label{fig:6}
\end{figure}

\paragraph{} Existem dois tipos de problemas associados ao AS, os problemas de Regressão e os problemas de Classificação.


\subsection{Problema de Regressão}

\paragraph{} Este problema envolve a predição de um número contínuo a partir dos dados de entrada \cite{muller2016introduction}. Para exemplificar, pode-se citar a probabilidade de uma pessoa desenvolver uma doença auto-imune a partir de indicadores médicos específicos. Ou também um índice que traz uma espectativa de quantos kilogramas de milho serão colhidos em uma safra a partir de dados geológicos e meteorológicos.

\subsection{Problema de Classificação}

\paragraph{} Os problemas de classificação buscam escolher um rótulo (ou classe) mais provável dentre uma lista de possibilidades finitas e pré-estabelecidas \cite{muller2016introduction}. Como aplicações, pode-se citar: a previsão de escolha eleitoral de pessoas a partir de indicadores socioeconômicos; o diagnóstico de câncer em pacientes a partir de informações médicas; ou mesmo a presença e ausência de animais catalogados em um conjunto de imagens.

\paragraph{} É importante mencionar que problemas de classificação precisam de atenção ao balanceamento das classes (i.e. mesma relevância para cada classe durante o treinamento). Em outras palavras, um conjunto de dados não balanceado pode gerar um modelo pouco complexo para uma aplicação não trivial, o que implica em um ilusório indíce de acurácia nos dados de teste. Isso acontece porque o modelo tende a quase sempre escolher a classe com maior frequencia em seu treinamento, independentemente da composição dos dados. Para corrigir este efeito, deve-se deixar todas as classes com a mesma relevância durante o treinamento do modelo, o que pode ser feito através dos seguintes métodos:

\begin{itemize}
    \item \textit{Undersampling}: Diminuição de amostras pertencentes à classe mais presente. É aconselhável quando o \textit{dataset} é grande o suficiente para suportar a perda de dados sem perda significativa de generalização. Como vantagem, diminui o tempo de treinamento de um modelo. Ver Figura \ref{fig:7}.
    \item \textit{Oversampling}: Replica ou gera sinteticamente amostras pertencentes à classe menos presente. Como consequência, não há perda de informação potencialmente relevante, porém pode gerar \textit{overfitting}. Pode ser uma boa opção em \textit{datasets} pequenos \cite{weiss2007cost}. Ver Figura \ref{fig:7}.
    \item \textit{Cost Sensitive Learning} (CSL): Ao invés de alterar o tamanho do \textit{dataset}, criam-se pesos diferentes para um erro de classificação durante o treinamento. Portanto um erro numa classe menos frequente deve ser mais penalizado do que o contrário. É aconselhável em \textit{datasets} grandes ($>$ 10000) \cite{weiss2007cost}.
\end{itemize}

\begin{figure}[h]
    \includegraphics[scale=0.55]{over_under_sampling.png}
    \centering
    \caption{\textit{Oversampling} e \textit{Undersampling} de classes desbalanceadas \cite{over_under_sampling}}
    \label{fig:7}
\end{figure}


\subsection{Algoritmos de Aprendizado Supervisionado}

\paragraph{} Esta seção trará uma visão simplificado sobre os algoritmos de AS mais pertinentes ao presente trabalho, em ordem crescente de complexidade. O exemplos citados serão focados em problemas de classificação apenas para entendimento do raciocínio por detrás dos modelos, porém todos possuem variantes para problemas de regressão.

\subsubsection*{\textit{k-Nearest Neighbors}}

\paragraph{} k-NN é talvez o algoritmo mais simples de todos. Consiste na memorização dos dados de treinamento para predizer a classe ou o valor a partir da média dos K registros mais próximos encontrados. A Figura \ref{fig:8} mostra como funciona o critério de seleção da classe de uma amostra de teste a partir dos dados de treinamento e do parâmetro K de vizinhos selecionados.

\begin{figure}[h]
    \includegraphics[scale=0.60]{knn_example.png}
    \centering
    \caption{Funcionamento de um algoritmo k-NN para o problema de classificação \cite{knn_classification}. Para K=3 a classe é B e para K=7 a classe é A.}
    \label{fig:8}
\end{figure}

\subsubsection*{\textit{Decision Tree}}

\paragraph{} Em essência, uma Árvore de Decisão\footnote{Em inglês: \textit{Decision Tree}.} é uma sequência hierárquica de estruturas de decisão \textit{if/else}\footnote{Em português: se/senão.} acerca das características do conjunto de dados. Tecnicamente, pode-se construir uma Árvore de Decisão até que todas as suas folhas\footnote{Em inglês: leafs.} estejam totalmente puras, ou seja, as sequências de decisão que levam a um resultado só englobam amostras de um tipo de classe. Ao contrário de folhas impuras, que contém a presença de mais de uma classe, onde normalmente se escolhe a de maior número de amostras como resultado. O problema é que a presença excessiva de folhas totalmente puras é acompanhado de um \textit{overfitting} do modelo, portanto precisa ser controlado. Para isso, é possível ajustar alguns parâmetros, como por exemplo: a profundidade, que define a quantidade máxima de camadas que a árvore atingirá qualquer que seja o ramo; o número mínimo de amostras para se criar uma nova ramificação; dentre outros.

\paragraph{} Algumas vantagens deste modelo estão no relativamente fácil entendimento e visualização dos critérios de decisão para o projetista em árvores pequenas. O tempo de processamento computacional envolvido na criação deste modelo é razoavelmente curto. Não é necessário um pré-processamento dos dados, uma vez que cada característica é processada separadamente. A Figura \ref{fig:9} mostra a estrutura por trás de uma Árvode de Decisão.

\begin{figure}[h]
    \includegraphics[scale=0.35]{decision_tree.png}
    \centering
    \caption{Visualização de uma Árvose de Decisão para um \textit{dataset} de câncer de mama \cite{muller2016introduction}.}
    \label{fig:9}
\end{figure}

\paragraph{} Por outro lado, uma desvantagem eminente é a tendência \textit{overfitting} e a baixa capacidade de generalização, que podem ser mitigados através de um algoritmo derivado chamado \textit{Random Forest}.


\subsubsection*{\textit{Random Forest}}

\paragraph{} Um dos modelos mais utilizado atualmente, o algoritmo \textit{Random Forest}\footnote{Em português: Floresta Aleatória.} é a combinação de diversas Árvores de Decisão ligeiramente diferentes entre si \cite{muller2016introduction}. A ideia é que apesar da tendência de \textit{overfitting} existente, a média dos resultados de cada árvore tende a diminuir esse fator. Além dos parâmetros responsáveis por configurar as árvores individualmente, este modelo também precisa no número de árvores que serão utilizadas.

\paragraph{} Normalmente é preferível utilizar \textit{Random Forests} ao invés de Árvores de Decisão, salvo casos em que o entendimento e a visualização clara do modelo se torna um fator importante, o que difícil de ser analisado quando existem muitras árvores. É possível compensar o aumento do tempo de processamente envolvido na criação deste modelo com paralelização em núcleos de processamento da CPU\footnote{Do inglês: \textit{Central Process Unit}.}.

\section{Trabalhos Relacionados}

\paragraph{} Tendo em vista o conflito de interesses existente por trás de trabalhos de cujo tema está relacionado à previsibilidade do mercado financeiro, pode-se questionar se as estratégias mais promissoras de fato são encontradas em domínio público. Isso ocorre pois a democratização de uma estratégia lucrativa poderia implicar na redução das lucratividades individuais, especialmente se for utilizada em escala.

\paragraph{} Segundo KIM \cite{kim2010electronic}, somente a partir dos anos 80 que as corretoras começaram a utilizar protocolos de comunicação eletrônica para substituir a corretagem por voz. Essa inovação permitiu o desenvolvimento do Algorithmic Trading, que é a automatização da tomada de decisões de estratégias por um computador capaz de enviar ordens de compra e venda diretamente ao mercado.

\paragraph{} Para efeito de simplificação, os modelos de AT aplicados ao mercado financeiro serão agrupados em três metodologias centrais: modelos baseados em indicadores técnicos; modelos baseados em processos estocásticos; e modelos baseados em aprendizado de máquina.


\subsection{Modelos Baseados em Indicadores Técnicos}

\paragraph{} Este tipo de abordagem utiliza informações derivadas da série temporal de preços para criar uma combinação de indicadores que possuam algum poder de previsibilidade da tendência de mercado. Quando comparada aos outros tipos, é a metodologia mais simples e democrática, uma vez que pessoas com pouco ou nenhum conhecimento sobre estatística e inteligência artificial podem operar em estratégias próprias.

\paragraph{} Diversos \textit{traders}\footnote{Em português: negociantes. Pessoas que compram e vendem bens, moedas ou ações com o objetivo de lucrar, mas não necessariamente com foco em investimento, podendo até assumir um viés especulativo.} e investidores utilizam este tipo de abordagem. Dentre eles podemos citar MORAES \cite{moraes2007se}, de cujas contribuições servirão como base neste trabalho para um aperfeiçoamento via aprendizado de máquina.


\subsection{Modelos Baseados em Processos Estocásticos}

\paragraph{} De acordo com GODFREY \cite{godfrey1964random}, a hipótese de que a flutuação de preços no mercado de ações poderia ser explicada por uma Random Walk\footnote{Processo aleatório definido pela equação \begin{math}y_t = y_{t-1} + X\end{math}, onde X é uma variável aleatória e \(y\) é a variável resultante.} foi feita por BACHELIER \cite{bachelier1900theorie}. A partir da década de 60, muitos trabalhos acadêmicos foram realizados nessa linha na tentativa de entender o comportamento e a previsibilidade do mercado \cite{fama1970efficient, solnik1973note, cooper1982world}, assim como estratégias \cite{malkiel2019random}. Nota-se que até hoje utiliza-se Random Walks para testar a hipótese de eficiência de mercados \cite{said2015efficiency}.

\paragraph{} Outra abordagem utilizada são os Modelos Ocultos de Markov (do inglês Hidden Markov Model, ou HMM) \cite{rabiner1989tutorial}. Uma Cadeia de Markov é um processo estocástico que modela um sistema por meio de uma sequência finita de estados. A mudança ou a permanência em cada estado é determinada por probabilidades que dependem somente do estado atual. Em uma Cadeia de Markov, pressupõe-se que seus estados sejam observáveis, o que para algumas aplicações, pode não ser verdade. Nesse sentido surge o modelo HMM, que busca aprender sobre um processo não observável (oculto) a partir de um processo observável.

\paragraph{} Em sua pesquisa, JADHAV et al \cite{jadhav2021forecasting} utiliza um modelo HMM para previsão do preço de fechamento do dia seguinte para ações FAANG\footnote{Facebook, Amazon, Apple, Netflix, Google}. A partir da série histórica de preços OHLC\footnote{Open, High, Low, Close. Em português: Abertura, Máximo, Mínimo, Fechamento}, seu modelo atinge uma eficiência de 97\%-99\%, calculado a partir do erro percentual absoluto médio\footnote{Mean Absolute Percentage Error (MAPE): \begin{math} \frac{1}{N}\sum_{i=1}^{N} \frac{|Predicted(i)-Actual(i)|}{Actual(i)} \end{math}}.

\paragraph{} Uma outra aplicação de modelos HMM é dada por DE ANGELIS et al \cite{de2013dynamic}, que criou uma metodologia a partir de índices da bolsa americana capaz de identificar períodos estáveis e instáveis (i.e. crises econômicas), assim como as probabilidades de transição entre um estado e o outro.

\paragraph{} Por fim, pode-ser mencionar o uso de modelos ARCH\footnote{Em português: Heteroscedasticidade Condicional Auto-regressiva.} (Autoregressive Conditional Heteroskedasticity). A ideia central está na modelagem de uma variância (volatilidade) condicional, ou seja, que muda de acordo o instante da série \cite{enders2008applied}. Essa característica se faz muito útil em séries que possuem períodos de alta volatilidade se alternando com períodos de baixa volatilidade. Para um modelo genérico ARCH(q), seja \begin{math}\epsilon_t\end{math} o erro (resíduo) no instante t e \begin{math}\alpha_0\end{math} um ruído branco, pode-se descrever a variância condicional de acordo com a Equação \ref{eq:4}.

\begin{equation} \label{eq:4}
\sigma_t^2 = \alpha_0 + \sum_{i=1}^{q} \alpha_i\epsilon_{t-1}^{2}
\end{equation}

\paragraph{} O modelo ARCH foi proposto por ENGLE em 1982 para estimar a variância da inflação do Reino Unido \cite{engle1982autoregressive}. A partir daí, várias derivações surgiram, como por exemplo: GARCH \footnote{Generalised ARCH.} por BOLLERSLEV \cite{bollerslev1986generalized} em 1986, EGARCH \footnote{Exponential Generalised ARCH.} por NELSON em 1991, NGARCH \footnote{Non-linear Generalised ARCH.} por HIGGINS e BERA\cite{higgins1992class} em 1992, TGARCH \footnote{Threshold Generalised ARCH.} por ZAKOIAN e RABEMANANJARA \cite{rabemananjara1993threshold} em 1993, dentre outros. Alguns dos modelos da família ARCH podem ser encontrados nos trabalhos de FRANSES e DIJK \cite{franses1996forecasting}, de MARCUCCI \cite{marcucci2005forecasting} e de ALBERG et al \cite{alberg2008estimating}.


\subsection{Modelos Baseados em Aprendizado de Máquina}

\paragraph{} Existem registros de estudos sobre inteligência artificial aplicados ao mercado financeiro por volta da década de 70 \cite{felsen1975artificial}, porém ainda em um estágio embrionário devido às dificuldades de processamento computacional e de acesso a dados na época. Por ser uma área de estudo extremamente dependente de ambas as questões, conforme elas foram evoluindo, mais trabalhos puderam ser realizados sobre o tema.

\paragraph{} NTI et al \cite{nti2020systematic} relata que dos 122 trabalhos mais relevantes publicados entre 2007 e 2018 com o tema de predição do mercado financeiro usando ML, 66\% são baseados em AT, 23\% são baseados em AF e 11\% usam análises mistas. Além disso, os algoritmos mais utilizados são ANN\footnote{Em português: Redes Neurais Artificiais.} (\textit{Artificial Neural Networks}) e SVM\footnote{Em português: Máquina de Vetor de Suporte.} (\textit{Support Vector Machine}).

\paragraph{} De forma semelhante, GANDHMAL e KUMAR \cite{gandhmal2019systematic} verificaram que a partir de uma análise detalhada de 50 trabalhos com o tema de predição do mercado financeiro, os algoritmos que mais costumam trazer resultados efetivos são ANN e técnicas baseadas em lógica \textit{Fuzzy}\footnote{Em português: Difuso.}

\paragraph{} É possível encontrar também modelos híbridos, com uma combinação de GARCH com ANN feita por BILDIRCI e ERSIN \cite{bildirici2009improving}.
\chapter{Metodologia}
\label{cap3}

\section{Resumo}

\paragraph{} As seções a seguir trazem detalhes quanto a estrutura técnica do projeto. Portanto, a Figura \ref{fig:100} mostra uma noção geral de como as estrutras se conectam.

\begin{figure}[h]
    \includegraphics[scale=0.90]{no_image.jpeg}
    \centering
    \caption{Estrutura do técnica do projeto (Imagem em construção)}
    \label{fig:100}
\end{figure}

\paragraph{} Primeiro, tem-se início a etapa de pré-processamento de dados, onde ocorre a leitura e interpretação do arquivo de configuração para se saber quantas estratégias executar, quais os ativos envolvidos e seus recpectivos intervalos de tempo. Uma vez verificado no banco de dados os dados já existentes, faz-se um \textit{download} apenas dos dados necessários. Se houver alguma atualização de dados, as \textit{features} de uso geral são calculadas e armazenadas no banco a fim de servir de insumo para as estratégias que estarão por vir.

\paragraph{} Completada a etapa de pré-processamento, inicia-se a simulação das estratégias. O arquivo de configuração foi projetado para ser capaz de designar diversas estratégias de parâmetros distintos a uma mesma ordem de execução de programa. Dessa forma, faz-se uso da biblioteca \textit{multiprocessing} para paralelizar as simulações, cujos resultados e estatísticas pertinentes são salvas no banco para posterior análise.

\paragraph{} É possível visualizar os resultados de forma clara através de uma aplicação secundária responsável por criar um \textit{dashboard} interativo.

\paragraph{} A aplicação foi desenvolvida em \textit{Python} com o apoio das bibliotecas \textit{yfinance}, \textit{pandas}, \textit{dash} e \textit{multiprocessing}. Foi estruturado um banco de dados \textit{PostgreSQL} para armazenamento dos \textit{candlesticks} baixados, das \textit{features} geradas e arquivamento das estratégias executadas. Também foi incorporado o uso de \textit{Docker} especificamente para a execução de estratégias sem necessidade de configuração de ambiente.

\section{Pré-Processamento}

\subsection{Arquivo de Configuração}
\paragraph{} EXCLUIR - Falar motivação do arquivo; Print de exemplo; Apontar para subsection de lista de parâmetros; Possibilidade de rodar várias simulações.

\paragraph{} O Arquivo de Configuração é um arquivo no formato JSON responsável por configurar detalhadamente cada parâmetro da sequência de estratégias que se deseja executar. Uma ordem de execução do programa pode conter diversas simulações de estratégias, que são configuradas neste Arquivo. A Figura \ref{fig:101} mostra sua estrutura.

\begin{figure}[h]
    \includegraphics[scale=0.90]{no_image.jpeg}
    \centering
    \caption{Arquivo de Configuração para execução singular (Imagem em construção)}
    \label{fig:101}
\end{figure}

\paragraph{} Nota-se que no topo são listados os parâmetros de uso geral, cujos valores precedem quaisquer outros listados na estratégias subsequentes, caso possam ser reescritos. Em seguida abre-se o vetor de tipos de estratégias, onde o campo \textit{name} representa o nome da classe criada, sendo este o elemento que conecta o usuário ao tipo de estratégia desejada. Após a seleção do nome, são configurados os parâmetros internos da estratégia, que podem ter natureza obrigatória ou opcional.

\subsubsection*{Lista de Parâmetros}
















\subsection{Coleta de Dados}
\paragraph{} yfinance; Problema com proventos; Normalização de proventos do preço das ações.

\subsection{Armazenamento de Dados}
\paragraph{} Banco de dados; Criação de Candles semanais.

\subsection{Geração de Features de Uso Geral}
\paragraph{} Quais features; Cuidados com não-causalidade.


\section{Simulação de Estratégia}

\subsection{Estrutura}
\paragraph{} Carteira com N ativos de datas distintas; Regra de 3 para 1 entre stop e alvo; 1 operação por ativo.

\subsection{Premissas}
\paragraph{} Compra na abertura do merdado; Sem venda no dia da compra; Prioridades durante venda (stop primeiro).

\subsection{Período Máximo de Dias por Operação}
\paragraph{} Motivação da escolha dos 45 dias; Gráfico entre ABEV e MGLU.

\subsection{Gerenciamento de Risco}
\paragraph{} Coeficiente de Risco-Capital

\subsection{Risco de Entrada por Operação}
\paragraph{} Cálculo do risco mínimo; Cálculo do risco Máximo.

\subsection{Descanso por Tendência de Baixa}
\paragraph{}

\subsection{Descanso por Identificação de Crises}
\paragraph{}

\subsection{Lista de Parâmetros de Configuração}
\paragraph{} Lista todos e explicar o que fazem.

\subsection{Ensaios Paralelos}
\paragraph{} Parâmetros que estão implementados e não trouxeram resultados expressivos.



\section{Otimizações de Gerenciamento de Carteira}

\subsection{Normalização por Frequência de Operações}
\paragraph{}

\subsection{Controle Proporcional para Uso de Capital}
\paragraph{}

\subsection{Compensação por Lucratividade}
\paragraph{}



\section{Criação de Modelos}

\subsection{Resumo}
\paragraph{}

\subsection{\textit{Feature Selection}}
\paragraph{}

\subsection{Geração de \textit{Datasets}}
\paragraph{}

\subsection{\textit{Walk Forward Optimization}}
\paragraph{}

\subsection{Critérios de Escolha}
\paragraph{}


\section{Análise de Resultados}
\paragraph{} Dashboard; Baseline
\chapter{Resultados}
\label{cap4}



\paragraph{} A partir dos parâmetros encontrados nas Seções anteriores, utilizou-se a seguinte configuração para encontrar a simulação com os melhores índices de performance:

\begin{itemize}
    \item 71 \textit{tickers} (Tabela \ref{tab:5})
    \item Período de simulação: 01/01/2019 a 31/12/2021
    \item Capital: R\$ 100000,00
    \item Período máximo de dias por operação: 45
    \item Risco de Entrada por Operação: 0,29
    % \item Descanso por Identificação de Crises: Sim
    % \item Descanso por Tendência de Baixa: Sim
    \item RCC: 0,000025
    \item Controle Proporcional para Uso de Capital (RCC Dinâmico): Sim
    \item Valor de Referência para Uso de Capital: 100\%
    \item Constante K de Ganho Proporcional: 30000
\end{itemize}

\paragraph{} Os resultados encontrados podem ser verificados pela Tabela \ref{tab:13} e pela Figura \ref{fig:250}.

\begin{table}[h!] %ID 485
    \begin{center}
        \begin{tabular}{ l|c|c }
            Parâmetro & Estratégia & \textit{Baseline} \\
            \hline
            Rendimento Final & 72,79\% & 73,94\% \\
            Volatilidade & 63,20\% & 55,48\% \\
            Índice de Sharpe & 0,64 & 0,68 \\
            Índice de Sortino & 0,74 & 0,74 \\
            Correlação de Spearman (c/ \textit{Baseline}) & 0,99 & - \\
            Correlação de Spearman (c /Ibovespa) & 0,92 & - \\
            Uso Máximo de Capital & 100\% & 100\% \\
            Uso Médio de Capital & 97,22\% & 100\% \\
            Máximo de Operações Ativas & 71 & - \\
            Média de Operações Ativas & 65,97 & - \\
            Desvio Padrão de Operações Ativas & 6,13 & -\\
            Operações Totais & 3145 & 71 \\
            Operações de Sucesso & 891 (28,3\%) & - \\
            Operações de Falha & 2048 (65,1\%) & - \\
            Operações de \textit{Timeout} & 147 (4,7\%) & - \\
            Operações Incompletas & 59 (1,9\%) & - \\
        \end{tabular}
        \caption{Resultado final}
        \label{tab:13}
    \end{center}
\end{table}

\begin{figure}[!htb]
    \includegraphics[scale=0.50]{performance_final.png}
    \centering
    \caption{Performance final}
    \label{fig:250}
\end{figure}

\chapter{Considerações Finais}
\label{cap5}



\FloatBarrier
\section{Conclusão}



\paragraph{} Neste trabalho, foi proposta a criação de uma estratégia de \textit{swing trade} com a utilização de técnicas de aprendizado de máquina. A partir dos resultados apresentados no Capítulo \ref{cap4}, foi possível observar que a estratégia proposta possui indicadores de performance muito próximos da estratégia \textit{baseline}, que representa a média de rendimento do mercado.

\paragraph{} Como não há consideração de proventos nos dados utilizados, a tendência é que ambas as estratégias apresentem uma performance real melhor do que a simulada. No entanto, a estratégia \textit{baseline} deve revelar um aumento de performance relativamente maior, pois os papéis adquiridos ficam sempre em posse durante todo o período de simulação, não deixando intervalos de tempo descobertos. Também se deve ressaltar que o presente trabalho considerou premissas pessimistas durante as simulações, o que traz perspectivas ligeiramente mais promissoras.

\paragraph{} Por fim, os resultados encontrados mostram que o uso de modelos de aprendizado de máquina pode auxiliar investidores no mercado de ações, no entanto para superar a média do mercado são necessários estudos mais aprofundados antes de uma implementação real.



\FloatBarrier
\section{Trabalhos Futuros}



\paragraph{} Este trabalho se baseou em premissas e regras que moldaram uma estrutura na qual os modelos pudessem operar. No entanto, alguns valores escolhidos podem ser revisitados a fim de serem aprimorados. Dentre eles, podemos citar a escolha do valor de 45 dias para o período máximo de dias por operação. A criação dos modelos de ML necessita de um valor prévio deste parâmetro. Assim, para se otimizar o valor do período, é necessário criar vários conjuntos de modelos, um para cada dia de período máximo que se deseja considerar, seguido pela execução da simulação. O problema desta abordagem é o tempo total de criação de apenas um conjunto de modelos, pois devido ao WFA, um conjunto requer 12 modelos para cada um dos 71 \textit{tickers} da listagem geral (Figura \ref{fig:580} e Tabela \ref{tab:5}), o que demora mais de um dia de execução em um computador de uso doméstico nos padrões atuais.

\paragraph{} Deve-se mencionar também o parâmetro de razão entre ganho e perda, já que foi utilizado o valor constante de 3 a partir da recomendação do André Moraes. É possível abrir uma frente de trabalho visando encontrar um valor mais otimizado, o que necessitaria, assim como o caso anterior do período máximo de dias por operação, a criação de conjuntos de modelos com valores diferentes, o que também não é trivial.

\paragraph{} A consideração dos proventos durante a simulação das estratégias


% ---------------------------------------------------------------
% Bibliografia
% ---------------------------------------------------------------
\normalsize
\cleardoublepage
\addcontentsline{toc}{chapter}{Bibliografia}
\bibliographystyle{coppe}
\bibliography{biblio}


% Apendices

\appendix

% Apendice A


\chapter{Inconsistência de Proventos na Biblioteca \textit{yfinance}}
\label{ApendiceA}

\paragraph{} Apesar da praticidade de obtenção dos \textit{candlesticks} diários que a biblioteca \textit{yfinance} (Python) traz, seus valores de proventos (dividendos e juros sobre capital próprio) não são totalmente confiáveis. O estudo em questão mostra inconsistências tanto por duplicação quanto por inserção incorreta de proventos. Para isso, uma análise de caso foi realizada para a companhia Magazine Luiza (\textit{ticker} MGLU3), onde foram comparados os dados obtidos do \textit{yfinance} via \textit{script} com o site de relações com investidores da mesma \cite{mglu_ri}. Além disso, utilizou-se a plataforma \textit{TradingView} \cite{tradingview} para confirmação dos valores de preço de fechamento.

% \paragraph{} Referencia \ref{codeA1}

\paragraph{} A Tabela \ref{tab:ap1} mostra o histórico dos preços de fechamento para alguns dias específicos e datas importantes, como distribuição de proventos e desdobramentos \footnote{Em inglês: \textit{split}}, além de outros períodos. A Tabela está ordenada do \textit{candle} mais recente para o mais antigo e as marcações: em vermelho indicam valores incorretos; em azul indicam valores corretos; e em laranja indicam valores que prograparam erros a partir dos valores incorretos. A data de execução do \textit{script} é de 19/09/2020, o que é relevante, uma vez que a plataforma sempre retorna os preços dos \textit{candles} já normalizado por todos os desdobramentos acumulados.


\begin{table}[h!]
    \begin{center}
        \resizebox{\textwidth}{!}{
        \begin{tabular}{ c|ccc|cc|cc }
            Data & Preço Fch            & Preço Fch & Preço Fch             & Provento/Ação     & Provento/Ação & \textit{Split}    & \textit{Split} \\
                 & \textit{yfinance}    & Site RI   & \textit{TradingView}  & \textit{yfinance} & Site RI       & \textit{yfinance} & Site RI \\
                 & (R\$)                & (R\$)     & (R\$)                 & (R\$)             & (R\$/ação)    &                   & \\
            \hline
            16/09/2022 & 4,46 & 4,46 & 4,46 & - & - & - & - \\
            01/07/2022 & 2,20 & 2,20 & 2,20 & - & - & - & - \\
            03/01/2022 & 6,72 & 6,72 & 6,72 & - & - & - & - \\

            07/07/2021 & 22,01 & 22,01 & 22,01 & - & - & - & - \\
            06/07/2021 & 21,07 & 21,07 & 21,07 & \color{blue} 0,015494 \color{black} & 0,0154942583 & - & - \\
            05/07/2021 & 21,3645 & 21,37 & 21,36 & - & - & - & - \\

            04/01/2021 & 25,1817 & 25,18 & 25,18 & - & - & - & - \\
            30/12/2020 & 24,9319 & 24,93 & 24,93 & \color{blue} 0,026301 \color{black} & 0,0263019985 & - & - \\
            29/12/2020 & 25,2354 & 25,24 & 25,24 & - & - & - & - \\

            15/10/2020 & 25,4650 & 25,47 & 25,46 & - & - & - & - \\
            14/10/2020 & 25,6347 & 25,64 & \color{red} 25,54 \color{black} & - & - & \color{blue} 1:4 \color{black} & 1:4 \\
            13/10/2020 & 25,9541 & 25,96 & 25,95 & - & - & - & - \\

            03/08/2020 & 20,6061 & 20,61 & 20,61 & \color{red} 0,094176 \color{black} & - & - & - \\
            31/07/2020 & \color{orange} 20,0479 \color{black} & 20,15 & 20,14 & \color{blue} 0,023541 \color{black} & 0,094165968 & - & - \\
            30/07/2020 & \color{orange} 20,6654 \color{black} & 20,77 & 20,76 & - & - & - & - \\
            29/07/2020 & \color{orange} 19,9012 \color{black} & 20,00 & 19,99 & - & - & - & - \\

            15/04/2020 & \color{orange} 10,8798 \color{black} & 10,93 & 10,93 & - & - & - & - \\
            14/04/2020 & \color{orange} 10,6441 \color{black} & 10,70 & 10,69 & \color{red} 0,179508 \color{black} & - & - & - \\
            13/04/2020 & \color{orange} 10,2203 \color{black} & 10,45 & 10,45 & - & - & - & - \\
            09/04/2020 & \color{orange} 10,1569 \color{black} & 10,39 & 10,38 & - & - & - & - \\

            03/01/2020 & \color{orange} 11,9224 \color{black} & 12,19 & 12,19 & - & - & - & - \\
            02/01/2020 & \color{orange} 12,0297 \color{black} & 12,30 & 12,30 & \color{blue} 0,008947 \color{black} & 0,0357891574 & - & - \\
            30/12/2019 & \color{orange} 11,6235 \color{black} & 11,89 & 11,88 & - & - & - & - \\

            09/10/2019 & \color{orange} 9,6619 \color{black} & 9,88 & 9,88 & - & - & - & - \\
            08/10/2019 & \color{orange} 9,2598 \color{black} & 9,47 & 9,47 & \color{blue} 0,018402 \color{black} & 0,0736066061 & - & - \\
            07/10/2019 & \color{orange} 9,3394 \color{black} & 9,55 & 9,55 & - & - & - & - \\

            06/08/2019 & \color{orange} 8,9016 \color{black} & 9,11 & 9,10 & - & - & \color{blue} 1:8 \color{black} & 1:8 \\

            17/04/2019 & \color{orange} 4,8588 \color{black} & 4,97 & 4,97 & - & - & - & - \\
            16/04/2019 & \color{orange} 4,9463 \color{black} & 5,06 & 5,06 & \color{blue} 0,011571 \color{black} & 0,370259884 & - & - \\
            15/04/2019 & \color{orange} 4,9594 \color{black} & 5,07 & 5,07 & - & - & - & - \\

            03/01/2019 & \color{orange} 5,5812 \color{black} & 5,71 & 5,71 & - & - & - & - \\
            02/01/2019 & \color{orange} 5,6416 \color{black} & 5,77 & 5,77 & \color{blue} 0,018522 \color{black} & 0,59270489 & - & - \\
            28/12/2018 & \color{orange} 5,4744 \color{black} & 5,60 & 5,60 & - & - & - & - \\

        \end{tabular}}
        \caption{Análise de Consistência de Proventos: MGLU3}
        \label{tab:ap1}
    \end{center}
\end{table}

\paragraph{} Analisando a Tabela \ref{tab:ap1} de cima para baixo, nota-se que a primeira irregularidade notável ocorre no dia 14/10/2020, onde a plataforma \textit{TradingView} apresenta um preço de fechamento discrepante em relação ao site de RI da própria companhia e do \textit{yfinance}. Como se trata de um evento singular e não é o foco deste estudo, ele foi desconsiderado.

\paragraph{} Em seguida, nos dias 03/08/2020 e 31/07/2020, o \textit{yfinance} registrou a presença dos proventos de R\$0,094176/ação e R\$0,023541/ação, respectivamente. O problema aqui é que além de ser muito improvável que qualquer empresa na bolsa brasileira distribuia proventos duas vezes em dois dias úteis seguidos, pode-se notar que o valor de R\$0,023541/ação equivalete ao de R\$0,094176/ação quando multiplicado por 4. Em outras palavras, normalizando pelo desdobramento de 1:4 ocorrido em 14/10/2020, conclui-se que um dos proventos é duplicado. Como confirmação, o site de RI da Magazine Luiza dispões de um comunidado sobre a distribuição de proventos de R\$0,094165968/ação em 31/07/2020.

\paragraph{} Continuando a análise, é possível verificar que a partir da duplicata encontrada, os preços de fechamento do \textit{yfinance} vão acumulando o erro. Em 14/04/2020, o \textit{yfinance} contabilizou proventos de R\$0,179508/ação, no entanto, nada foi encontrado no site de RI, o que evidencia um lançamento incorreto. Nota-se também que o valor é relativamente alto quando comparado aos outros proventos de outras datas.

\paragraph{} Os restantes do valores de proventos do \textit{yfinance} em azul equivalem aos comunicados pelo site de RI da companhia, porém deve-se levar em consideração os desdobramentos acumulados.

\paragraph{} Por fim, pode-se concluir que o uso da plataforma \textit{yfinance} no que diz respeito à disponibilização de proventos no contexto deste projeto não pode ser deferida, uma vez que a presença e a magnitude dos valores incorretos não é desprezível.


% \begin{minted}{python}

% import yfinance as yf
% from pathlib import Path

% ticker = 'MGLU3.SA'
% msft = yf.Ticker(ticker)
% hist = msft.history(start='2015-01-01', end='2022-09-19',
%     interval='1d', prepost=False, back_adjust=True, rounding=True)
% destination = Path(__file__).parent / (ticker.upper()+'.csv')
% hist.to_csv(destination)

% \end{minted}


% % Apendice B

% \chapter{Encadernação do Projeto de Graduação}
% \label{ApendiceB}
% \include{ApendiceB}

% % Apendice C

% \chapter{O que é um anexo}
% \label{ApendiceC}
% \include{ApendiceC}

\backmatter

\end{document}