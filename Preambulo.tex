% Declaracao
\begin{center}
Declaração de Autoria e de Direitos
\end{center}

\vspace{0.5cm}

Eu, \emph{Pedro Henrque Barbosa Nori} CPF \emph{134.129.077-82}, autor da monografia \emph{\titulo{}}, subscrevo para os devidos fins, as seguintes informações:\\
1. O autor declara que o trabalho apresentado na disciplina de Projeto de Graduação da Escola Politécnica da UFRJ é de sua autoria, sendo original em forma e conteúdo.\\
2. Excetuam-se do item 1. eventuais transcrições de texto, figuras, tabelas, conceitos e idéias, que identifiquem claramente a fonte original, explicitando as autorizaçõees obtidas dos respectivos proprietários, quando necessárias.\\
3. O autor permite que a UFRJ, por um prazo indeterminado, efetue em qualquer mídia de divulgação, a publicação do trabalho acadêmico em sua totalidade, ou em parte. Essa autorização não envolve ônus de qualquer natureza à UFRJ, ou aos seus representantes.\\
4. O autor pode, excepcionalmente, encaminhar à Comissão de Projeto de Graduação, a não divulgação do material, por um prazo máximo de 01 (um) ano, improrrogável, a contar da data de defesa, desde que o pedido seja justificado, e solicitado antecipadamente, por escrito, à Congregação da Escola Politécnica.\\
5. O autor declara, ainda, ter a capacidade jurídica para a prática do presente ato, assim como ter conhecimento do teor da presente Declaração, estando ciente das sanções e punições legais, no que tange a cópia parcial, ou total, de obra intelectual, o que se configura como violaçõo do direito autoral previsto no Código Penal Brasileiro no art.184 e art.299, bem como na Lei 9.610.\\
6. O autor é o único responsável pelo conteúdo apresentado nos trabalhos acadêmicos publicados, não cabendo à UFRJ, aos seus representantes,  ou ao(s) orientador(es), qualquer responsabilização/ indenização nesse sentido.\\
7. Por ser verdade, firmo a presente declaração.\\

      \vspace{0.5cm}
      \begin{flushright}
         \parbox{10cm}{
            \hrulefill

            \vspace{-.375cm}
            \centering{Pedro Henrique Barbosa Nori}

            \vspace{0.1cm}
         }
      \end{flushright}

\pagebreak

% Copyright
      \vspace{0.5cm}

UNIVERSIDADE FEDERAL DO RIO DE JANEIRO \\
Escola Politécnica - Departamento de Eletrônica e de Computação \\
Centro de Tecnologia, bloco H, sala H-217, Cidade Universitária \\
Rio de Janeiro - RJ      CEP 21949-900\\
\vspace{0.5cm}
\paragraph{}Este exemplar é de propriedade da Universidade Federal do Rio de Janeiro, que poderá incluí-lo em base de dados, armazenar em computador, microfilmar ou adotar qualquer forma de arquivamento.
\paragraph{}É permitida a menção, reprodução parcial ou integral e a transmissão entre bibliotecas deste trabalho, sem modificação de seu texto, em qualquer meio que esteja ou venha a ser fixado, para pesquisa acadêmica, comentários e citações, desde que sem finalidade comercial e que seja feita a referência bibliográfica completa.
\paragraph{}Os conceitos expressos neste trabalho são de responsabilidade do(s) autor(es).


\pagebreak

% Dedicat�ria
\begin{center}
\textbf{DEDICATÓRIA}
\end{center}
      \vspace{0.5cm}

\paragraph{} \`A minha m\~ae engenheira mec\^anica que tanto amo no meu cora\c{c}\~ao.

\pagebreak


% Agradecimento
\begin{center}
\textbf{AGRADECIMENTO}
\end{center}
      \vspace{0.5cm}


\paragraph{} Agrade\c{c}o ao meu pa\'is, que esconde um povo t\~ao sofrido e ao mesmo tempo t\~ao amoroso. Este trabalho é apenas um pedacinho do pagamento da minha d\'ivida.

\paragraph{} INCLUIR - Pais, MinervaBots, PH, Professores, Heraldo

\pagebreak


% Resumo
\begin{center}
\textbf{RESUMO}
\end{center}
      \vspace{0.5cm}

\paragraph{} Todos os dias, diversas negociações são realizadas nas bolsas de valores no mundo inteiro. Com os mais diversos objetivos, investidores buscam um aumento crescente de patrimônio de forma consistente. Paralelamente, inteligências artificias vem substituindo cada vez mais atividades antes desempenhadas pelo homem.

\paragraph{} Nesse sentido, este trabalho visa a aplicação de técnicas de aprendizado de máquina para aumento de performance de uma estratégia de swing trade no mercado de ações brasileiro (B3). Para isso, é realizada a reprodução aproximada da estratégia, seguida pela substituição dos critérios de decisão de entrada nas operações e preços alvos de venda por um modelo de aprendizado de máquina.

\paragraph{}
\noindent Palavras-Chave: Machine Learning, Análise Técnica, Swing Trade, Mercado Financeiro.

\pagebreak


% Abstract
\begin{center}
\textbf{ABSTRACT}
\end{center}
      \vspace{0.5cm}

\paragraph{}Insert your abstract here. Insert your abstract here. Insert your abstract here. Insert your abstract here. Insert your abstract here.
\paragraph{}
\noindent Key-words: word, word, word.

\pagebreak


% Siglas
\begin{center}
\textbf{SIGLAS}
\end{center}
      \vspace{0.5cm}

\paragraph{}AF - Análise Fundamentalista
\paragraph{}API - \textit{Application Programming Interface}
\paragraph{}ANN - \textit{Artificial Neural Networks}
% \paragraph{}ANS - Aprendizado Não Supervisionado
\paragraph{}ARCH - \textit{Autoregressive Conditional Heteroskedasticity}
\paragraph{}AS - Aprendizado Supervisionado
\paragraph{}AT - Análise Técnica
\paragraph{}B3 - Bolsa, Brasil, Balção
\paragraph{}CPU - \textit{Central Process Unit}
\paragraph{}CSL - \textit{Cost Sensitive Learning}
\paragraph{}CSV - \textit{Comma-separated values}
\paragraph{}DT - \textit{Decision Tree}
\paragraph{}EGARCH - \textit{Exponential Generalised ARCH}
\paragraph{}EMA - \textit{Exponential Moving Average}
\paragraph{}GARCH - \textit{Generalised ARCH}
\paragraph{}HME - Hipótese do Mercado Eficiente
\paragraph{}HMM - \textit{Hidden Markov Model}
\paragraph{}iBovespa - Índice Bovespa
\paragraph{}JSON - \textit{JavaScript Object Notation}
\paragraph{}k-NN - \textit{K Nearest Neighbors}
\paragraph{}MACD - \textit{Moving Average Convergence/Divergence}
\paragraph{}ML - \textit{Machine Learning}
\paragraph{}MME - Média Móvel Exponencial
\paragraph{}NGARCH - \textit{Non-linear Generalised ARCH}
\paragraph{}RCC - \textit{Risk-Capital Coefficient}
\paragraph{}RF - \textit{Random Forest}
% \paragraph{}RMSE - \textit{Root Mean Squared Error}
\paragraph{}SVM - \textit{Support Vector Machine}
\paragraph{}TGARCH - \textit{Threshold Generalised ARCH}
\paragraph{}UFRJ - Universidade Federal do Rio de Janeiro

\pagebreak