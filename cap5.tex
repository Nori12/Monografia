\chapter{Conclusão}
\label{cap5}



\paragraph{} Neste trabalho, foi proposto a criação de uma estratégia de \textit{swing trade} com a utilização de técnicas de aprendizado de máquina. A partir dos resultados apresentados no Capítulo \ref{cap4}, foi possível observar que a estratégia proposta supera consideravelmente a estratégia \textit{baseline} em diversos aspectos: rendimento final, índice de Sharpe, índice de Sortino e volatilidade.

\paragraph{} Como não há consideração de proventos nos dados utilizados, a tendência é que ambas as estratégias apresentem uma performance real melhor do que a simulada. No entanto, a estratégia \textit{baseline} deve revelar um aumento de performance maior, pois os papéis adquiridos ficam sempre em posse durante todo o período de simulação.

\paragraph{} Por fim, os resultados encontrados mostram que o uso de modelos de aprendizado de máquina pode auxiliar investidores no mercado de ações. O presente trabalho considerou premissas pessimistas durante as simulações, o que traz perspectivas promissoras. Mesmo assim, é necessário e prudente um aprofundamento maior das técnicas utilizadas antes de uma implementação real.

