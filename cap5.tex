\chapter{Considerações Finais}
\label{cap5}



\FloatBarrier
\section{Conclusão}



\paragraph{} Neste trabalho, foi proposta a criação de uma estratégia de \textit{swing trade} com a utilização de técnicas de aprendizado de máquina. Para isso, foi projetada uma estrutura de \textit{software} que possibilitasse o manuseio desta estratégia de forma prática.

\paragraph{} A partir dos resultados apresentados no Capítulo \ref{cap4}, foi possível observar que a melhor estratégia apresentada possui indicadores de performance ainda distantes de seu \textit{baseline}. Mais especificamente, o índice de Sharpe foi de 0,67 enquanto o mesmo índice do \textit{baseline} foi de 1,15.

\paragraph{} Como não há consideração de proventos nos dados utilizados, a tendência é que ambas as estratégias apresentem uma performance real melhor do que a simulada. No entanto, a estratégia \textit{baseline} deve revelar um aumento de performance relativamente maior, pois os papéis adquiridos ficam sempre em posse durante todo o período de simulação, não deixando intervalos de tempo descobertos. Também se deve ressaltar que o presente trabalho considerou algumas premissas pessimistas durante as simulações, o que traz perspectivas ligeiramente mais promissoras.

\paragraph{} Por fim, os resultados encontrados mostram que o uso de modelos de aprendizado de máquina pode auxiliar investidores no mercado de ações, no entanto para superar a média do mercado são necessários estudos mais aprofundados antes de uma implementação real.



\FloatBarrier
\section{Trabalhos Futuros}



\paragraph{} Este trabalho se baseou em premissas e regras que moldaram uma estrutura na qual os modelos pudessem operar. No entanto, alguns dos valores escolhidos podem ser revisitados a fim de serem aprimorados. Dentre eles, podemos citar a escolha do valor de 45 dias para o período máximo de dias por operação. Ao invés de um único valor geral, pode-se considerar uma análise estatística para cada ativo e encontrar um valor que seja mais adequado individualmente.

\paragraph{} Deve-se mencionar também o parâmetro de razão entre ganho e perda, já que foi utilizado como uma constante de valor 3. Aqui também é cabível uma análise estatística para cada ativo a fim de se encontrar valores mais refinados.

\paragraph{} Não houve consideração de proventos durante a simulação das estratégias pela dificuldade de se obter uma base de dados gratuita e confiável. Portanto, vale mencionar que a incorporação desta base pode enriquecer os estudos aqui realizados.
