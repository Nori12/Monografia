\chapter{Conclusão}
\label{cap5}



\paragraph{} Neste trabalho, foi proposta a criação de uma estratégia de \textit{swing trade} com a utilização de técnicas de aprendizado de máquina. A partir dos resultados apresentados no Capítulo \ref{cap4}, foi possível observar que a estratégia proposta se aproxima bastante da estratégia \textit{baseline} nos aspectos: rendimento final, índice de Sharpe, índice de Sortino. Por outro lado, sua volatilidade foi relativamente maior.

\paragraph{} Como não há consideração de proventos nos dados utilizados, a tendência é que ambas as estratégias apresentem uma performance real melhor do que a simulada. No entanto, a estratégia \textit{baseline} deve revelar um aumento de performance maior, pois os papéis adquiridos ficam sempre em posse durante todo o período de simulação. Também se deve ressaltar que o presente trabalho considerou premissas pessimistas durante as simulações, o que traz perspectivas ligeiramente mais promissoras.

\paragraph{} Por fim, os resultados encontrados mostram que o uso de modelos de aprendizado de máquina pode auxiliar investidores no mercado de ações, no entanto a superação da média do mercado, representada pelo modelo \textit{baseline}, requer estudos mais aprofundados antes de uma implementação real.

