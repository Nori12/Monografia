\chapter{Introdução}
\label{cap1}

\section{Tema}

\paragraph{} O tema deste trabalho se resume no aperfeiçoamento de uma estratégia de swing trade na bolsa de valores através de métodos de aprendizado de máquina.

\paragraph{} Nesse contexto, o problema a ser abordado é a identificação do momento apropriado para compra de um determinado ativo tendo em vista uma variação positiva de seu preço.


\section{Delimitação}

\paragraph{} Este trabalho se limita aos ativos negociados na Bolsa de Valores de São Paulo, a B3, cujos dados diários são de domínio público e foram adquiridos através da plataforma Yahoo Finance por uma API open-source. A duração das operações tem em vista um horizonte mínimo de dias, sendo portanto operações de swing trade. Não são levadas em consideração informações sobre proventos (dividendos e juros sobre capital próprio) devido à dificuldade técnica para automatização da busca de tais dados.

\section{Justificativa}

\paragraph{} O crescimento do número de investidores na bolsa de valores brasileira\cite{aumento_investidores} demonstra um maior interesse da população na busca por um complemento da renda familiar ou até uma substituição da fonte de renda principal.

\paragraph{} No cenário global, o aumento do uso de robôs de trading (ou algoritmos) já vem se mostrando uma tendência\cite{robos_investidores}, sejam por pessoas físicas ou fundos de investimento, de forma total ou parcial. Por outro lado, tal crescimento não vem sendo igualmente representado no Brasil devido às pecualiaridades do mercado de capitais nacional, como a alta volatilidade e alta sensibilidade a notícias\cite{robos_e_fundos}.

\paragraph{} Paralelamente, estudos relacionados a aprendizado de máquina já vem trazendo resultados práticos no dia-a-dia das pessoas, desde o clássico exemplo do reconhecimento de mensagens de spam em um caixa de email à identificação do perfil de consumo de clientes em uma loja. Da mesma forma, instituições financeiras e bancos centrais também vem, com cautela, incorporando cada vez mais aplicações de aprendizado de máquina em tarefas internas\cite{ml_finantial_usage}.

\paragraph{} Apesar das dificuldades inerentes ao cenário atual do mercado de capitais brasileiro, não se pode ignorar o potencial que os algortimos podem trazer. Desta forma, o presente trabalho visa a união de técnicas de aprendizado de máquina à estratégias de trading de forma a trazer uma melhor performance, colaborando assim para uma maior variedade de opções de investimentos à população brasileira.

\section{Objetivos}

\paragraph{} O objetivo geral deste trabalho é implementar um software que simule uma estratégia de swing trade utilizada no mercado financeiro e gerar uma nova estratégia que utilize machine learning a fim de melhorar sua performance. Especificamente, o software deve: (1) Criar um ambiente automatizado que permita buscar, atualizar e armazenar dados diários da bolsa brasileira de forma simples e conforme necessidade do usuário da aplicação; (2) Simular a estratégia de swing trade do trader André Moraes da forma mais fidedigna que a janela de dados diária permita; (3) Criar e simular um novo algoritmo baseado na estratégia anterior utilizando aprendizado de máquina; (4) Criar e simular uma estratégia de benchmark referente à estratégia de machine learning; (5) Analisar os modelos gerados.

\section{Metodologia}

\paragraph{} TODO Como é a abordagem do assunto. Como foi feita a pesquisa, se
vai houve validação, etc. Em resumo, você de explicar qual foi sua estratégia
para atender ao objetivo do trabalho (tamanho do texto: livre).


\section{Descrição}

\paragraph{}No capítulo 2 será feita a apresentação introdutória sobre
privacidade, conceitos associados e suas relevâncias no contexto socioeconômico
contemporâneo.

\paragraph{}O capítulo 3 apresenta um panorama de ferramentas de anonimização
cuja licença garanta a abertura do código-fonte com o intuito de prospectar as
funcionalidades fornecidas por esses programas.
