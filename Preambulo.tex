% Declaracao
\begin{center}
Declaração de Autoria e de Direitos
\end{center}

\vspace{0.5cm}

Eu, \emph{Pedro Henrque Barbosa Nori} CPF \emph{134.129.077-82}, autor da monografia \emph{\titulo{}}, subscrevo para os devidos fins, as seguintes informações:\\
1. O autor declara que o trabalho apresentado na disciplina de Projeto de Graduação da Escola Politécnica da UFRJ é de sua autoria, sendo original em forma e conteúdo.\\
2. Excetuam-se do item 1. eventuais transcrições de texto, figuras, tabelas, conceitos e idéias, que identifiquem claramente a fonte original, explicitando as autorizaçõees obtidas dos respectivos proprietários, quando necessárias.\\
3. O autor permite que a UFRJ, por um prazo indeterminado, efetue em qualquer mídia de divulgação, a publicação do trabalho acadêmico em sua totalidade, ou em parte. Essa autorização não envolve ônus de qualquer natureza à UFRJ, ou aos seus representantes.\\
4. O autor pode, excepcionalmente, encaminhar à Comissão de Projeto de Graduação, a não divulgação do material, por um prazo máximo de 01 (um) ano, improrrogável, a contar da data de defesa, desde que o pedido seja justificado, e solicitado antecipadamente, por escrito, à Congregação da Escola Politécnica.\\
5. O autor declara, ainda, ter a capacidade jurídica para a prática do presente ato, assim como ter conhecimento do teor da presente Declaração, estando ciente das sanções e punições legais, no que tange a cópia parcial, ou total, de obra intelectual, o que se configura como violaçõo do direito autoral previsto no Código Penal Brasileiro no art.184 e art.299, bem como na Lei 9.610.\\
6. O autor é o único responsável pelo conteúdo apresentado nos trabalhos acadêmicos publicados, não cabendo à UFRJ, aos seus representantes,  ou ao(s) orientador(es), qualquer responsabilização/ indenização nesse sentido.\\
7. Por ser verdade, firmo a presente declaração.\\

      \vspace{0.5cm}
      \begin{flushright}
         \parbox{10cm}{
            \hrulefill

            \vspace{-.375cm}
            \centering{Pedro Henrique Barbosa Nori}

            \vspace{0.1cm}
         }
      \end{flushright}

\pagebreak

% Copyright
      \vspace{0.5cm}

UNIVERSIDADE FEDERAL DO RIO DE JANEIRO \\
Escola Politécnica - Departamento de Eletrônica e de Computação \\
Centro de Tecnologia, bloco H, sala H-217, Cidade Universitária \\
Rio de Janeiro - RJ      CEP 21949-900\\
\vspace{0.5cm}
\paragraph{}Este exemplar é de propriedade da Universidade Federal do Rio de Janeiro, que poderá incluí-lo em base de dados, armazenar em computador, microfilmar ou adotar qualquer forma de arquivamento.
\paragraph{}É permitida a menção, reprodução parcial ou integral e a transmissão entre bibliotecas deste trabalho, sem modificação de seu texto, em qualquer meio que esteja ou venha a ser fixado, para pesquisa acadêmica, comentários e citações, desde que sem finalidade comercial e que seja feita a referência bibliográfica completa.
\paragraph{}Os conceitos expressos neste trabalho são de responsabilidade do(s) autor(es).


\pagebreak

% Dedicat�ria
\begin{center}
\textbf{DEDICATÓRIA}
\end{center}
      \vspace{0.5cm}

\paragraph{} À minha mãe, engenheira mecânica.

\pagebreak


% Agradecimento
\begin{center}
\textbf{AGRADECIMENTO}
\end{center}
      \vspace{0.5cm}


% \paragraph{} Agrade\c{c}o ao meu pa\'is, que esconde um povo t\~ao sofrido e ao mesmo tempo t\~ao amoroso. Este trabalho é apenas um pedacinho do pagamento da minha d\'ivida.

\paragraph{} Agradeço à minha mãe Ana Christina e ao meu pai Adilson Nori por todo apoio e paciência que pude receber durante todos esses anos. Agradeço também ao meu irmão João por estar comigo nessa grande jornada da vida.

\paragraph{} Agradeço aos professores que tive a oportunidade de conhecer, a começar pelo Aridio Schiappacassa do CEFET/RJ por toda a paciência em me ajudar na montagem do meu primeira rádio FM e nos primeiros passos com PIC. Agradeço também aos meus professores da UFRJ, os quais guardo enorme respeito, carinho e admiração. Em especial: Casé, Heraldo, Luiz Wagner, Pino, Brafman, Teodósio, Wallace e Jomar.

\paragraph{} Deixo um grande abraço a todos os meus companheiros da equipe de robótica MinervaBots, onde tanto aprendi e tanto amei pertencer.

\paragraph{} A todas as experiências que puder compartilhar com meus amigos. Deixo um abraço especial para o meu amigo de infância Daniel Iunes Monteiro,  que infelizmente não se encontra mais nesta vida.

\paragraph{} Por fim, agradeço ao meu orientador Heraldo por me receber de braços abertos e por todo o suporte no cumprimento deste projeto.

\pagebreak


% Resumo
\begin{center}
\textbf{RESUMO}
\end{center}
      \vspace{0.5cm}

\paragraph{} Todos os dias, diversas negociações são realizadas nas bolsas de valores do mundo inteiro. Com os mais diversos objetivos, investidores buscam um aumento crescente de patrimônio de forma consistente. Paralelamente, inteligências artificias estão substituindo cada vez mais atividades antes desempenhadas pelo homem. Nesse sentido, este trabalho visa a aplicação de técnicas de aprendizado de máquina para a elaboração de uma estratégia de \textit{swing trade} no mercado acionário brasileiro. Para isso, é concebida uma estrutura de regras e premissas que criam uma base ao modelo de aprendizado de máquina, responsável por decidir o momento de entrada em operações a partir de um conjunto de dados. Ao final, algumas estratégias são simuladas e suas performances analisadas com um modelo \textit{baseline}.

\paragraph{}
\noindent Palavras-Chave: \textit{Machine Learning}, \textit{Random Forest}, Análise Técnica, \textit{Swing Trade}, Mercado Financeiro.

\pagebreak


% Abstract
\begin{center}
\textbf{ABSTRACT}
\end{center}
      \vspace{0.5cm}

\paragraph{} Every day, several negotiations are carried out on stock exchanges around the world. With the most diverse objectives, investors seek a consistently growing increase in equity. At the same time, artificial intelligences are increasingly replacing activities previously performed by man. In this context, this work aims at the application of machine learning techniques for the elaboration of a swing trade strategy in the Brazilian stock market. So, a structure of rules and assumptions is conceived to create a basis for the machine learning model, responsible for deciding when to enter into operations given a set of data. In the end, some strategies are simulated and its results compared to a \textit{baseline} model.

\noindent Key-words: Machine Learning, Random Forest, Technical Analysis, Swing Trade, Stock Market.

\pagebreak


% Siglas
\begin{center}
\textbf{SIGLAS}
\end{center}
% \vspace{0.5cm}

\paragraph{}AF - Análise Fundamentalista
\paragraph{}API - \textit{Application Programming Interface}
\paragraph{}ANN - \textit{Artificial Neural Networks}
% \paragraph{}ANS - Aprendizado Não Supervisionado
\paragraph{}ARCH - \textit{Autoregressive Conditional Heteroskedasticity}
\paragraph{}AS - Aprendizado Supervisionado
\paragraph{}AT - Análise Técnica
\paragraph{}B3 - Brasil, Bolsa, Balção
\paragraph{}CPU - \textit{Central Process Unit}
\paragraph{}CSL - \textit{Cost Sensitive Learning}
\paragraph{}CSV - \textit{Comma-separated values}
\paragraph{}CVM - Comissão de Valores Mobiliários
\paragraph{}DT - \textit{Decision Tree}
\paragraph{}EGARCH - \textit{Exponential Generalised ARCH}
\paragraph{}EMA - \textit{Exponential Moving Average}
\paragraph{}ETF - \textit{Exchange-Traded Fund}
\paragraph{}GARCH - \textit{Generalised ARCH}
\paragraph{}HME - Hipótese do Mercado Eficiente
\paragraph{}HMM - \textit{Hidden Markov Model}
\paragraph{}iBovespa - Índice Bovespa
\paragraph{}IPO - \textit{Initial Public Offering}
% \paragraph{}IBM - \textit{International Business Machines}
\paragraph{}IIR - \textit{Infinite Impulse Reponse}
\paragraph{}IL - Índice de Lucratividade
\paragraph{}JSON - \textit{JavaScript Object Notation}
\paragraph{}k-NN - \textit{K Nearest Neighbors}
\paragraph{}MACD - \textit{Moving Average Convergence/Divergence}
\paragraph{}ML - \textit{Machine Learning}
\paragraph{}MME - Média Móvel Exponencial
\paragraph{}NFO - Normalização por Frequência de Operações
\paragraph{}NGARCH - \textit{Non-linear Generalised ARCH}
\paragraph{}RCC - \textit{Risk-Capital Coefficient}
\paragraph{}RF - \textit{Random Forest}
\paragraph{}RI - Relações com Investidores
% \paragraph{}RMSE - \textit{Root Mean Squared Error}
\paragraph{}SVM - \textit{Support Vector Machine}
\paragraph{}TGARCH - \textit{Threshold Generalised ARCH}
\paragraph{}UFRJ - Universidade Federal do Rio de Janeiro
\paragraph{}WFA - \textit{Walk-Forward Analysis}

\pagebreak