\chapter{Metodologia}
\label{cap3}

\section{Resumo}

\paragraph{} As seções a seguir trazem detalhes quanto a estrutura técnica do projeto. Portanto, a Figura \ref{fig:100} mostra uma noção geral de como as estrutras se conectam.

\begin{figure}[h]
    \includegraphics[scale=0.90]{no_image.jpeg}
    \centering
    \caption{Estrutura do técnica do projeto (Imagem em construção)}
    \label{fig:100}
\end{figure}

\paragraph{} Primeiro, tem-se início a etapa de pré-processamento de dados, onde ocorre a leitura e interpretação do arquivo de configuração para se saber quantas estratégias executar, quais os ativos envolvidos e seus recpectivos intervalos de tempo. Uma vez verificado no banco de dados os dados já existentes, faz-se um \textit{download} apenas dos dados necessários. Se houver alguma atualização de dados, as \textit{features} de uso geral são calculadas e armazenadas no banco a fim de servir de insumo para as estratégias que estarão por vir.

\paragraph{} Completada a etapa de pré-processamento, inicia-se a simulação das estratégias. O arquivo de configuração foi projetado para ser capaz de designar diversas estratégias de parâmetros distintos a uma mesma ordem de execução de programa. Dessa forma, faz-se uso da biblioteca \textit{multiprocessing} para paralelizar as simulações, cujos resultados e estatísticas pertinentes são salvas no banco para posterior análise.

\paragraph{} É possível visualizar os resultados de forma clara através de uma aplicação secundária responsável por criar um \textit{dashboard} interativo.

\paragraph{} A aplicação foi desenvolvida em \textit{Python} com o apoio das bibliotecas \textit{yfinance}, \textit{pandas}, \textit{dash} e \textit{multiprocessing}. Foi estruturado um banco de dados \textit{PostgreSQL} para armazenamento dos \textit{candlesticks} baixados, das \textit{features} geradas e arquivamento das estratégias executadas. Também foi incorporado o uso de \textit{Docker} especificamente para a execução de estratégias sem necessidade de configuração de ambiente.

\section{Pré-Processamento}

\subsection{Arquivo de Configuração}
\paragraph{} EXCLUIR - Falar motivação do arquivo; Print de exemplo; Apontar para subsection de lista de parâmetros; Possibilidade de rodar várias simulações.

\paragraph{} O Arquivo de Configuração é um arquivo no formato JSON responsável por configurar detalhadamente cada parâmetro da sequência de estratégias que se deseja executar. Uma ordem de execução do programa pode conter diversas simulações de estratégias, que são configuradas neste Arquivo. A Figura \ref{fig:101} mostra sua estrutura.

\begin{figure}[h]
    \includegraphics[scale=0.90]{no_image.jpeg}
    \centering
    \caption{Arquivo de Configuração para execução singular (Imagem em construção)}
    \label{fig:101}
\end{figure}

\paragraph{} Nota-se que no topo são listados os parâmetros de uso geral, cujos valores precedem quaisquer outros listados na estratégias subsequentes, em caso de sobreposição. Em seguida abre-se o vetor de tipos de estratégias, onde cada campo \textit{name} representa o nome da classe selecionada, sendo este o elemento que conecta o usuário ao tipo de estratégia desejada. Após a seleção do nome, são configurados os parâmetros internos da estratégia, que podem ser de natureza obrigatória ou opcional. A Tabela \ref{tab:1} descreve todos os parâmetros disponíveis.

\paragraph{} Para se criar mais de um perfil de simulação, é necessário modificar o parâmetro divergente para uma lista de valores, conforme a Figura cite{fig:102}. Automaticamente, o código interpreta que existe mais de uma simulação a executar, com todos os parâmetros em comum exceto aqueles em formato de listas. Caso haja mais de um parâmetro no formato de lista, seus comprimentos precisam ser iguais.

\begin{figure}[h]
    \includegraphics[scale=0.90]{no_image.jpeg}
    \centering
    \caption{Arquivo de Configuração para execução múltipla (Imagem em construção)}
    \label{fig:102}
\end{figure}

\begin{center}
\resizebox{\textwidth}{!}{%
    \begin{tabular}{ | m{10em} | m{3em}| m{5em}| m{16em} | }
        \hline
        \multicolumn{4}{|c|}{Lista de Parâmetros} \\
        \hline
        Nome do Parâmetro & Escopo & Obrigatório & Descrição \\
        \hline
        show\_results & Geral & Não & Exibe \textit{dashboard} da última simulação completada ao final. Tipo: Boolean. Valor \textit{default}: True. \\
        \hline
        min\_risk\_features & Geral & Não & Mínimo risco para o cálculo de \textit{features}. Tipo: Float. Valor \textit{default}: 0.01. \\
        \hline
        max\_risk\_features & Geral & Não & Máximo risco para o cálculo de \textit{features}. Tipo: Float. Valor \textit{default}: 0.10. \\
        \hline
        name & Local & Sim & Nome da estratégia a ser executada. Valores válidos: "ML Derivation". Tipo: \textit{String}. \\
        \hline
        comment & Local & Não & Comentário. Tipo: \textit{String}. \\
        \hline
        capital & Local & Sim & Capital total da carteira em reais. Tipo: \textit{Float}. \\
        \hline
        risk\_capital\_coefficient & Local & Sim & Coeficiente de risco-capital geral. Tipo: \textit{Float}. \\
        \hline
        tickers\_bag & Local & Não & Grupo de ativos a escolher dentro do parâmetro stock\_targets. "listed\_first": ordem de listagem; "random": ordem aleatória. Valor \textit{default}: "listed\_first". \\
        \hline
    \end{tabular}
    \label{tab:1}
}
\end{center}

Terminar.




\subsection{Coleta de Dados}

\paragraph{} EXCLUIR - yfinance; Problema com proventos; Normalização de proventos do preço das ações.

\paragraph{} A Coleta de Dados ocorre através da biblioteca \textit{open-source} \textit{yfinance} \cite{yfinance}, uma ferramenta de terceiros que transmite dados públicos da plataforma \textit{Yahoo! Finance} \cite{yahoo_finance}, um subsistema da rede \textit{Yahoo!}.

\paragraph{} A escolha desta biblioteca como fonte de dados se deve à sua gratuidade e facilidade de uso, porém após alguns tester e verificações com outras fontes, deve-se ressaltar algumas questões.

\begin{itemize}
    \item Os valores de proventos que a biblioteca disponibiliza não são consistentes, portanto não são utilizados por este projeto. Testes internos confirmaram a presença de diversos valores de dividendos e juros sobre capital próprio corretamente apresentados e ajustados pelos desdobramentos acumulados, porém somados a alguns \textit{outliers} inexistentes na realidade, o suficiente para questionar seu uso em escala (i.e., para vários ativos sem verificação individual).
    \item DIA DE MEIO EXPEDIENTE NO CARNAVAL
    \item PROVENTOS EMBUTIDOS NO PREÇO DAS AÇOES?
    \item VERIFICACAO COM TRADING VIEW
\end{itemize}

\paragraph{} Os dados obtidos são \textit{candlesticks} diários (OHLCV). Com a mesma facilidade, é possível adquirir janelas de tempo semanais, no entanto para evitar potenciais problemas de consistência de dados, as mesmas são calculadas internamente a partir da janela diária via comandos SQL\footnote{\textit{Structured Query Language}: Linguagem usada para administrar bancos de dados relacionais.}.







\subsection{Armazenamento de Dados}
\paragraph{} Banco de dados; Criação de Candles semanais.

\subsection{Geração de Features de Uso Geral}
\paragraph{} Quais features; Cuidados com não-causalidade.


\section{Simulação de Estratégia}

\subsection{Estrutura}
\paragraph{} Carteira com N ativos de datas distintas; Regra de 3 para 1 entre stop e alvo; 1 operação por ativo.

\subsection{Premissas}
\paragraph{} Compra na abertura do merdado; Sem venda no dia da compra; Prioridades durante venda (stop primeiro).

\subsection{Período Máximo de Dias por Operação}
\paragraph{} Motivação da escolha dos 45 dias; Gráfico entre ABEV e MGLU.

\subsection{Gerenciamento de Risco}
\paragraph{} Coeficiente de Risco-Capital

\subsection{Risco de Entrada por Operação}
\paragraph{} Cálculo do risco mínimo; Cálculo do risco Máximo.

\subsection{Descanso por Tendência de Baixa}
\paragraph{}

\subsection{Descanso por Identificação de Crises}
\paragraph{}

\subsection{Lista de Parâmetros de Configuração}
\paragraph{} Lista todos e explicar o que fazem.

\subsection{Ensaios Paralelos}
\paragraph{} Parâmetros que estão implementados e não trouxeram resultados expressivos.



\section{Otimizações de Gerenciamento de Carteira}

\subsection{Normalização por Frequência de Operações}
\paragraph{}

\subsection{Controle Proporcional para Uso de Capital}
\paragraph{}

\subsection{Compensação por Lucratividade}
\paragraph{}



\section{Criação de Modelos}

\subsection{Resumo}
\paragraph{}

\subsection{\textit{Feature Selection}}
\paragraph{}

\subsection{Geração de \textit{Datasets}}
\paragraph{}

\subsection{\textit{Walk Forward Optimization}}
\paragraph{}

\subsection{Critérios de Escolha}
\paragraph{}


\section{Análise de Resultados}
\paragraph{} Dashboard; Baseline