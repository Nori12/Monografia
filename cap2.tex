\chapter{Fundamentação Teórica}
\label{cap2}

\paragraph{} Neste capítulo, são introduzidos alguns conceitos chave para o entendimento do projeto. Nas próximas seções, são feitas contextualizações sobre o Mercado de Capitais, Bolsa de Valores, Ações e Aprendizado de Máquina.

\section{Mercado de Capitais, Bolsa de Valores e Ações}

\paragraph{} O Mercado de Capitais, também conhecido com Mercado de Valores Mobiliários, é um dos segmentos do sistema financeiro responsável por fazer o intermédio entre agentes superávitarios e agentes deficitários através da compra e venda valores mobiliários (i.e., ativos financeiros)\cite{mercado_de_capitais}. Consequentemente, há uma maior liquidez destes ativos e uma melhora no fluxo de capitais entre agentes econômicos, sejam eles os governos por meio dos bancos centrais, os bancos privados, as insituições financeiras ou até mesmo as pessoas físicas.

\paragraph{} No Brasil, o Mercado de Capitais é regulado e fiscalizado pela CVM (Comissão de Valores Mobiliários), uma autarquia federal vinculada ao Ministério da Fazenda e criada em 1976 através da Lei nº 6.385\cite{lei_6385}.

\paragraph{} A Bolsa de Valores é plataforma onde se negociam valores mobiliários do Mercado de Capitais, dentre eles ações de sociedades de capital aberto. No Brasil, a única Bolsa de Valores oficial existente é a B3 (Brasil, Bolsa, Balcão)\cite{b3}, que administra os sistemas de negociação, compensação, liquidação, depósito e registro para todas as principais classes de ativos.

\paragraph{}

\subsection{Teste 1}

\paragraph{}

\subsection{Teste 2}

\paragraph{}

\subsection{Teste 3}

\paragraph{}

\section{Aprendizado de Máquina}

\subsection{Teste 4}

\paragraph{}


\section{Trabalhos Relacionados}