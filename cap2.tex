\chapter{Fundamentação Teórica}
\label{cap2}

\paragraph{} Neste capítulo, são introduzidos alguns conceitos chave para o entendimento do projeto. Nas próximas seções, são feitas contextualizações sobre o Mercado de Capitais, Bolsa de Valores, Ações e Aprendizado de Máquina.

\section{Mercado de Capitais, Bolsa de Valores e Ações}

\paragraph{} O Mercado de Capitais, também conhecido como Mercado de Valores Mobiliários, é um dos segmentos do sistema financeiro responsável por fazer o intermédio entre agentes superávitarios, que tem capital de investimento, e agentes deficitários, que buscam capital para rentabilizá-lo, através da compra e venda valores mobiliários (i.e., ativos financeiros) \cite{mercado_de_capitais}. Consequentemente, gera-se uma maior liquidez destes ativos e também uma melhora no fluxo de capitais entre os agentes econômicos, sejam eles os governos por meio dos bancos centrais, os bancos privados, as insituições financeiras ou até mesmo as pessoas físicas.

\paragraph{} No Brasil, o Mercado de Capitais é regulado e fiscalizado pela CVM (Comissão de Valores Mobiliários), uma autarquia federal vinculada ao Ministério da Fazenda e criada em 1976 através da Lei nº 6.385 \cite{lei_6385}.

\paragraph{} A Bolsa de Valores é uma plataforma onde se negociam os valores mobiliários do Mercado de Capitais, dentre eles ações (i.e., fatias, pedaços) de sociedades anônimas (ou companhias). No Brasil, a única Bolsa de Valores oficial existente é a B3 (Brasil, Bolsa, Balcão) \cite{b3}, que administra os sistemas de negociação, compensação, liquidação, depósito e registro para todas as principais classes de ativos.

\paragraph{} O processo de abertura de capital de uma empresa é uma iniciativa que possui vantanges estratégicas \cite{vantagens_sa} como: o aumento da confiança na perspectiva do mercado, seja para o consumidor final ou para parceiros comerciais; a solução de problemas decorrentes de processos sucessórios; e também a captação de capital de investimento, a fim de contribuir para o crescimento ou para a consolidação da companhia. Esse processo acontece através de uma oferta pública \cite{oferta_publica}, ou IPO (Initial Public Offering), onde as ações que compõe o capital social \cite{capital_social} de uma companhia são vendidas pela primeira vez ao público geral. Uma vez encerrado o IPO, estas mesmas ações passam para o mercado secundário \cite{mercado_secundario}, onde investidores as negociam entre si. Em retorno ao capital adquirido pela companhia, surgem algumas responsabilidades, dentre elas a publicação de demonstrações financeiras \cite{dem_finan}, auditadas pela própria CVM \cite{audi_dem_finan}.

\paragraph{} Para o acionista de uma sociedade anônima, existem duas formas de se obter lucro: através de proventos (dividendos e juros sobre capital próprio) \cite{proventos}; ou através de operações de compra e de venda de ações, mediante oscilações de seu valor de mercado. Conforme a expectavida corretamente induz, o lucro é comumente aferido durante a venda de um determinado papél (i.e., ação) posteriormente à sua aquisição a um preço de compra inferior. No entando, também é possível trabalhar com posições vendidas (short selling) \cite{short_selling}, onde um investidor aluga ações de outro investidor por meio de um contrato. Em seguida as vende para posteriormente recomprá-las a um preço inferior, devolvendo-as assim ao respectivo dono. Neste caso, o lucro é obtido quando expectativa de queda de um ativo se mostra verdadeira.

\subsection{Hipótese do Mercado Eficiente}

\paragraph{} A Hipótese do Mercado Eficiente, definida por FAMA \cite{hme}, afirma que idealmente o preço de um ativo reflete toda a informação disponível sobre seu valor intrínseco. Em outras palavras, quanto menor o efeito de fatores que contribuam para uma inércia no fluxo de capital de investidores e na transmissão de informações, mais o mercado tende a ser eficiente. São estudados os três níveis de hipóteses:

\begin{itemize}
    \item HME fraca: O preços atuais refletem o todo o histórico de informações disponibilizados publicamente.
    \item HME semi-forte: Engloba a HME fraca, acrescentando a existência de uma mudança instantânea que os preços sofrem ao surgirem novas informações.
    \item HME forte: Engloba a HME semi-forte, porém entende que a mudança instantânea dos preços acompanha toda e qualquer informação existente sobre o ativo. Assim, absolutamente nenhum investidor conseguiria obter lucro superior a outro, pois não há como acessar nenhuma informação privilegiada, uma vez que ela já estaria refletido no preço corrente do ativo.
\end{itemize}

\paragraph{} O autor menciona que o HME forte não é estritamente válida na realidade, o que é uma afirmação coerente quando se verifica a existência de casos em que o vazamento de informações confidenciais trouxe aos acusados uma lucratividade significativa \cite{insider_trading}.

\paragraph{} A HME fraca foi verificada devido à consistência da correlação dos preços dia após dia de determinadas ações, mesmo que esta fosse baixa.

\paragraph{} A hipótese semi-forte também foi sustentada por alguns fatores, dentre eles a verificação de que os futuros pagamentos de dividendos das companhias se refletem, em média, no preços das ações \cite{new_info_in_stock}.

\paragraph{} Em resumo, o estudo das Hipóteses de Mercado Eficiente traz informações relevantes quanto se avalia a possibilidade da aplicação de estratégias de trading no mercado financeiro. No entanto, é importante ressaltar que outros autores questionam algumas das premissas levantadas por FAMA (citar), deixando assim um consenso mínimo de ...

\subsection{Índice de Bolsa de Valores}

\paragraph{}

\subsection{Análise Técnica e Análise Fundamentalista}

\paragraph{} A variação do preço das ações listadas na Bolsas de Valores mundiais é decorrente de diversos fatores

\paragraph{}

\section{Aprendizado de Máquina}

\subsection{Teste 4}

\paragraph{}


\section{Trabalhos Relacionados}