\chapter{Fundamentação Teórica}
\label{cap2}

\paragraph{} Neste capítulo, são introduzidos alguns conceitos chave para o entendimento do projeto. Nas próximas seções, são feitas contextualizações sobre o Mercado de Capitais, Bolsa de Valores, Ações e Aprendizado de Máquina.

\section{Mercado de Capitais, Bolsa de Valores e Ações}

\paragraph{} O Mercado de Capitais, também conhecido com Mercado de Valores Mobiliários, é um dos segmentos do sistema financeiro responsável por fazer o intermédio entre agentes superávitarios e agentes deficitários através da compra e venda valores mobiliários (i.e., ativos financeiros) \cite{mercado_de_capitais}. Consequentemente, há uma maior liquidez destes ativos e uma melhora no fluxo de capitais entre agentes econômicos, sejam eles os governos por meio dos bancos centrais, os bancos privados, as insituições financeiras ou até mesmo as pessoas físicas.

\paragraph{} No Brasil, o Mercado de Capitais é regulado e fiscalizado pela CVM (Comissão de Valores Mobiliários), uma autarquia federal vinculada ao Ministério da Fazenda e criada em 1976 através da Lei nº 6.385 \cite{lei_6385}.

\paragraph{} A Bolsa de Valores é plataforma onde se negociam valores mobiliários do Mercado de Capitais, dentre eles ações sociedades anônimas (companhias). No Brasil, a única Bolsa de Valores oficial existente é a B3 (Brasil, Bolsa, Balcão) \cite{b3}, que administra os sistemas de negociação, compensação, liquidação, depósito e registro para todas as principais classes de ativos.

\paragraph{} O processo de abertura de capital de uma empresa é uma iniciativa que possui vantanges estratégicas \cite{vantagens_sa} como: o aumento da confiança na perspectiva do mercado, seja para o consumidor final ou para parceiros comerciais; a solução de problemas decorrentes de processos sucessórios; e também a captação de capital de investimento, a fim de contribuir para o crescimento ou para a consolidação da companhia. Esse processo acontece através de uma oferta pública \cite{oferta_publica}, ou IPO (Initial Public Offering), onde as ações (i.e., fatias, pedaços) que compõe o capital social \cite{capital_social} da companhia são vendidas pela primeira vez ao público geral. Uma vez encerrado este processo, as ações são negociadas no mercado secundário \cite{mercado_secundario}, onde investidores passam a negociá-las entre si. Em retorno ao capital adquirido pela companhia, surgem algumas responsabilidades, dentre elas a publicação de demonstrações financeiras \cite{dem_finan}, auditadas pela própria CVM \cite{audi_dem_finan}.

\paragraph{} Para o acionista de uma sociedade anônima, existem duas formas de se obter lucro: através de proventos (dividendos e juros sobre capital próprio) \cite{proventos}; ou através de operações de compra e de venda de ações, mediante oscilações de seu valor de mercado. Conforme a expectavida corretamente induz, o lucro é comumente aferido durante a venda de um determinado papél (i.e., ação) posteriormente à sua aquisição por um preço de compra inferior. No entando, também é possível trabalhar com posições vendidas (short selling) \cite{short_selling}, onde um investidor aluga ações de outro investidor por meio de um contrato e em seguida as vende para posteriormente recomprá-las a um preço inferior, devolvendo-as assim ao respectivo dono. Neste caso, o lucro é obtido quando expectativa de queda de um ativo se mostra verdadeira.

\paragraph{} Falar sobre HME

\subsection{Teste 1}

\paragraph{}

\subsection{Teste 2}

\paragraph{}

\subsection{Teste 3}

\paragraph{}

\section{Aprendizado de Máquina}

\subsection{Teste 4}

\paragraph{}


\section{Trabalhos Relacionados}