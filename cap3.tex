\chapter{Metodologia}
\label{cap3}

\section{Resumo}

\paragraph{} As seções a seguir trazem detalhes quanto a estrutura técnica do projeto. Portanto, a Figura \ref{fig:100} apresenta uma noção geral de como as estrutras se conectam.

\begin{figure}[h]
    \includegraphics[scale=0.52]{resumo_projeto.png}
    \centering
    \caption{Estrutura do técnica do projeto}
    \label{fig:100}
\end{figure}

\paragraph{} Primeiro, antes da execução do código principal, é necessário garantir que os modelos estão devidamente localizados em pasta apropriada. Para isso, faz-se imprescindível a criação dos \textit{datasets} para cada ação a ser simulada, pois servem de entrada de dados para a criação e seleção dos modelos, etapa esta que deve ser executada logo em sequência. A biblioteca \textit{multiprocessing} foi utilizada para minimizar o tempo total gasto nestas etapas.

\paragraph{} Após a criação dos modelos, tem-se início a etapa de pré-processamento de dados, onde ocorre a leitura e interpretação do arquivo de configuração para se obter o número de estratégias a executar, quais os ativos envolvidos e seus recpectivos intervalos de tempo. Uma vez verificado no banco os dados já existentes, faz-se um \textit{download} apenas dos dados necessários. Se houver alguma atualização de dados, as \textit{features} de uso geral são recalculadas e armazenadas no banco a fim de servir de insumo para as estratégias que estarão por vir.

\paragraph{} Completada a etapa de pré-processamento, inicia-se a simulação das estratégias. O arquivo de configuração foi projetado para ser capaz de designar diversas estratégias de parâmetros distintos a uma mesma ordem de execução de programa. Também fez-se uso da biblioteca \textit{multiprocessing} para paralelizar as simulações, cujos resultados e estatísticas são salvas no banco para posterior análise.

\paragraph{} Por fim, é possível visualizar os resultados de forma clara através de uma aplicação secundária responsável por criar um \textit{dashboard} interativo.

\paragraph{} Em relação às tecnologias utilizadas, a aplicação foi desenvolvida em \textit{Python} com o apoio das bibliotecas \textit{yfinance}, \textit{pandas}, \textit{dash} e \textit{multiprocessing}. Foi estruturado um banco de dados \textit{PostgreSQL} para armazenamento dos \textit{candlesticks} obtidos, das \textit{features} geradas e das estratégias simuladas. Também foi incorporado o uso de \textit{Docker} especificamente para a execução de estratégias sem a necessidade de configuração de ambiente.

\section{Pré-Processamento}

\subsection{Arquivo de Configuração}

\paragraph{} O Arquivo de Configuração é um arquivo no formato JSON responsável por configurar detalhadamente cada parâmetro da sequência de estratégias que se deseja executar. Uma ordem de execução do programa pode conter diversas simulações de estratégias, que são configuradas neste Arquivo. A Figura \ref{fig:101} mostra sua estrutura.

\begin{figure}[h]
    \includegraphics[scale=0.50]{config_file_estrutura.png}
    \centering
    \caption{Estrutura do Arquivo de Configuração}
    \label{fig:101}
\end{figure}

\paragraph{} Nota-se que no topo são listados os parâmetros de uso geral, ou variáveis de escopo global, de cujos valores precedem quaisquer outros listados a seguir, em caso de sobreposição. Em seguida abre-se o vetor de tipos de estratégias, onde o campo \textit{name} representa o nome da classe selecionada, sendo este o elemento que conecta o usuário ao tipo de estratégia desejada. Após a seleção do nome, são configurados os parâmetros internos da estratégia. A Tabela \ref{tab:3} descreve todos os parâmetros disponíveis.

\paragraph{} Para se criar mais de um perfil de simulação, é necessário modificar o Arquivo conforme a Figura \ref{fig:102}. Automaticamente, o código interpreta que existe mais de uma simulação a executar, com todos os parâmetros em comum exceto aqueles em formato de listas. Caso haja mais de um parâmetro no formato de lista, seus comprimentos precisam ser iguais. No caso da Figura \ref{fig:102}, a primeira simulação utilizará os valores (100, 0.01) para o par (variável\_local\_1, variável\_local\_2), a segunda utilizará (200, 0.02) e assim sucessivamente.

\begin{figure}[h]
    \includegraphics[scale=0.50]{config_file_mult_exec.png}
    \centering
    \caption{Arquivo de Configuração para Execuções Múltiplas}
    \label{fig:102}
\end{figure}

\begin{center}
    {\small
    \begin{longtable}[m]{| m{11em} | m{3em}| m{21em} |}

        \hline
        \multicolumn{3}{|c|}{Lista de Parâmetros} \\
        \hline
        Nome do Parâmetro & Escopo & Descrição \\
        \hline
        \endfirsthead

        \hline
        \multicolumn{3}{|c|}{Continuação da Tabela \ref{tab:3}} \\
        \hline
        Nome do Parâmetro & Escopo & Descrição \\
        \hline
        \endhead

        \hline
        \endfoot

        \hline
        \multicolumn{3}{|c|}{Fim da Tabela \ref{tab:3}} \\
        \hline
        \caption{Lista de parâmetros detalhados.\label{tab:3}}
        \endlastfoot

        \hline
        show\_results & Geral & Exibe \textit{dashboard} da última simulação completada ao final. Tipo: \textit{Boolean}. \textit{Default}: \textit{True}. Listável: Não. \\
        \hline
        min\_risk\_features & Geral & Risco mínimo para o cálculo de \textit{features}. Tipo: \textit{Float}. \textit{Default}: 0,01. Listável: Não. \\
        \hline
        max\_risk\_features & Geral & Risco máximo para o cálculo de \textit{features}. Tipo: \textit{Float}. \textit{Default}: 0,10. Listável: Não. \\
        \hline
        \textbf{name} & Local & \textbf{(OBRIGATÓRIO)} Nome da estratégia a ser executada. Valores válidos: ``ML Derivation". Tipo: \textit{String}. Listável: Não. \\
        \hline
        comment & Local & Comentário. Tipo: \textit{String}. \textit{Default}: \textit{String} vazia. Listável: Não. \\
        \hline
        capital & Local & Capital total da carteira em reais (R\$). Tipo: \textit{Float}. Listável: Sim. \\
        \hline
        risk\_capital\_coefficient & Local & Coeficiente de risco-capital (RCC) geral. Tipo: \textit{Float}. \textit{Default}: 0,001. Listável: Sim. \\
        \hline
        tickers\_bag & Local & Grupo de ativos a escolher dentro de ``stock\_targets". Valores aceitos: ``listed\_first" (ordem de listagem); ``random" (ordem aleatória). \textit{Default}: ``listed\_first". Listável: Sim. \\
        \hline
        tickers\_number & Local & Número de ativos a escolher dentro de ``stock\_targets", de acordo com ``tickers\_bag". Tipo: \textit{Int}. \textit{Default}: 0 (todos). Listável: Sim. \\
        \hline
        min\_order\_volume & Local & Volume mínimo por operação. Tipo: \textit{Int}. \textit{Default}: 1. Listável: Sim. \\
        \hline
        gain\_loss\_ratio & Local & Razão entre ganho e perda. Para uma unidade de risco (delta pencentual entre preço de compra e \textit{stop loss}) são utilizadas N unidades de risco acima no preço preço de compra para definir o preço alvo. Tipo: \textit{Float}. \textit{Default}: 3. Listável: Sim. \\
        \hline
        max\_days\_per\_operation & Local & Número máximo de dias por operação. Inclui o dia de compra. Caso excedido, ocorre venda compulsória pelo preço de fechamento no último dia da contagem. Tipo: \textit{Int}. \textit{Default}: 45. Listável: Não. \\
        \hline
        min\_risk & Local & Risco mínimo por operação. Tipo: \textit{Float}. \textit{Default}: 0,003. Listável: Sim. \\
        \hline
        max\_risk & Local & Risco máximo por operação. Tipo: \textit{Float}. \textit{Default}: 0,10. Listável: Sim. \\
        \hline
        max\_risk & Local & Risco máximo por operação. Tipo: \textit{Float}. \textit{Default}: 0,10. Listável: Sim. \\
        \hline
        enable\_frequency\hspace{2em} \_normalization & Local & Uso de normalização por frequência de operações. Ativos com N vezes mais operações que a média receberão N vezes menos capital. Ver Seção \ref{freq_norm}. Tipo: \textit{Boolean}. \textit{Default}: \textit{False}. Listável: Sim. \\
        \hline
        enable\_profit\hspace{4em} \_compensation & Local & Uso de compensação por lucratividade acumulada. Ver Seção \ref{profit_comp}. Tipo: \textit{Boolean}. \textit{Default}: \textit{False}. Listável: Sim. \\
        \hline
        enable\_crisis\_halt & Local & Bloqueio de novas aquisições em caso de identificação de potenciais crises financeiras (para ativo). Ver Seção \ref{crisis_halt}. Tipo: \textit{Boolean}. \textit{Default}: \textit{False}. Listável: Sim. \\
        \hline
        enable\_downtrend\_halt & Local & Bloqueio de novas aquisições em caso de identificação de tendências de baixo nos preços (para ativo). Ver Seção \ref{downtrend_halt}. Tipo: \textit{Boolean}. \textit{Default}: \textit{False}. Listável: Sim. \\
        \hline
        enable\_dynamic\_rcc & Local & Uso de Coeficiente de Risco-Capital dinâmico (para carteira). Ver Seção \ref{dynamic_rcc}. Tipo: \textit{Boolean}. \textit{Default}: \textit{False}. Listável: Sim. \\
        \hline
        dynamic\_rcc\_reference & Local & Valor de referência de uso de capital médio no controle do RCC dinâmico. Ver Seção \ref{dynamic_rcc}. Tipo: \textit{Float}. \textit{Default}: 0,80. Listável: Sim. \\
        \hline
        dynamic\_rcc\_k & Local & Valor do ganho proporcional K no controle do RCC dinâmico. Ver Seção \ref{dynamic_rcc}. Tipo: \textit{Float}. \textit{Default}: 3. Listável: Sim. \\
        \hline

        purchase\_margin & Local & Margem percentual aplicada ao valor de compra. Ex: Se o alvo de compra estiver configurado para R\$100, uma margem de 1\% permitirá a compra antecipada em R\$99. Tipo: \textit{Float}. \textit{Default}: 0. Listável: Sim. \\
        \hline
        stop\_margin & Local & Margem percentual aplicada ao valor do \textit{stop loss}. Ex: Se o \textit{stop} estiver configurado para R\$100, uma margem de 1\% permitirá a compra antecipada em R\$101. Tipo: \textit{Float}. \textit{Default}: 0. Listável: Sim. \\
        \hline
        partial\_sale & Local & Uso de saídas parciais. Tipo: \textit{Boolean}. \textit{Default}: \textit{False}. Listável: Sim. \\
        \hline
        stop\_type & Local & Tipo de \textit{stop loss} utilizado. Valores aceitos: ``normal"; ``staircase" (para cada patamar de unidade de risco que o preço atinge acima do valor de compra, o \textit{stop} sobe igualmente, até uma unidade de risco abaixo do preço alvo). Ver ``gain\_loss\_ratio". \textit{Default}: ``normal". Listável: Sim. \\
        \hline
        min\_days\_after\_successful \_operation & Local & Mínimo de dias sem novas aquisições após operação de sucesso, para cada ação. Ex: para 1 dia mínimo, se a última venda de sucesso ocorreu durante o dia X, a próxima compra só ocorrerá a partir do dia X+2, inclusive. Tipo: \textit{Int}. \textit{Default}: 0. Listável: Sim. \\
        \hline
        max\_days\_after\_failure \_operation & Local & Mínimo de dias sem novas aquisições após operação de falha, para cada ação. Ex: para 1 dia mínimo, se a última venda de falha ocorreu durante o dia X, a próxima compra só ocorrerá a partir do dia X+2, inclusive. Tipo: \textit{Int}. \textit{Default}: 0. Listável: Sim. \\
        \hline

        \textbf{stock\_targets} & Local & \textbf{(OBRIGATÓRIO)} \textit{Array} de ações a incluir na carteira. Formato indicado pela Figura \ref{fig:101}. Atenção ao parâmetro ``tickers\_bag". \\
        \hline

    \end{longtable}}
\end{center}



\subsection{Coleta de Dados}
\label{coleta_de_dados}

\paragraph{} A Coleta de Dados ocorre através da biblioteca \textit{open-source} \textit{yfinance} \cite{yfinance}, uma ferramenta não oficial que transmite dados públicos da plataforma \textit{Yahoo! Finance} \cite{yahoo_finance}, um subsistema da rede \textit{Yahoo!}.

\paragraph{} A escolha desta biblioteca como fonte primária de dados se deve principalmente pela ausência de custos associada à facilidade de uso. Contudo, alguns testes e verificações com outras fontes de dados evidenciaram destantagens significativas, porém não impeditivas para seu uso. São elas:

\begin{itemize}
    \item Os valores de proventos (i.e., dividendos e juros sobre capital próprio) que a biblioteca disponibiliza não são consistentes para a B3, portanto não são utilizados por este projeto. Testes internos confirmaram a presença de diversos proventos corretamente apresentados e ajustados pelos respectivos desdobramentos acumulados, porém somados a alguns \textit{outliers} inexistentes na realidade, o suficiente para questionar seu uso em escala (i.e., para vários ativos sem verificação individual). \color{red} HERALDO: Devo mostrar evidências do teste que corrobora esta afirmação? \colorend

    \item Os volumes de negociação disponibilizados não necessariamente coincidem com a plataforma TradingView em valores absolutos, porém coincidem em valores relativos (i.e., variação de volume dia após dia para um mesmo ativo), o que é suficiente para este trabalho. \color{red} HERALDO: (1) Na verdade, encontrei algumas evidências de que os valores relativos conferem, mas nenhum evidência de que não conferem. (2) Será que posso citar a plataforma TradingView? Ou melhor, devo tomar algum cuidado? \colorend

    \item \textit{Candlesticks} de janelas temporais inferiores à diária (\textit{intraday}) são disponibilizados, porém as limitações envolvidas inviabilizam seu uso, como: o limite de 730 dias para a busca dos dados; a inconsistência com os dados diários quanto ao volume; e a alguns \textit{bugs} como a ausência de \textit{candlesticks} em todo dia de parcial do pregão da B3 (Quarta-feira de Cinzas).
\end{itemize}

\paragraph{} Os dados obtidos são \textit{candlesticks} diários (OHLCV). Com a mesma facilidade, é possível adquirir janelas de tempo semanais, no entanto para evitar potenciais problemas de consistência de dados, as mesmas são calculadas internamente a partir da janela diária via comandos SQL\footnote{\textit{Structured Query Language}: Linguagem usada para administrar bancos de dados relacionais.}.

\paragraph{} Apenas os dados não existentes no banco são baixados via \textit{yfinance}. Para isso, um \textit{trigger}\footnote{Procedimento armazenado em um banco de dados que é chamado automaticamente sempre que ocorre um evento.} é acoplado às tabelas de \textit{candlesticks} e acionado sempre que operações de \textit{insert}, \textit{update}, \textit{delete} e \textit{truncate} são realizadas. Quando ativado, chama uma função responsável por atualizar a tabela de \textit{status}, que registra o intervalo de tempo representado nas tabelas de \textit{candlesticks} para cada \textit{ticker} envolvidos. Deve-se ressaltar que os devidos cuidados foram tomados para evitar buracos entre intervalos de tempo não adjacentes. Portanto, apenas uma consulta à tabela de \textit{status} é executada para se verificar a necessidade de \textit{download} de novos dados.



\subsection{Armazenamento de Dados}

\paragraph{} O Armazenamento de Dados é realizado por um banco de dados \textit{PostgreSQL}, criado com o objetivo de salvar: os resultados das simulações; as \textit{features} de uso geral; e os \textit{candlesticks} obtidos. As vantagens de um banco de dados em relação a um arquivo CSV ou a uma planilha de Excel dispensam comentários. Contudo, quanto ao escopo deste trabalho, pode-se mencionar os seguintes pontos:

\begin{itemize}
    \item Fácil acesso aos resultados das simulações de forma estruturada e consistente, recurso este utilizado pela aplicação que gera o \textit{dashboard} de resultados.
    \item Economia de processamento devido ao armazenamento das \textit{features} de uso geral, uma vez que estratégias simuladas não necessitam recalculá-las a cada execução.
    \item Independência da plataforma \textit{Yahoo! Finance} para o caso de não continuidade dos dados ou qualquer alteração repentida.
    \item Diminuição do tráfego na rede pela persistência dos \textit{candlesticks} já obtidos.
\end{itemize}

\paragraph{} Os \textit{candlesticks} semanais são calculados via \textit{query} SQL para garantir a consistência dos dados, já que a possibilidade de inconsistência se fez presente entre dados \textit{intraday} e diários, conforme mencionado na Seção \ref{coleta_de_dados}.

\paragraph{} A figura \ref{fig:103} mostra o ERD\footnote{\textit{Entity-Relationship Diagram}. Em português: Diagrama de Entidade Relacionamento.} do banco. Os \textit{scripts} de criação e população inicial do banco de dados pode ser encontrado em \cite{github_projeto}. \color{red} HERALDO: (1) Vale a pena mostrar o ERD do banco? (2) Devo mencionar as constraints de banco que garantem consistência, como por exemplo a comparação do preço máximo de um candle com seus outros valores a fim de garantir que este de fato é máximo? Preços não negativos, valores não nulos, etc. \colorend

\begin{figure}[h]
    \includegraphics[scale=0.90]{no_image.jpeg}
    \centering
    \caption{ERD do Banco de Dados}
    \label{fig:103}
\end{figure}



\subsection{Geração de \textit{Features} de Uso Geral}

\paragraph{} As \textit{Features} de Uso Geral são características derivadas dos \textit{candlesticks} que podem auxiliar qualquer decisão interna de uma estratégia, porém seu objetivo principal está no suporte à escolha do momento de entrada apropriado nas operações, o que é fundalmentalmente a responsabilidade do modelo de \textit{Machine Learning}. Como podem ser utilizadas por qualquer estratégia, são calculadas antes do início das simulações e somente quando há necessidade, ou seja, quando os \textit{candlesticks} são inseridos pela primeira vez no banco ou quando são atualizados. Ao final dos cálculos, são armazenadas nas tabelas de \textit{features} para posteriores consultas durante as simulações.

\paragraph{} Nesta etapa do projeto, assim como em diversas outras, faz-se necessário atenção e cuidados quanto a erros de não-causalidade, que mesmo sendo sutis, podem influenciar drasticamente os resultados finais. \color{red} HERALDO: Devo omitir esse parágrafo? A intenção é dizer que o autor não subestima essa problema, portanto tomou cuidados a nível de implementação para evitá-lo. Por outro lado, talvez essa preocupação já esteja subentendida, sendo desnecessário enfatizá-la. \colorend

\paragraph{} As \textit{features} de Uso Geral utilizadas são:

\begin{itemize}
    \item \sout{Média Móvel Exponencial de 17 períodos no gráfico diário.} \\
    \color{red} HERALDO: Feature original do André Moraes. Não utilizado nos modelos. \colorend

    \item \sout{Média Móvel Exponencial de 72 períodos no gráfico diário.} \\
    \color{red} HERALDO: Feature original do André Moraes. Não utilizado nos modelos. \colorend

    \item \sout{Média Móvel Exponencial de 72 períodos no gráfico semanal.} \\
    \color{red} HERALDO: Feature original do André Moraes. Não utilizado nos modelos. \colorend

    \item \sout{\textit{Flag} de Tendência de Alta.} \\
    \color{red} HERALDO: Feature original do André Moraes. Não utilizado nos modelos DA FORMA QUE O ANDRÉ USA, por isso o risco no texto. Ele se baseia em picos e vales ascendentes para justificar uma tendência de alta (de acordo com a teoria de Dow). A nível de código, a estratégia mais simples é comparar os últimos 2 picos de mínimo e 2 picos de máximo, verificar se um é maior que o outro com alguma margem de tolerância, que se caracteriza tendência de alta. Porém essa informação é ruidosa, principalmente em mercados em consolidação (=muito lateralizados). O que ocorre na prática do André, é que não são só os ultimos 4 picos que são analisados. Primeiro, ele já tira uma noção visual se o mercado anda em consolidação, isso requer olhar uma quantidade variável de picos que possuem algum grau de proximidade nas magnitudes, mas se comportam entre linhas de suporte e resistência. Quantos dias passados devo olhar para medir a consolidação? (retórica rs). Se uma ação vem em tendência de baixa, por exemplo, ele não espera exatamente o par perfeito de 4 picos ascendentes para dizer se o mercado está em alta, muitas vezes porque o quarto pico ainda nem se formou consistentemente. Tudo isso se baseia em uma boa identificação de picos, que felizmente foi implementada, porém tem um atraso mínimo de 9 dias para identificar qualquer pico. Enfim, embora possível, não é trivial uma boa métrica fidedigna ao André, seguindo esta lógica. \colorend

    \item \sout{\textit{Stop Loss} no último pico de rompimento/reversão de tendência (Pressupõe preço de compra definido).} \\
    \color{red} HERALDO: Não utilizado mais pois era feature do André Moraes. Não utilizado nos modelos. Em particular, essa não é trivial de se calcular e foi uma das quais distanciou a simulação feita da estratégia real do André. Os picos relevantes que retratam a memória do mercado muitas vezes estão relacionados a acontecimentos notáveis, como crises financeiras, acidentes industriais, relatórios jurícos, escândalos, aquisições novas, etc. O André pode não avaliar os acontecimentos menores na escolha do stop por ser grafista e não estar inteiramente ligado nas notícias, mas leva em consideração os mais marcantes. Como análise de notícias está completamente fora do escopo, obter "grau de importância" de um pico com alguma precisão requer no mínimo olhar os dados intraday (seção de coleta de dados explica porque não usei) e avaliar o volume de negociações na regiões. Contudo, idealmente deve-se olhar o livro de ofertas para tirar métricas das forças de compra e de venda, talvez aliadas ao volume de negociação e assim obter um valor razoável. Quando implementei na tentativa de simular o André, usei simplemente o pico mais próximo abaixo do preço de compra, dado uma distância mínima de 1\%. \colorend

    \item \textbf{\textit{Flag} de Identificação de Crises} \\ \\
        \textit{Flag} criado para prever crises financeiras através da identificação de anomalias nos volumes de negociação. Seu objetivo é impedir que o modelo de ML entre em qualquer operação durante sua presença. Para isso, utiliza-se a média \begin{math} \overline{V} \end{math} e o desvio padrão \begin{math} \sigma_V \end{math} do volume de negociações dos últimos 60 dias úteis, junto com o volume \begin{math} V \end{math} do dia corrente. Adiciona-se um efeito de inércia de 2 dias úteis consecutivos para ativar uma anomalia de volume e outra inércia de 8 dias úteis para persistência do \textit{flag}. As Equações \ref{eq:20} e \ref{eq:20} mostram o cálculo e a Figura \ref{fig:104} representa um estudo de caso para mostrar a eficácia o \textit{flag}.

        \begin{equation} \label{eq:20}
            V_{anomaly(i)} = \begin{cases} 1, & \mbox{se } V_{(i)} \geq \overline{V} + \sigma_V \quad \textrm{e} \quad V_{(i-1)} \geq \overline{V} + \sigma_V \\ 0, & \mbox{caso contr\'ario} \end{cases}
        \end{equation}

        \begin{equation} \label{eq:21}
            F_{crisis(i)} = \begin{cases} 1, & \mbox{se } V_{anomaly(i)} = 1 \\ 1, & \mbox{se } V_{anomaly(i)} = 0 \quad e \quad F_{crisis(i-1)} = 1 \quad \textrm{(At\'e 8 vezes consecutivas)} \\ 0, & \mbox{caso contr\'ario} \end{cases}
        \end{equation}

        \begin{figure}[h]
            \includegraphics[scale=0.70]{no_image.jpeg}
            \centering
            \caption{\textit{Flag} de Identificação de Crises para a ação XYZ no período de XX/YY/ZZZZ a XX/YY/ZZZZ}
            \label{fig:104}
        \end{figure}

    \item \textbf{\textit{Flag} de Tendência de Baixa} \\ \\
    Semelhante ao \textit{Flag} de Identificação de Crises, este \textit{Flag} também tem o objetivo de impedir entrada em operações pelo modelo de ML durante sua presença. No entanto, o critério é diferente. Primeiro, calcula-se a derivada dos preços médios \begin{math} \dot{P_{mid}} \end{math} entre o dia corrente e o anterior normalizado pela média dos preços médios (Equações \ref{eq:22} e \ref{eq:23}). Em seguida, ajusta-se um filtro digital IIR passa-baixas de coeficiente de amortecimento \begin{math} \alpha \end{math} (Equação \ref{eq:24}). Por fim, acrescenta-se um efeito de inércia de 3 dias úteis consecutivos para persistência do \textit{flag} em caso de ocorrência (Equação \ref{eq:25}). \color{red} HERALDO: Devo adicionar na fundamentação um tópico falando sobre filtro digital IIR passa-baixas e mostrando de onde vem essa fórmula? \colorend

    \begin{equation} \label{eq:22}
        P_{mid} = \dfrac{P_{open} + P_{close}}{2}
    \end{equation}

    \begin{equation} \label{eq:23}
        \dot{P_{mid(i)}} = \dfrac{ P_{mid(i)} - P_{mid(i-1)} }{ \dfrac{1}{2}(P_{mid(i)} + P_{mid(i-1)}) }
    \end{equation}

    \begin{equation} \label{eq:24}
        \dot{P_{mid\_LPF(i)}} = \alpha \dot{P_{mid(i)}} + (1 - \alpha) \dot{P_{mid\_LPF(i-1)}},  \quad \textrm{onde} \quad 0 \le \alpha \le 1
    \end{equation}

    \begin{equation} \label{eq:25}
        F_{downtrend(i)} = \begin{cases} 1, & \mbox{se } \dot{P_{mid\_LPF(i)}} < 0 \\ 1, & \mbox{se } \dot{P_{mid\_LPF(i)}} \geq 0 \quad e \quad F_{downtrend(i-1)} = 1 \quad \textrm{(At\'e 3 vezes consecutivas)} \\ 0, & \mbox{caso contr\'ario} \end{cases}
    \end{equation}

    Foi utilizado \begin{math} \alpha = 0.10 \end{math}, pois trantando-se de \textit{flag} que pode impedir diretamente o fluxo de negociações, faz-se mais necessário um baixo ruído à inércia da medida. \\

    \item \textbf{Risco Mínimo} \\ \\
    O Risco Mínimo serve de suporte à escolha do risco de entrada em uma operação, não sendo assim consumido diretamente pelo modelo de ML. Ressalta-se que o conceito de risco no escopo deste trabalho está relacionado à diferença percentual que o \textit{stop loss} é colocado abaixo do preço de compra. Sua fórmula é composta por uma parcela fixa somada a uma parcela variável, conforme mostrado pela Equação \ref{eq:30}.

    \begin{equation} \label{eq:30}
        Risk_{min} = Risk_{min_f} + Risk_{min_v}
    \end{equation}

    Seja \begin{math} \Delta \end{math} a diferença entre o preço máximo e mínimo de um \textit{candlestick} (Equação \ref{eq:31}), pode-se definir \begin{math} Risk_{min_f} \end{math} como o valor mínimo de risco necessário para superar as oscilações diárias dos preços dos últimos 20 dias úteis (Equação \ref{eq:32}).

    \begin{equation} \label{eq:31}
        \Delta = P_{high} - P_{low}
    \end{equation}

    \begin{equation} \label{eq:32}
        Risk_{min_f} = \dfrac{ \sigma_{\Delta} }{ P_{mid} }
    \end{equation}

    A parcela variável \begin{math} Risk_{min_v} \end{math} representa o inverso da derivada de preços ajustada por um filtro digital IIR passa-baixas (Equações \ref{eq:24} e \ref{eq:33}), onde \begin{math} \alpha \end{math} é o coeficiente de amortecimento.

    \begin{equation} \label{eq:33}
        Risk_{min_v} = - \dot{P_{mid\_LPF(i)}}
    \end{equation}

    Diferentemente do \textit{flag} de Tendência de Baixa, foi utilizado \begin{math} \alpha = 0.30 \end{math}, uma vez que neste caso é muito mais interessante uma resposta rápida do que um baixo ruído. \\

    Por fim, adicionou-se um segundo filtro de passa-baixas de \begin{math} \alpha = 0.10 \end{math} apenas aos movimentos de descida dos valores de \begin{math} Risk_{min} \end{math} a fim de se aumentar a cautela apenas durante momentos mais turbulentos do mercado.

    \item \textbf{Risco Máximo} \\ \\
    O Risco Máximo serve de suporte à escolha do risco de entrada em uma operação, não sendo assim consumido diretamente pelo modelo de ML. Ressalta-se que o conceito de risco no escopo deste trabalho está relacionado à diferença percentual que o \textit{stop loss} é colocado abaixo do preço de compra. A escolha risco também implica diretamente no valor do preço alvo de uma operação, pois o mesmo é definido como 3 vezes a magnitude do risco escolhido, percentualmente acima do preço de compra.

    A ideia central está na análise estatística das oscilações de vale para crista entre os últimos picos no gráfico diário.

\end{itemize}

% \begin{math}R_b\end{math}
% \color{red} HERALDO: \colorend

\section{Simulação de Estratégia}

\subsection{Estrutura}
\paragraph{} EXCLUIR - Carteira com N ativos de datas distintas; Regra de 3 para 1 entre stop e alvo; 1 operação por ativo.

\subsection{Premissas}
\paragraph{} EXCLUIR - Compra na abertura do merdado; Sem venda no dia da compra; Prioridades durante venda (stop primeiro).

\subsection{Período Máximo de Dias por Operação}
\paragraph{} EXCLUIR - Motivação da escolha dos 45 dias; Gráfico entre ABEV e MGLU.

\subsection{Gerenciamento de Risco}
\paragraph{} EXCLUIR - Coeficiente de Risco-Capital

\subsection{Risco de Entrada por Operação}
\paragraph{} EXCLUIR - Cálculo do risco mínimo; Cálculo do risco Máximo.

\subsection{Descanso por Tendência de Baixa}
\label{downtrend_halt}
\paragraph{}

\subsection{Descanso por Identificação de Crises}
\label{crisis_halt}
\paragraph{}

\subsection{Lista de Parâmetros de Configuração}
\paragraph{} EXCLUIR - Lista todos e explicar o que fazem.

\subsection{Ensaios Paralelos}
\paragraph{} EXCLUIR - Parâmetros que estão implementados e não trouxeram resultados expressivos.



\section{Otimizações de Gerenciamento de Carteira}

\subsection{Normalização por Frequência de Operações}
\label{freq_norm}
\paragraph{}

\subsection{Compensação por Lucratividade}
\label{profit_comp}
\paragraph{}

\subsection{Controle Proporcional para Uso de Capital}
\label{dynamic_rcc}
\paragraph{}




\section{Criação de Modelos}

\subsection{Resumo}
\paragraph{}

\subsection{\textit{Feature Selection}}
\paragraph{}

\subsection{Geração de \textit{Datasets}}
\paragraph{}

\subsection{\textit{Walk Forward Optimization}}
\paragraph{}

\subsection{Critérios de Escolha}
\paragraph{}


\section{Análise de Resultados}
\paragraph{} EXCLUIR - Dashboard; Baseline