\chapter{Introdução}
\label{cap1}



\section{Tema}

\paragraph{} O tema deste trabalho se resume na elaboração de uma estratégia de \textit{swing trade} na bolsa de valores brasileira através de métodos de aprendizado de máquina.

\paragraph{} Nesse contexto, o problema a ser abordado é a identificação do momento apropriado para compra de um determinado ativo, como também os preços alvos determinantes para venda, tendo em vista uma variação positiva de seu preço.



\section{Delimitação}

\paragraph{} Este trabalho se limita aos ativos negociados na Bolsa de Valores de São Paulo, a B3, de cujos dados diários de domínio público foram adquiridos através da plataforma \textit{Yahoo Finance} pela API \textit{open-source yfinance}, disponível em Python. Não são levadas em consideração informações sobre proventos (dividendos e juros sobre capital próprio) devido à inconsistência dos mesmos na API supracitada, aliada à dificuldade técnica para automatização da busca de tais dados.

\paragraph{} A duração das operações tem em vista um horizonte mínimo de um dia, sendo portanto operações de \textit{swing trade}. Não são realizas vendas a descoberto\footnote{Venda a descoberto (\textit{short selling}): Venda de ações anterior a compra das mesmas através de um contrato de aluguél. O lucro ocorre na queda de preço do mercado.}, portanto só há lucro em variações positivas de preço dos ativos. Apenas uma operação por ativo pode existir em um determinado instante de tempo para uma estratégia. Em outras palavras, só é possível comprar mais ações de uma empresa após a venda completa das ações da mesma, caso existam.

\paragraph{} A incidência de impostos devidos (e.g., imposto de renda) está fora do escopo. Assim como a utilização de critérios baseados em análise fundamentalista, por causa da dificuldade de obtenção dessas informações de maneira automatizada e estruturada.



\section{Justificativa}

\paragraph{} O crescimento do número de investidores na bolsa de valores brasileira \cite{aumento_investidores} demonstra um maior interesse da população na busca por um complemento da renda familiar ou até na substituição da fonte de renda principal.

\paragraph{} No cenário global, o aumento do uso de robôs de \textit{trading} (ou algoritmos) tem se mostrando expressivo \cite{robos_investidores}, seja por pessoas físicas ou fundos de investimento, de forma total ou parcial em suas estratégias. Por outro lado, tal crescimento não vem sendo igualmente representado no Brasil devido às pecualiaridades do mercado de capitais nacional, como a alta volatilidade e a alta sensibilidade a notícias \cite{robos_e_fundos}.

\paragraph{} Paralelamente, estudos relacionados a aprendizado de máquina vem trazendo resultados práticos no dia-a-dia das pessoas, desde o clássico exemplo de reconhecimento de mensagens de \textit{spam} em um caixa de email à identificação do perfil de consumo de clientes em uma loja. Da mesma forma, instituições financeiras e bancos centrais também estão, com cautela, incorporando aplicações de aprendizado de máquina em tarefas internas \cite{fernandez2019artificial}.

\paragraph{} Apesar das dificuldades inerentes ao cenário atual do mercado de capitais brasileiro, não se pode ignorar o potencial que os algortimos podem trazer. Desta forma, o presente trabalho visa a união de técnicas de aprendizado de máquina a práticas de \textit{trading}, consolidando uma estratégia que sirva de suporte a uma maior variedade de opções de investimento à população brasileira.



\section{Objetivos}

\paragraph{} O objetivo geral deste trabalho é implementar um \textit{software} capaz de simular uma estratégia de \textit{swing trade} utilizando aprendizado de máquina. Especificamente, o software deve: (1) Criar um ambiente automatizado que permita buscar, atualizar e armazenar dados diários da bolsa brasileira de forma simples e conforme necessidade do usuário da aplicação; (2) Criar a estrutura de uma estratégia por meio de um conjunto de regras e premissas baseadas em práticas de \textit{trading}; (3) Gerar os modelos de aprendizado de máquina e acoplá-los à estrutura criada; (4) Simular a estratégia obtida; (5) Criar um mecanismo de fácil visualização dos resultados das simulações; (6) Analisar os resultados gerados.




\section{Metodologia}

\paragraph{} O trabalho tem início na criação de um ambiente propício à simulação de estratégias, bem como sua configuração e manutenção. Para isso, a fim de: otimizar o tráfego de dados pela internet; minimizar o processamento necessário para a geração de dados derivados (pré-processamento); e armazenar os resultados das estratégias de forma organizada, foi utilizado um banco de dados PostgreSQL. Dentre as atividades realizadas durante o pré-processamento dos dados, anteriores à simulação, é possível citar a identificação de momentos de tendência de baixa e crises do mercado, a identificação de picos na série histórica e os valores de risco mínimo e máximo a serem utilizados nas operações.

\paragraph{} Em seguida, a etapa de simulação começa na leitura de um arquivo JSON contendo todos parâmetros necessários para a execução das estratégias. Nesta etapa, o programa itera dia após dia para cada estratégia configurada verificando os momentos e os valores de compra e de venda para cada ativo que compõe as carteiras. Ao final, registram-se no banco todas as operações executadas, independente da obtenção de lucro, junto com as informações estatísticas necessárias para a avaliação da performance.

\paragraph{} Por fim, com o objetivo de facilitar a análise dos resultados gerados, criou-se um \textit{dashboard} resposável por centralizar todas as informações pertinentes a uma execução de estratégia em uma única página web.

\paragraph{} Observa-se que além do uso de estruturas do banco de dados PostgreSQL, como \textit{triggers} e \textit{functions}, o código foi construído em Python devido à ampla variedade de bibliotecas e ao suporte da comunidade, apesar da desvantagem de desempenho por ser uma linguagem interpretada. Bastante foco foi dado à escalabilidade e à manutenção do código, que contou com as bibliotecas e as APIs \textit{yfinance}, \textit{pandas}, \textit{numpy}, \textit{scikit-learn}, \textit{multiprocessing}, \textit{matplotlib} e \textit{dash}. Também utilizou-se \textit{containers} Docker para simplificar a execução.



\section{Descrição}

\paragraph{} No capítulo 2 é desenvolvida a fundamentação teórica acerca de temas relevantes ao entendimento geral de mercado financeiro e de aprendizado de máquina. Também é realizada uma revisão dos trabalhos relacionados ao tema, em outras palavras, a revisão bibliográfica.

\paragraph{} No capítulo 3, toda a metodologia é descrita, o que envolve o pré-processamento dos dados as simulações e a criação dos modelos de ML.

\paragraph{} Por fim, o capítulo 4 encerra com a conclusão e uma recomendação de trabalhos futuros.