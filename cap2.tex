\chapter{Fundamentação Teórica}
\label{cap2}

\paragraph{} Neste capítulo, são introduzidos alguns conceitos chave para o entendimento do projeto. Nas próximas seções, são feitas contextualizações sobre o Mercado de Capitais, Bolsa de Valores, Ações e Aprendizado de Máquina.

\section{Mercado de Capitais, Bolsa de Valores e Ações}

\paragraph{} O Mercado de Capitais, também conhecido como Mercado de Valores Mobiliários, é um dos segmentos do sistema financeiro responsável por fazer o intermédio entre agentes superávitarios, que tem capital de investimento, e agentes deficitários, que buscam capital para rentabilizá-lo, através da compra e venda valores mobiliários (i.e., ativos financeiros) \cite{mercado_de_capitais}. Consequentemente, gera-se uma maior liquidez destes ativos e também uma melhora no fluxo de capitais entre os agentes econômicos, sejam eles os governos por meio dos bancos centrais, os bancos privados, as insituições financeiras ou até mesmo as pessoas físicas.

\paragraph{} No Brasil, o Mercado de Capitais é regulado e fiscalizado pela CVM (Comissão de Valores Mobiliários), uma autarquia federal vinculada ao Ministério da Fazenda e criada em 1976 através da Lei nº 6.385 \cite{lei_6385}.

\paragraph{} A Bolsa de Valores é uma plataforma onde se negociam os valores mobiliários do Mercado de Capitais, dentre eles ações (i.e., fatias, pedaços) de sociedades anônimas (ou companhias). No Brasil, a única Bolsa de Valores oficial existente é a B3 (Brasil, Bolsa, Balcão) \cite{b3}, que administra os sistemas de negociação, compensação, liquidação, depósito e registro para todas as principais classes de ativos.

\paragraph{} O processo de abertura de capital de uma empresa é uma iniciativa que possui vantanges estratégicas \cite{vantagens_sa} como: o aumento da confiança na perspectiva do mercado, seja para o consumidor final ou para parceiros comerciais; a solução de problemas decorrentes de processos sucessórios; e também a captação de capital de investimento, a fim de contribuir para o crescimento ou para a consolidação da companhia. Esse processo acontece através de uma oferta pública \cite{oferta_publica}, ou IPO (Initial Public Offering), onde as ações que compõe o capital social \cite{capital_social} de uma companhia são vendidas pela primeira vez ao público geral. Uma vez encerrado o IPO, estas mesmas ações passam para o mercado secundário \cite{mercado_secundario}, onde investidores as negociam entre si. Em retorno ao capital adquirido pela companhia, surgem algumas responsabilidades, dentre elas a publicação de demonstrações financeiras \cite{dem_finan}, auditadas pela própria CVM \cite{audi_dem_finan}.

\paragraph{} Para o acionista de uma sociedade anônima, existem duas formas de se obter lucro: através de proventos (dividendos e juros sobre capital próprio) \cite{proventos}; ou através de operações de compra e de venda de ações, mediante oscilações de seu valor de mercado. Conforme a expectavida corretamente induz, o lucro é comumente aferido durante a venda de um determinado papél (i.e., ação) posteriormente à sua aquisição a um preço de compra inferior. No entando, também é possível trabalhar com posições vendidas (short selling) \cite{short_selling}, onde um investidor aluga ações de outro investidor por meio de um contrato. Em seguida as vende para posteriormente recomprá-las a um preço inferior, devolvendo-as assim ao respectivo dono. Neste caso, o lucro é obtido quando expectativa de queda de um ativo se mostra verdadeira.

\subsection{Hipótese do Mercado Eficiente}

\paragraph{} A Hipótese do Mercado Eficiente, definida por FAMA \cite{fama1970efficient}, afirma que idealmente o preço de um ativo reflete toda a informação disponível sobre seu valor intrínseco. Em outras palavras, quanto menor o efeito de fatores que contribuam para uma inércia no fluxo de capital de investidores e na transmissão de informações, mais o mercado tende a ser eficiente. São estudados os três níveis de hipóteses:

\begin{itemize}
    \item HME fraca: O preços atuais refletem o todo o histórico de informações disponibilizados publicamente.
    \item HME semi-forte: Engloba a HME fraca, acrescentando a existência de uma mudança instantânea que os preços sofrem ao surgirem novas informações.
    \item HME forte: Engloba a HME semi-forte, porém entende que a mudança instantânea dos preços acompanha toda e qualquer informação existente sobre o ativo. Assim, absolutamente nenhum investidor conseguiria obter lucro superior à média do mercado, pois não há como acessar nenhuma informação privilegiada, uma vez que ela já estaria refletido no preço corrente do ativo.
\end{itemize}

\paragraph{} O autor menciona que o HME forte não é estritamente válida na realidade, o que é uma afirmação coerente quando se verifica a existência de casos em que o vazamento de informações confidenciais trouxe aos acusados uma lucratividade significativa \cite{insider_trading}.

\paragraph{} A HME fraca foi verificada devido à consistência da correlação dos preços dia após dia de determinadas ações, mesmo que esta fosse baixa.

\paragraph{} A hipótese semi-forte também foi sustentada por alguns fatores, dentre eles a verificação de que os futuros pagamentos de dividendos das companhias se refletem, em média, no preços das ações \cite{fama1969adjustment}.

\paragraph{} Em resumo, o estudo das Hipóteses de Mercado Eficiente traz informações relevantes quanto se avalia a teoria por trás da possibilidade de aplicação de estratégias de trading no mercado financeiro. No entanto, é importante ressaltar que outros autores questionam ao menos parcialmente os estudos realizados por FAMA, sejam por resultados inconclusivos ou por anomalias detectadas no comportamento do mercado. Por exemplo, SHOSTAK \cite{shostak1997defense} critica abertamente a premissa de que todos os investidores teriam a mesma expectativa sobre os retornos da empresa. O ganhador do prêmio Nobel em ciências econômicas Paul Samuelson, que afirma que o a HME funciona muito melhor para ações individuais do que para o mercado como um todo \cite{jung2005samuelson}. Já o investidor Jack Schwager afirma que a HME está correta pelos motivos errados \cite{schwager2012market}, pois é muito difícil bater a média do mercado de forma consistente ao mesmo tempo que investidores possuem habilidades diferentes, portanto a informação não é interpretada e aplicada por todos da mesma forma.

\subsection{Índice de Bolsa de Valores}

Índices de Bolsas de Valores \cite{stock_index} são métricas criadas para avaliar a saúde de um determinado grupo de ações negociadas na bolsa. Cada índice possui uma regra própria de criação que define quais ações são englobadas e com quais pesos, como por exemplo:

\begin{itemize}
    \item S\&P 500: Um dos mais conhecidos no mercado. É a média ponderada pelo capital social das 500 maiores companhias do mercado americano.
    \item Dow Jones Industrial Average: É a média ponderada pelo preço da ação das 30 maiores blue-chips industriais e financeiras do mercado americano (i.e., companhias bem conhecidas, bem estabelecidas e com grande capital social).
    \item Ibovespa: Principal indicador de desempenho do mercado brasileiro. Possui alguns critérios específicos, mas basicamente é composto pelas ações com maior volume de negociação na B3 \cite{ibovespa}.
\end{itemize}

\paragraph{} Índices não são negociáveis pois não passam de métricas de mercado. Para isso existem fundos de investimentos chamados ETFs (Exchange-Traded Funds) \cite{etf}, especializados em seguir um determinado índice.

\paragraph{} No Brasil, um investidor que deseja que uma parte de seu capital acompanhe um rendimento equivalente ao iBovespa deverá investir no ETF, cujo código de negociação é BOVA11.

\subsection{Mercado Fracionário}

\paragraph{} Ações são negociadas em múltiplos de um lote, que representa uma quantidade mínima de papéis a transacionar. Nesse contexto, o Mercado Fracionário \cite{mercado_fracionario} surge com o objetivo de facilitar negiociações de volumes menores que o lote mínimo permitido. Na prática, ações fracionárias são agrupadas até formarem um lote para então serem negociadas. Normalmente o Mercado Fracionário possui menor liquidez e maior volatilidade, mas sempre acompanha o preço do ativo negociado no mercado aberto.

\paragraph{} Ações fracionárias podem ser criadas devido: a um desdobramento de ações que não gera resultado par (e.g., 3 para 2); ou a fusões e aquisições de empresas que combinam suas ações em uma razão predeterminada.

\paragraph{} Grandes investidores e fundos de investimentos não possuem problemas quanto ao capital mínimo necessário para a compra de um lote de ações, visto que negociam em quantidades muito maiores. O problema surge quando um investidor com pouco aporte financeiro deseja entrar no mercado e não consegue encontrar ativos cujo lote mínimo esteja dentro de seu orçamento.

\paragraph{} No Brasil, o lote mínimo é de 100 ações e o Mercado Fracionário permite a compra de no mínimo 1 ação.

\section{Tipos de Análises}

\subsection{Análise Fundamentalista}
\label{subsection:af}

\paragraph{} A Análise Fundamentalista (AF) é um muito utilizada para identificar tendências de flutuação no preço de ações tendo em vista um horizonte de longo prazo \cite{bulkowski2012fundamental}. Ela se baseia em fatores econômicos relacionados à companhia, como: o quadro de diretores e dirigentes maiores; o fluxo de caixa; a saúde e a situação financeira; o contexto político do país; os concorrentes de mercado; as circunstâncias climáticas; os desastres climáticos, naturais ou não, dentre outros fatores.

\paragraph{} Devido à natureza desorganizada e desestruturada dos dados que representam os fatores mencionados, torna-se muito difícil implementar uma automação.

\subsection{Análise Técnica}
\label{subsection:at}

\paragraph{} A Análise Técnica (AT) busca identificar tendências de curto prazo na série temporal de preços de ações através da identificação de padrões e da criação de informações derivadas (indicadores técnicos) \cite{murphy1999technical, edwards2018technical}. Em teoria, o preço das ações é consequência de todos os acontecimentos relacionados direta ou indiretamente a uma companhia, assim como as análises feitas por investidores grandes.

\paragraph{} Diferentemente da AF, a automação desta análise é muito mais fácil pois os dados normalmente são organizados e estruturados. No entanto, como são obtidos a posteriori, a dificuldade desta análise se dá na separação entre o que é ruído e o que é de fato tendência de mercado, além da criação de informações derivadas que se mostram relativamente úteis.

\section{Aprendizado de Máquina}

\paragraph{} Aprendizado de Máquina (Machine Learning) é um campo de estudo dentro de Ingeligência Artificial \cite{ibm_ai} que engloba estatística e ciência da computação. O objetivo é extrair conhecimento a partir de uma conjunto de dados \cite{muller2016introduction}. A terminologia foi criada por um pesquisador da IBM chamado SAMUEL em 1959 \cite{ibm_ml} para um estudo de caso do jogo de damas \cite{arthur1959some}.

\paragraph{} Em geral, algoritmos de ML buscam realizar tarefas extremamente complexas computacionalmente sem serem explitamente programadas caso a caso. Alguns exemplos de aplicações que deixam evidente os benefícios deste método são: visão computacional, recomendação de produtos, identificação de rosto em imagens, identificação transações financeiras fraudulentas, suporte a diagnósticos médicos, dentre diversos outros.

\paragraph{} Algoritmos de ML podem ser baseados em Aprendizado Supervisionado, Aprendizado Não Supervisionado ou até mesmo um modelo híbrido. Este trabalho utiliza apenas AS para a criação de modelos.

\subsection{Aprendizado Supervisionado}

\paragraph{} Uma das metodologias mais comuns de ML, seu objetivo é a predição de um resultado a partir de um conjunto de dados de entrada, com a condição de que o modelo tem acesso a vários exemplos de entrada e saída de dados para uma melhor performance \cite{muller2016introduction}. Existem dois tipos de problemas associados ao AS, os problemas de Regressão e os problemas de Classificação.

\subsection{Problema de Regressão}

\subsection{Problema de Classificação}

\section{Trabalhos Relacionados}

\paragraph{} Tendo em vista o conflito de interesses existente por trás de trabalhos de cujo tema está relacionado à previsibilidade do mercado financeiro, pode-se questionar se as estratégias mais promissoras de fato são encontradas em domínio público. Isso ocorre pois a democratização de uma estratégia lucrativa poderia implicar na redução das lucratividades individuais, especialmente se for utilizada em escala.

\paragraph{} Segundo KIM \cite{kim2010electronic}, somente a partir dos anos 80 que as corretoras começaram a utilizar protocolos de comunicação eletrônica para substituir a corretagem por voz. Essa inovação permitiu o desenvolvimento do Algorithmic Trading, que é a automatização da tomada de decisões de estratégias por um computador capaz de enviar ordens de compra e venda diretamente ao mercado.

\paragraph{} Para efeito de simplificação, as estratégias de AT aplicadas ao mercado financeiro serão agrupadas em três metodologias centrais: modelos baseados em indicadores técnicos; modelos baseados em processos estocásticos; e modelos baseados em inteligência artificial.

\subsection{Modelos Baseadas em Indicadores Técnicos}

\paragraph{} Este tipo de abordagem utiliza informações derivadas da série temporal de preços para criar uma combinação de indicadores que possuam algum poder de previsibilidade da tendência de mercado. Quando comparada aos outros tipos, é a metodologia mais simples e democrática, uma vez que pessoas com pouco ou nenhum conhecimento sobre estatística e inteligência artificial podem operar em estratégias próprias.

\paragraph{} Dentre os indicadores mais famosos e portanto utilizados, podemos citar: o volume financeiro; a identificação de tendências de alta, de baixa e de consolidação de acordo com a Teoria de Dow \cite{kirkpatrick2022dow}; as linhas de suporte e resistência do mercado; as médias móveis; as bandas de Bollinger \cite{bollinger2002bollinger}; e o MACD (Moving Average Convergence-Divergence) \cite{appel2007understanding}.

\paragraph{} Citar MORAES \cite{moraes2007se}.

\subsection{Modelos Baseados em Processos Estocásticos}

\paragraph{} De acordo com GODFREY \cite{godfrey1964random}, a hipótese de que a flutuação de preços no mercado de ações poderia ser explicada por uma Random Walk\footnote{Processo aleatório definido pela equação \begin{math}y_t = y_{t-1} + X\end{math}, onde X é uma variável aleatória e \(y\) é a variável resultante.} foi feita por BACHELIER \cite{bachelier1900theorie}. A partir da década de 60, muitos trabalhos acadêmicos foram realizados nessa linha na tentativa de entender o comportamento e a previsibilidade do mercado \cite{fama1970efficient, solnik1973note, cooper1982world}, assim como estratégias \cite{malkiel2019random}. Nota-se que até hoje utiliza-se Random Walks para testar a hipótese de eficiência de mercados \cite{said2015efficiency}.

\paragraph{} Outra abordagem utilizada são os Modelos Ocultos de Markov (do inglês Hidden Markov Model, ou HMM) \cite{rabiner1989tutorial}. Uma Cadeia de Markov é um processo estocástico que modela um sistema por meio de uma sequência finita de estados. A mudança ou a permanência em cada estado é determinada por probabilidades que dependem somente do estado atual. Em uma Cadeia de Markov, pressupõe-se que seus estados sejam observáveis, o que para algumas aplicações, pode não ser verdade. Nesse sentido surge o modelo HMM, que busca aprender sobre um processo não observável (oculto) a partir de um processo observável.

\paragraph{} Em sua pesquisa, JADHAV et al \cite{jadhav2021forecasting} utiliza um modelo HMM para previsão do preço de fechamento do dia seguinte para ações FAANG\footnote{Facebook, Amazon, Apple, Netflix, Google}. A partir da série histórica de preços OHLC\footnote{Open, High, Low, Close. Em português: Abertura, Máximo, Mínimo, Fechamento}, seu modelo atinge uma eficiência de 97\%-99\%, calculado a partir do erro percentual absoluto médio\footnote{Mean Absolute Percentage Error (MAPE): \begin{math} \frac{1}{N}\sum_{i=1}^{N} \frac{|Predicted(i)-Actual(i)|}{Actual(i)} \end{math}}.

\paragraph{} Uma outra aplicação de modelos HMM é dada por DE ANGELIS et al \cite{de2013dynamic}, que criou uma metodologia a partir de índices da bolsa americana capaz de identificar períodos estáveis e instáveis (i.e. crises econômicas), assim como as probabilidades de transição entre um estado e o outro.

\paragraph{} Por fim, pode-ser mencionar o uso de modelos ARCH\footnote{Em português: Heteroscedasticidade Condicional Auto-regressiva} (Autoregressive Conditional Heteroskedasticity). A ideia central está na modelagem de uma variância (volatilidade) condicional, ou seja, que muda de acordo o instante da série \cite{enders2008applied}. Essa característica se faz muito útil em séries que possuem períodos de alta volatilidade se alternando com períodos de baixa volatilidade. Para um modelo genérico ARCH(q), seja \begin{math}\epsilon_t\end{math} o erro (resíduo) no instante t e \begin{math}\alpha_0\end{math} um ruído branco, pode-se descrever a variância condicional de acordo com a Equação \ref{eq:1}.

\begin{equation} \label{eq:1}
\sigma_t^2 = \alpha_0 + \sum_{i=1}^{q} \alpha_i\epsilon_{t-1}^{2}
\end{equation}

\paragraph{} O modelo ARCH foi proposto por ENGLE em 1982 para estimar a variância da inflação do Reino Unido \cite{engle1982autoregressive}. A partir daí, várias derivações surgiram, como por exemplo: GARCH \footnote{Generalised ARCH} por BOLLERSLEV \cite{bollerslev1986generalized} em 1986, EGARCH \footnote{Exponential Generalised ARCH} por NELSON em 1991, NGARCH \footnote{Non-linear Generalised ARCH} por HIGGINS e BERA\cite{higgins1992class} em 1992, TGARCH \footnote{Threshold Generalised ARCH} por ZAKOIAN e RABEMANANJARA \cite{rabemananjara1993threshold} em 1993, dentre outros. Alguns dos modelos da família ARCH podem ser encontrados nos trabalhos de FRANSES e DIJK \cite{franses1996forecasting}, de MARCUCCI \cite{marcucci2005forecasting} e de ALBERG et al \cite{alberg2008estimating}.

\subsection{Modelos Baseados em Inteligência Artificial}

\paragraph{} Existem registros de estudos sobre inteligência artificial aplicados ao mercado financeiro por volta da década de 70 \cite{felsen1975artificial}, porém ainda em um estágio embrionário devido às dificuldades de acesso à informação e de processamento computacional na época. Conforme ambos os fatores foram evoluindo, a dificuldade de se testar algoritmos cada vez mais complexos diminuiu.

\paragraph{} Combinações de GARCH com ANN \cite{bildirici2009improving}